\section{Interface Description}
%==============================
The model is called from the \texttt{Compute\_MW\_SfcOptics()} functions in the \texttt{CRTM\_MW\_Water\_SfcOptics} module. The main source module is \texttt{CRTM\_LowFrequency\_MWSSEM} and it contains the public entities shown in table \ref{tab:main_procedures_list}. Note that the internal variable structure is usable but not accessible outside the \texttt{CRTM\_LowFrequency\_MWSSEM} module.
\begin{table}[htp]
  \centering
  \begin{tabular}{|c|c|}
    \hline
    \textbf{Name} & \textbf{Description} \\
    \hline\hline
    \multicolumn{2}{|c|}{\textbf{Data types}}\\
    \hline
    \texttt{iVar\_type}               & Internal variable struture \\
    \hline
    \multicolumn{2}{|c|}{\textbf{Subroutines}}\\
    \hline
    \texttt{LowFrequency\_MWSSEM}     & Forward model \\
    \texttt{LowFrequency\_MWSSEM\_TL} & Tangent-linear model \\
    \texttt{LowFrequency\_MWSSEM\_AD} & Adjoint model \\
    \hline
  \end{tabular}
  \caption{List of public procedures in the \texttt{CRTM\_LowFrequency\_MWSSEM} module}
  \label{tab:main_procedures_list}
\end{table}

The interface and argument descriptions for the forward model are shown in figure \ref{fig:fwd_interface}.

The interface and argument descriptions for the tangent-linear model are shown in figure \ref{fig:tl_interface}. Note that the ``internal variable'' argument, \texttt{iVar}, is now an input as this structure contains intermediate forward model variables computed within the forward model. Also note that there are no frequency and zenith angle tangent-linear inputs. This model does not compute sensitivities of the emissivity to those quantities.

The interface and argument descriptions for the adjoint model are shown in figure \ref{fig:ad_interface}. As with the tangent-linear interface, the internal variable argument, \texttt{iVar}, is an input. Note that if an argument is an input in the tangent-linear model, its corresponding adjoint argument is an output. Similarly, adjoint input arguments correspond with forward model output arguments. Note that although the adjoint emissivity is an input to the model, upon exiting the adjoint subroutine it is set to zero.

The temperature, salinity, and wind speed adjoints are all summed over the number of Stokes vector components as shown below,
\begin{eqnarray*}
  \textrm{Temperature\_AD} &=& \sum^{N}_i \frac{\partial e_i}{\partial T}\\
  \textrm{Salinity\_AD}    &=& \sum^{N}_i \frac{\partial e_i}{\partial S}\\
  \textrm{Wind\_Speed\_AD} &=& \sum^{N}_i \frac{\partial e_i}{\partial W}
\end{eqnarray*}
where $N$ is the number of Stokes vector components. Currently, this is fixed at 2 (vertical and horizontal polarisations only).
 
\begin{figure}[htp]
  \centering
  \doublebox{
  \begin{minipage}[b]{6.5in}
    \begin{ttfamily}
      \begin{verbatim}
  SUBROUTINE LowFrequency_MWSSEM( Frequency   , &  ! Input
                                  Zenith_Angle, &  ! Input
                                  Temperature , &  ! Input
                                  Salinity    , &  ! Input
                                  Wind_Speed  , &  ! Input
                                  Emissivity  , &  ! Output
                                  iVar          )  ! Internal variable output
    REAL(fp),        INTENT(IN)     :: Frequency
    REAL(fp),        INTENT(IN)     :: Zenith_Angle
    REAL(fp),        INTENT(IN)     :: Temperature
    REAL(fp),        INTENT(IN)     :: Salinity
    REAL(fp),        INTENT(IN)     :: Wind_Speed
    REAL(fp),        INTENT(OUT)    :: Emissivity(:)
    TYPE(iVar_type), INTENT(IN OUT) :: iVar
      \end{verbatim}
    \end{ttfamily}
    \centering
    \begin{tabular}{c|c|c|c}
      \textbf{Argument} & \textbf{Description}                    & \textbf{Units}   & \textbf{Intent} \\
      \hline\hline
      Frequency          & Microwave frequency                    & GHz              & IN      \\
      \hline
                         & Satellite zenith angle                 &                  &         \\
      \rb{Zenith\_Angle} & at the sea surface                     & \rb{Degrees}     & \rb{IN} \\ 
      \hline
      Temperature        & Sea surface temperature                & Kelvin           & IN      \\
      \hline
      Salinity           & Salinity of sea water                  & \textperthousand & IN      \\
      \hline
      Wind\_Speed        & Sea surface wind speed                 & m.s$^{-1}$       & IN      \\
      \hline
                         & The surface emissivity at vertical     &                  &         \\
      \rb{Emissivity}    & and horizontal polarizations           & \rb{N/A}         & \rb{OUT}\\
      \hline
                         & Structure containing internal          &                  &         \\
      iVar               & variables required for subsequent      & N/A              & OUT     \\
                         & tangent-linear or adjoint model calls. &                  &         \\
    \end{tabular}
  \end{minipage}
  }
  \caption{Forward model interface and argument description for the low frequency microwave sea surface emissivity model.}
  \label{fig:fwd_interface}
\end{figure}

\begin{figure}[htp]
  \centering
  \doublebox{
  \begin{minipage}[b]{6.5in}
    \begin{ttfamily}
      \begin{verbatim}
  SUBROUTINE LowFrequency_MWSSEM_TL( Frequency     , &  ! Input
                                     Zenith_Angle  , &  ! Input
                                     Temperature   , &  ! FWD Input
                                     Salinity      , &  ! FWD Input
                                     Wind_Speed    , &  ! FWD Input
                                     Temperature_TL, &  ! TL  Input
                                     Salinity_TL   , &  ! TL  Input
                                     Wind_Speed_TL , &  ! TL  Input
                                     Emissivity_TL , &  ! TL  Output
                                     iVar            )  ! Internal variable input
    REAL(fp),        INTENT(IN)  :: Frequency
    REAL(fp),        INTENT(IN)  :: Zenith_Angle
    REAL(fp),        INTENT(IN)  :: Temperature
    REAL(fp),        INTENT(IN)  :: Salinity
    REAL(fp),        INTENT(IN)  :: Wind_Speed
    REAL(fp),        INTENT(IN)  :: Temperature_TL
    REAL(fp),        INTENT(IN)  :: Salinity_TL
    REAL(fp),        INTENT(IN)  :: Wind_Speed_TL
    REAL(fp),        INTENT(OUT) :: Emissivity_TL(:)
    TYPE(iVar_type), INTENT(IN)  :: iVar
      \end{verbatim}
    \end{ttfamily}
    \centering
    \begin{tabular}{c|c|c|c}
      \textbf{Argument} & \textbf{Description}                    & \textbf{Units}   & \textbf{Intent} \\
      \hline\hline
      Frequency          & Microwave frequency                    & GHz              & IN      \\
      \hline
                         & Satellite zenith angle                 &                  &         \\
      \rb{Zenith\_Angle} & at the sea surface                     & \rb{Degrees}     & \rb{IN} \\ 
      \hline
      Temperature        & Sea surface temperature                & Kelvin           & IN      \\
      \hline
      Salinity           & Salinity of sea water                  & \textperthousand & IN      \\
      \hline
      Wind\_Speed        & Sea surface wind speed                 & m.s$^{-1}$       & IN      \\
      \hline
      Temperature\_TL    & Sea surface temperature perturbation   & Kelvin           & IN      \\
      \hline
      Salinity\_TL       & Salinity of sea water perturbation     & \textperthousand & IN      \\
      \hline
      Wind\_Speed\_TL    & Sea surface wind speed perturbation    & m.s$^{-1}$       & IN      \\
      \hline
                         & The surface emissivity perturbation at &                  &         \\
      \rb{Emissivity\_TL}& vertical and horizontal polarizations  & \rb{N/A}         & \rb{OUT}\\
      \hline
                         & Structure containing internal            &                &         \\
      \rb{iVar}          & variables. Output from the forward model.& \rb{N/A}       & \rb{IN} \\ 
    \end{tabular}
  \end{minipage}
  }
  \caption{Tangent-linear model interface and argument description for the low frequency microwave sea surface emissivity model.}
  \label{fig:tl_interface}
\end{figure}

\begin{figure}[htp]
  \centering
  \doublebox{
  \begin{minipage}[b]{6.5in}
    \begin{ttfamily}
      \begin{verbatim}
  SUBROUTINE LowFrequency_MWSSEM_AD( Frequency     , &  ! Input
                                     Zenith_Angle  , &  ! Input
                                     Temperature   , &  ! FWD Input
                                     Salinity      , &  ! FWD Input
                                     Wind_Speed    , &  ! FWD Input
                                     Emissivity_AD , &  ! AD  Input
                                     Temperature_AD, &  ! AD  Output
                                     Salinity_AD   , &  ! AD  Output
                                     Wind_Speed_AD , &  ! AD  Output
                                     iVar            )  ! Internal variable input
    REAL(fp),        INTENT(IN)     :: Frequency
    REAL(fp),        INTENT(IN)     :: Zenith_Angle
    REAL(fp),        INTENT(IN)     :: Temperature
    REAL(fp),        INTENT(IN)     :: Salinity
    REAL(fp),        INTENT(IN)     :: Wind_Speed
    REAL(fp),        INTENT(IN OUT) :: Emissivity_AD(:)
    REAL(fp),        INTENT(IN OUT) :: Temperature_AD
    REAL(fp),        INTENT(IN OUT) :: Salinity_AD
    REAL(fp),        INTENT(IN OUT) :: Wind_Speed_AD
    TYPE(iVar_type), INTENT(IN)     :: iVar
      \end{verbatim}
    \end{ttfamily}
    \centering
    \begin{tabular}{c|c|c|c}
      \textbf{Argument} & \textbf{Description}                    & \textbf{Units}   & \textbf{Intent} \\
      \hline\hline
      Frequency          & Microwave frequency                    & GHz              & IN      \\
      \hline
                         & Satellite zenith angle                 &                  &         \\
      \rb{Zenith\_Angle} & at the sea surface                     & \rb{Degrees}     & \rb{IN} \\ 
      \hline
      Temperature        & Sea surface temperature                & Kelvin           & IN      \\
      \hline
      Salinity           & Salinity of sea water                  & \textperthousand & IN      \\
      \hline
      Wind\_Speed        & Sea surface wind speed                 & m.s$^{-1}$       & IN      \\
      \hline
                         & The surface emissivity adjoint at      &                  &         \\
      \rb{Emissivity\_AD}& vertical and horizontal polarizations  & \rb{N/A}         & \rb{IN OUT}\\
      \hline
      Temperature\_AD    & Sea surface temperature adjoint        & (Kelvin)$^{-1}$  & IN OUT \\
      \hline
      Salinity\_AD       & Salinity of sea water adjoint          & (\textperthousand)$^{-1}$ & IN OUT  \\
      \hline
      Wind\_Speed\_AD    & Sea surface wind speed adjoint         & (m.s$^{-1}$)$^{-1}$       & IN OUT  \\
      \hline
                         & Structure containing internal variables. &                &         \\
      \rb{iVar}          & Output from the forward model.           & \rb{N/A}       & \rb{IN} \\ 
    \end{tabular}
  \end{minipage}
  }
  \caption{Adjoint model interface and argument description for the low frequency microwave sea surface emissivity model.}
  \label{fig:ad_interface}
\end{figure}
