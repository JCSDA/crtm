\chapter*{What's Next}
%=====================
\addcontentsline{toc}{chapter}{What's Next}

\begin{description}
\item[Single- and double-precision aware procedures] Currently the setting of the default real and integer kind definitions to work with either single- or double-precision LBLRTM files is done via the compilation step via preprocessing. That is, the default real and integer kind definitions are explicitly set. So, once the library is built a partiular way, e.g. to read double-precision files, it cannot be used to read the other type. This was done because (a) it used already existing functionality in the CRTM codebase, and (b) for the CRTM at least we always use double-precision files.

However, to make the library more flexible, the relevant definitions can be made and procedures overloaded for programmatic switching between reading single- or double-precision LBLRTM datafiles.


\item[NetCDF4 conversion and I/O] A conversion procedure is planned to read the LBLRTM format datafile and output a netCDF4 datafile.

 
\item[More comprehensive use of OOP] This is being planned mostly to insulate users from changes to the underlying data structures and objects. Procedures to get and set attributes of the various objects are planned so that direct access of those attributes is not necessary. Use of type-bound procedures is also planned to simplify the calling syntax. Fully-compliant Fortran2003 compilers are still relatively rare so this change may take some time.

\end{description}
