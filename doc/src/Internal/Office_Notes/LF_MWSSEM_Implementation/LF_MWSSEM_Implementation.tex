% The generic preamble
\documentclass[10pt,letterpaper,fleqn,titlepage]{report}

% Define packages to use
\usepackage{natbib}
\usepackage[dvips]{graphicx,color}
\usepackage{amsmath,amssymb}
\usepackage{bm}
\usepackage{caption}
\usepackage{xr}
\usepackage{ifthen}
\usepackage[dvipdfm,colorlinks,linkcolor=blue,citecolor=blue,urlcolor=blue]{hyperref}
\usepackage{fancybox}
\usepackage{textcomp}
\usepackage{fancyhdr}
\usepackage{titlesec}
\usepackage{multirow}
\usepackage{alltt}
\usepackage{svn}
\usepackage{longtable}

\titleformat{\chapter}[frame]
  {\normalfont}
  {\filright\slshape\Huge\enspace\thechapter\enspace}
  {8pt}
  {\normalfont\Huge\filcenter\slshape\sffamily} 

\titleformat{\section}[hang]
  {\normalfont}
  {\filright\sffamily\bfseries\Large\thesection}
  {5pt}
  {\normalfont\Large\filright\sffamily\bfseries}[\vspace{2pt}\titlerule]

\titleformat{\subsection}[hang]
  {\normalfont}
  {\filright\slshape\sffamily\large\thesubsection}
  {5pt}
  {\normalfont\large\filright\slshape\sffamily} 
  
% Redefine default page
\setlength{\textheight}{9.0in}  % 1" above and below
\setlength{\textwidth}{6.75in}   % 0.5" left and right
\setlength{\oddsidemargin}{-0.25in}
\setlength{\topmargin}{0.0pt}
\setlength{\headsep}{16.0pt}

% Redefine default paragraph
\setlength{\parindent}{0pt}
\setlength{\parskip}{1.5ex plus 0.5ex minus 0.2ex}

% Define caption width and default fonts
\setlength{\captionmargin}{0.5in}
\renewcommand{\captionfont}{\sffamily}
\renewcommand{\captionlabelfont}{\bfseries\sffamily}

% Defined commands
\newcommand{\superscript}[1]{\ensuremath{^\textrm{#1}}}
\newcommand{\subscript}[1]{\ensuremath{_\textrm{#1}}}
\newcommand{\invcm}{\textrm{cm\superscript{-1}}}
\newcommand{\micron}{\ensuremath{\mu\textrm{m}}}
\newcommand{\textbfm}[1]{\boldmath\ensuremath{#1}\unboldmath}
\newcommand{\water}{\textrm{H\subscript{2}O}}
\newcommand{\carbondioxide}{\textrm{CO\subscript{2}}}
\newcommand{\ozone}{\textrm{O\subscript{3}}}
\newcommand{\methane}{\textrm{CH\subscript{4}}}
\newcommand{\nitrousoxide}{\textrm{N\subscript{2}O}}
\newcommand{\carbonmonoxide}{\textrm{CO}}

% Define how equations are numbered
\numberwithin{equation}{chapter}
\numberwithin{figure}{chapter}
\numberwithin{table}{chapter}

% Space/nudging commands
\newcommand{\rb}[1]{\raisebox{1.5ex}[0pt]{#1}}

%Redefine the enumerate environment to decrease the item spacing.
\let\oldenumerate=\enumerate
\let\endoldenumerate=\endenumerate
\renewenvironment{enumerate}{%
  \begin{oldenumerate}%
    \setlength{\itemsep}{0ex}%
  }%
  {%
  \end{oldenumerate}%
  }

% Define a command for title page author email footnote
\newcommand{\email}[1]
{%
  \renewcommand{\thefootnote}{\alph{footnote}}%
  \footnote{#1}
  \renewcommand{\thefootnote}{\arabic{footnote}}
}

% Redefine the maketitle macro
\makeatletter
\renewcommand{\maketitle}
{%
  \thispagestyle{empty}
  \vspace*{1in}
  \begin{center}%
     \sffamily
     {\huge\bfseries Joint Center for Satellite Data Assimilation\par}%
  \end{center}
  \begin{flushleft}%
     \sffamily
     \vspace*{0.5in}
     {\Large\bfseries CRTM: \@title\par}%
     \medskip
     {\large\@author\par}%
     \medskip
     {\large\@date\par}%
     \bigskip\hrule\vspace*{2pc}%
  \end{flushleft}%
  \newpage
  \setcounter{footnote}{0}
}
\makeatother


% Define a command for a DRAFT watermark
\usepackage{eso-pic}
\newcommand{\draftwatermark}
{
  \AddToShipoutPicture{%
    \definecolor{lightgray}{gray}{.85}
    \setlength{\unitlength}{1in}
    \put(2.5,3.5){%
      \rotatebox{45}{%
        \resizebox{4in}{1in}{%
          \textsf{\textcolor{lightgray}{DRAFT}}
        }
      }
    }
  }
}


% Define included documents
\includeonly{Introduction,Interface_Description,Model_Test,Component_Test,Conclusions,Fresnel_Reflectivity.appendix,Poly_Routine_Speed.appendix}

% Local definitions
\newcommand{\rb}[1]{\raisebox{1.5ex}[0pt]{#1}}
\newcommand{\de}{\ensuremath{\delta\epsilon}}
\newcommand{\es}{\ensuremath{\epsilon_{s}}}
\newcommand{\einf}{\ensuremath{\epsilon_{\infty}}}
\newcommand{\eo}{\ensuremath{\epsilon_{0}}}
\newcommand{\twopnt}{\ensuremath{2\pi\nu\tau}}
\newcommand{\dstar}{\ensuremath{\delta^{*}\!}}

% Title info
\title{Implementation of a Low Frequency Microwave Sea Surface Emissivity Model}
\author{Paul van Delst\email{paul.vandelst@noaa.gov}\\JCSDA/EMC/SAIC}
\date{July, 2008}
\docnumber{(unassigned)}
\docseries{CRTM}


%-------------------------------------------------------------------------------
%                            Ze document begins...
%-------------------------------------------------------------------------------
\begin{document}
\maketitle

\draftwatermark

\begin{abstract}
Implementing the low frequency microwave sea surface emissivity model of Kazumori \textit{et al}, 2008\cite{Kazumori_etal_2008} in the CRTM involved testing the consistency of the forward, tangent-linear and adjoint model. The code to compute the ocean surface permittivities according to the Guillou\cite{Guillou_etal_1998} model were also tested separately, as were the Fresnel reflectivity codes.

\textbf{Keywords}: CRTM, low frequency microwave sea surface emissivity, Guillou permittivity, Fresnel reflectivity, forward, tangent-linear, adjoint model.
\end{abstract}


% Include all the other various sections
%=======================================
\section{Introduction}
%=====================
A central design principle of the components of the CRTM Tangent-linear (TL) and Adjoint (AD) models is that the Forward model is always called first. Thus, in principle, forward model calculations are always available for the various TL and AD components.

The REL-1.1 CRTM AtmScatter modules obtained the interpolated cloud and aerosol optical properties from the CloudCoeff and AerosolCoeff lookup tables (LUTs) as shown in figure \ref{fig:AtmScatter_Interpolation}. For the tangent-linear and adjoint models (figures \ref{fig:AtmScatter_Interpolation}(b) and (c)), the necessary interpolation parameters such as the bracketing indices and interpolating polynomials were always recomputed.
 
\begin{figure}[htp]
  \centering
  % The generic preamble
\documentclass[10pt,letterpaper,fleqn,titlepage]{report}

% Define packages to use
\usepackage{natbib}
\usepackage[dvips]{graphicx,color}
\usepackage{amsmath,amssymb}
\usepackage{bm}
\usepackage{caption}
\usepackage{xr}
\usepackage{ifthen}
\usepackage[dvipdfm,colorlinks,linkcolor=blue,citecolor=blue,urlcolor=blue]{hyperref}
\usepackage{fancybox}
\usepackage{textcomp}
\usepackage{fancyhdr}
\usepackage{titlesec}
\usepackage{multirow}
\usepackage{alltt}
\usepackage{svn}
\usepackage{longtable}

\titleformat{\chapter}[frame]
  {\normalfont}
  {\filright\slshape\Huge\enspace\thechapter\enspace}
  {8pt}
  {\normalfont\Huge\filcenter\slshape\sffamily} 

\titleformat{\section}[hang]
  {\normalfont}
  {\filright\sffamily\bfseries\Large\thesection}
  {5pt}
  {\normalfont\Large\filright\sffamily\bfseries}[\vspace{2pt}\titlerule]

\titleformat{\subsection}[hang]
  {\normalfont}
  {\filright\slshape\sffamily\large\thesubsection}
  {5pt}
  {\normalfont\large\filright\slshape\sffamily} 
  
% Redefine default page
\setlength{\textheight}{9.0in}  % 1" above and below
\setlength{\textwidth}{6.75in}   % 0.5" left and right
\setlength{\oddsidemargin}{-0.25in}
\setlength{\topmargin}{0.0pt}
\setlength{\headsep}{16.0pt}

% Redefine default paragraph
\setlength{\parindent}{0pt}
\setlength{\parskip}{1.5ex plus 0.5ex minus 0.2ex}

% Define caption width and default fonts
\setlength{\captionmargin}{0.5in}
\renewcommand{\captionfont}{\sffamily}
\renewcommand{\captionlabelfont}{\bfseries\sffamily}

% Defined commands
\newcommand{\superscript}[1]{\ensuremath{^\textrm{#1}}}
\newcommand{\subscript}[1]{\ensuremath{_\textrm{#1}}}
\newcommand{\invcm}{\textrm{cm\superscript{-1}}}
\newcommand{\micron}{\ensuremath{\mu\textrm{m}}}
\newcommand{\textbfm}[1]{\boldmath\ensuremath{#1}\unboldmath}
\newcommand{\water}{\textrm{H\subscript{2}O}}
\newcommand{\carbondioxide}{\textrm{CO\subscript{2}}}
\newcommand{\ozone}{\textrm{O\subscript{3}}}
\newcommand{\methane}{\textrm{CH\subscript{4}}}
\newcommand{\nitrousoxide}{\textrm{N\subscript{2}O}}
\newcommand{\carbonmonoxide}{\textrm{CO}}

% Define how equations are numbered
\numberwithin{equation}{chapter}
\numberwithin{figure}{chapter}
\numberwithin{table}{chapter}

% Space/nudging commands
\newcommand{\rb}[1]{\raisebox{1.5ex}[0pt]{#1}}

%Redefine the enumerate environment to decrease the item spacing.
\let\oldenumerate=\enumerate
\let\endoldenumerate=\endenumerate
\renewenvironment{enumerate}{%
  \begin{oldenumerate}%
    \setlength{\itemsep}{0ex}%
  }%
  {%
  \end{oldenumerate}%
  }

% Define a command for title page author email footnote
\newcommand{\email}[1]
{%
  \renewcommand{\thefootnote}{\alph{footnote}}%
  \footnote{#1}
  \renewcommand{\thefootnote}{\arabic{footnote}}
}

% Redefine the maketitle macro
\makeatletter
\renewcommand{\maketitle}
{%
  \thispagestyle{empty}
  \vspace*{1in}
  \begin{center}%
     \sffamily
     {\huge\bfseries Joint Center for Satellite Data Assimilation\par}%
  \end{center}
  \begin{flushleft}%
     \sffamily
     \vspace*{0.5in}
     {\Large\bfseries CRTM: \@title\par}%
     \medskip
     {\large\@author\par}%
     \medskip
     {\large\@date\par}%
     \bigskip\hrule\vspace*{2pc}%
  \end{flushleft}%
  \newpage
  \setcounter{footnote}{0}
}
\makeatother


% Define a command for a DRAFT watermark
\usepackage{eso-pic}
\newcommand{\draftwatermark}
{
  \AddToShipoutPicture{%
    \definecolor{lightgray}{gray}{.85}
    \setlength{\unitlength}{1in}
    \put(2.5,3.5){%
      \rotatebox{45}{%
        \resizebox{4in}{1in}{%
          \textsf{\textcolor{lightgray}{DRAFT}}
        }
      }
    }
  }
}


% Define included documents
\includeonly{Optical_Property_Dependencies.section,Test_Cloud_and_Aerosol_Profiles.section,Forward_Model_Impact.section,Forward_Tangent-Linear_Model_Tests.section,Tangent-Linear_Adjoint_Model_Tests.section}


% Definitions for tables
\newcommand{\rb}[1]{\raisebox{1.5ex}[0pt]{#1}}
\newcommand{\po}{\ensuremath{p_{0}}}
\newcommand{\bpo}{\boldmath\po\unboldmath}
\newcommand{\Dp}{\ensuremath{\Delta p}}
\newcommand{\bDp}{\boldmath\Dp\unboldmath}
\newcommand{\reff}{\ensuremath{R_{eff}}}
\newcommand{\breff}{\boldmath\reff\unboldmath}
\newcommand{\bhpa}{\textbf{(hPa)}}
\newcommand{\bmicron}{\boldmath\micron\unboldmath}

% Title info
\title{Impact of Optical Property Interpolation on Atmospheric Scattering Computations}
\author{Paul van Delst\email{paul.vandelst@noaa.gov}\\JCSDA/EMC/SAIC}
\date{April, 2008}
\docnumber{(unassigned)}
\docseries{CRTM}


%-------------------------------------------------------------------------------
%                            Ze document begins...
%-------------------------------------------------------------------------------
\begin{document}
\maketitle

%\draftwatermark

\begin{abstract}
The CRTM interpolates the optical properties of clouds and aerosols from look-up tables (LUTs) for use in the scattering radiative transfer. This document details the impact of three different interpolation methodologies on the forward, tangent-linear and adjoint components of the cloud and aerosol scattering computations. Linear, cubic, and averaged quadratic inteprolation schemes were tested with the latter having the property of piecewise continuity of derivatives across LUT hingepoints.

\textbf{Keywords}: CRTM, scattering, clouds, aerosols, interpolation, derivative continuity
\end{abstract}


\section{Interpolation Schemes}
%==============================
For all of the CRTM cloud and aerosol optical property look-up table (LUT) interpolations described in this document, four different interpolation schemes were tested:
 \begin{itemize}
   \item{``old'' linear interpolation}
   \item{new linear interpolation}
   \item{cubic interpolation}
   \item{averaged quadratic interpolation}
 \end{itemize}
The difference between the old and new linear (or 2-point) interpolation routines are only in the way the interpolating polynomials are computed; the old scheme computed the polynomials explicitly like so,
\begin{eqnarray*}
  p_{1}(x) & = & \frac{x-x_{2}}{x_{1}-x_{2}} \\
  p_{2}(x) & = & \frac{x-x_{1}}{x_{2}-x_{1}}
\end{eqnarray*}
whereas the new scheme used the more generic form for computing Lagrangian polynomials for any set of $n+1$ points,
\begin{equation}
  p_{j}(x) = \prod_{\scriptstyle k=1 \atop \scriptstyle k \ne j}^{n+1}\frac{x-x_{k}}{x_{j}-x_{k}}
  \label{eqn:lagrangian_poly}
\end{equation}
with $n=1$. For cubic (or 4-point) interpolation, the same code was used but with $n=3$. In all cases the actual interpolating polynomial is computed using
\begin{equation}
  P(x) = \sum_{j=1}^{n+1} p_{j}(x)\cdot y_{j}
  \label{eqn:interpolating_poly}
\end{equation}

The averaged quadratic scheme\cite{Purser_AvgQuad_Interpolation} uses a weighted average of two adjacent 3-point (i.e. $n=2$) interpolating polynomials to perform interpolation in the overlapping region,
\begin{equation}
  P(x) = W_{l}\sum_{j=1}^{n+1} p_{j}(x)\cdot y_{j} + W_{r}\sum_{j=2}^{n+2} p_{j}(x)\cdot y_{j}
  \label{eqn:avgquad_poly}
\end{equation}
with
\begin{eqnarray*}
  W_{l} & = & 1 - \frac{x-x_{2}}{x_{3}-x_{2}}\\
   & & \\
  W_{r} & = & 1 - W_{l}
\end{eqnarray*}
This scheme preserves the interpolating function derivative continuity, in a piecewise fashion, across an interpolation boundary (also referred to here as a hingepoint). For end-point interpolation, the averaging weights are set accordingly to $W_{l}=1,W_{r}=0$, or $W_{l}=0,W_{r}=1$.

Comparisons between the old and new linear interpolation tests were run as a ``sanity-check'' only; they agreed and are not shown here. All of the subsequent test comparisons are between the linear, cubic, and averaged quadratic runs. Where applicable, tests were carried out for NOAA-18 HIRS/4, AMSU-A, and MHS; GOES-11 imager;  DMSP-16 SSMIS; and IASI Band 1 (645-1210\invcm) although not all the results are shown here. 

% Include all the various sections
%=================================
\section{Optical Property Dependencies}
%======================================

\subsection{Cloud Optical Properties}
%------------------------------------
The cloud optical property look-up table (LUT) contains extinction coefficient ($k_{e}$), single scatter albedo ($w$), asymmetry factor ($g$), and phase function Legendre polynomial coefficient ($P_{i}$) data for six different cloud types. These optical properties vary with repect to physical quantities such as frequency ($f$), effective particle radius ($R_{eff}$), temperature ($T$), and density ($\rho$). The data ranges for these independent variables in the current CRTM cloud optical properties LUT is shown in table \ref{tab:CloudCoeff.Independent.ranges}

\begin{table}[htp]
  \centering
  \begin{tabular}{|c | c | c|}
    \hline
    \textbf{Independent variable} & \textbf{Range} & \textbf{Units} \\
    \hline\hline
    Microwave frequency & 1.4-190.31   & GHz \\
    Infrared frequency  & 102-2902     & \invcm \\
    Microwave $R_{eff}$ & 10-1000      & \micron \\
    Infrared $R_{eff}$  & 5-100        & \micron \\
    Temperature         & 263.16-300.0 & K \\
    Density             & 0.1-0.9      & kg/m\superscript{3} \\
    \hline
  \end{tabular}
  \caption{The range of the independent data in the CRTM cloud optical properties LUT}
  \label{tab:CloudCoeff.Independent.ranges}
\end{table}

\begin{table}[htp]
  \centering
  \begin{tabular}{| c | c |}
    \hline
    \textbf{Data type} & \textbf{Dependency} \\
    \hline\hline
    Microwave frequencies, liquid phase & $f$, $R_{eff}$, $T$\\
    Microwave frequencies, solid phase  & $f$, $R_{eff}$, $\rho$\\
    Infrared frequencies, liquid phase & $f$, $R_{eff}$\\
    Infrared frequencies, solid phase  & $f$, $R_{eff}$, $\rho$\\
    \hline
  \end{tabular}
  \caption{The dependencies for the different gross data types  in the CRTM cloud optical properties LUT}
  \label{tab:CloudCoeff.Dependent.ranges}
\end{table}

For infrared frequencies, the cloud properties are interpolated across frequencies and radii for given densities; for microwave frequencies, the cloud properties are interpolated across frequency, radii and temperature (for liquid water clouds only) for given densities. Thus, depending on the spectral region and cloud type, one-, two-, or three-dimensional LUT interpolation may be performed, as shown in table \ref{tab:cloud_opt_interp}

\begin{table}[htp]
  \centering
  \begin{tabular}{|c | c | c|}
    \hline
    \textbf{Cloud Type} & \textbf{Infrared} & \textbf{Microwave} \\
    \hline\hline
    Water      & 2-D ($f,R_{eff}$) for $\rho_{0}$ & 2-D ($f,T$) for $R_{eff,1}$ \\
    Ice        & 2-D ($f,R_{eff}$) for $\rho_{3}$ & 1-D ($f$) for $R_{eff,1},\rho_{3}$\\
    Rain       & 2-D ($f,R_{eff}$) for $\rho_{0}$ & 3-D ($f,R_{eff},T$) \\
    Snow       & 2-D ($f,R_{eff}$) for $\rho_{1}$ & 2-D ($f,R_{eff}$) for $\rho_{1}$ \\
    Graupel    & 2-D ($f,R_{eff}$) for $\rho_{2}$ & 2-D ($f,R_{eff}$) for $\rho_{2}$ \\
    Hail       & 2-D ($f,R_{eff}$) for $\rho_{3}$ & 2-D ($f,R_{eff}$) for $\rho_{3}$ \\
    \hline
  \end{tabular}
  \caption{The type and dependency of the interpolation performed on the cloud optical properties.}
  \label{tab:cloud_opt_interp}
\end{table}


\subsection{Aerosol Optical Properties}
%--------------------------------------
Similar to the cloud optical property LUT, the aerosol optical property LUT contains $k_{e}$, $w$, $g$, and $P_{i}$ data for eight different aerosol types. The data ranges for these independent variables in the current CRTM aerosol optical properties LUT is shown in table \ref{tab:AerosolCoeff.Independent.ranges}

\begin{table}[htp]
  \centering
  \begin{tabular}{| c | c | c | c |}
    \hline
    \textbf{Independent variable} & \multicolumn{2}{|c|}{\textbf{Range}} & \textbf{Units} \\
    \hline\hline
    Frequency      & \multicolumn{2}{|c|}{250-3125}          & \invcm \\
    \hline
                   & Dust               &   0.0098 - 7.9887  & \\
                   & Sea salt-SSAM      &  0.79790 - 3.7987  & \\
                   & Sea salt-SSCM      &   5.7235 - 28.0934 & \\
                   & Dry Organic carbon &   0.0872 - 0.2122  & \\
    \rb{$R_{eff}$} & Wet Organic carbon &   0.0872 - 0.2122  & \rb{\micron} \\
                   & Dry Black carbon   &    0.039 - 0.0738  & \\
                   & Wet Black carbon   &    0.039 - 0.0738  & \\
                   & Sulfate            &   0.2424 - 0.7929  & \\
    \hline
  \end{tabular}
  \caption{The range of the independent data in the CRTM aerosol optical properties LUT}
  \label{tab:AerosolCoeff.Independent.ranges}
\end{table}

The aerosol data is organised differently in that different radii data are used for the different aerosol types and, thus, two-dimensional interpolation as a function of frequency and effective radius is used for all the aerosol types.

\section{Test Cloud and Aerosol Profiles}
%========================================
Six atmospheric profiles were used corresponding to the standard climatological profiles: Tropical, Midlatitude Summer, Midlatitude Winter, Subarctic Summer, Subarctic Winter, and the U.S. Standard Atmosphere.

Cloud and aerosol profiles were artificially constructed with only cursory correspondence to the climatology - the goal was simply to create a dataset that sufficiently exercises the source code.  The profile shapes were built as a function of pressure, $p$, using Gaussian-like functions,
\begin{equation}
  x(p) = \sum_{i=1}^{N} X_{i}\exp \left[ -\ln 2 \left(2\cdot \frac{|p-p_{0,i}|}{\Delta p_{i}}\right)^n\right]
\end{equation}
where $x \equiv R_{eff}$, cloud water content, or aerosol concentration; $p_{o}$ = the peak value layer pressure; $\Delta p$ = the peak pressure fullwidth at half maximum; and $X$ is the profile maximum value at $p_{o}$. For the effective radius profile, $n$=2, and for the water content and concentration profiles, $n$=3. The values used in constructing the six cloud and aerosol profiles are shown in tables \ref{tab:Test.Profile.cloud_parameters} and  \ref{tab:Test.Profile.aerosol_parameters}. Plots of the cloud and aerosol profiles are shown in figures \ref{fig:Test.Profile1} to \ref{fig:Test.Profile6}
 
\begin{table}[htp]
  \centering
  \begin{tabular}{|c|c|c|c|c|c|}
  \hline
  & & & & \multicolumn{2}{|c|}{\textbf{\itshape X}} \\
  \cline{5-6}
  \rb{\textbf{Cloud}} & \rb{\textbf{Associated}}  & \rb{\bpo}  & \rb{\bDp}  & \breff                & \textbf{Water content}\\
  \rb{\textbf{Type}}  & \rb{\textbf{Climatology}} & \rb{\bhpa} & \rb{\bhpa} & \bfseries{(\bmicron)} & \textbf{(kg/m\superscript{2})}\\
  \hline\hline
  Water   & Tropical           & 700 & 100 &   20 & 5 \\\hline
  Ice     & Subarctic summer   & 325 & 200 &  500 & 2 \\\hline
  Rain    & U.S. Std. Atm.     & 800 & 400 & 1000 & 5 \\\hline
  Snow    & Midlatitude winter & 400 & 200 &  500 & 1 \\\hline
  Graupel & Subarctic winter   & 800 & 100 & 1000 & 3 \\\hline
  Hail    & Midlatitude summer & 800 & 200 & 2000 & 2 \\\hline
  \end{tabular}
  \caption{Parameters used to construct the test cloud profiles.}
  \label{tab:Test.Profile.cloud_parameters}
\end{table}

\begin{table}[htp]
  \centering
  \begin{tabular}{|c|c|c|c|c|c|}
  \hline
  & & & & \multicolumn{2}{|c|}{\textbf{\itshape X}} \\
  \cline{5-6}
  \rb{\textbf{Aerosol}} & \rb{\textbf{Associated}}  & \rb{\bpo}  & \rb{\bDp}  & \breff                & \textbf{Concentration}\\
  \rb{\textbf{Type}}    & \rb{\textbf{Climatology}} & \rb{\bhpa} & \rb{\bhpa} & \bfseries{(\bmicron)} & \textbf{(kg/m\superscript{2})}\\
  \hline\hline
  Dust                    & Tropical                &  750 & 200 &    2 & 2.0   \\\hline
  Sea salt (SSAM)         & Subarctic summer        &  900 & 400 &  1.5 & 1.0   \\\hline
                          &                         &  800 & 200 & 0.15 & 0.06  \\\cline{3-6}
  \rb{Dry organic carbon} & \rb{U.S. Std. Atm.}     &  250 & 100 & 0.09 & 0.03  \\\hline
                          &                         &  800 & 200 & 0.15 & 0.4   \\\cline{3-6}
  \rb{Wet organic carbon} & \rb{Midlatitude winter} &  250 & 150 & 0.09 & 0.2   \\\hline
  Sea salt (SSCM)         & Subarctic winter        & 1000 & 200 & 12.0 & 0.05  \\\hline
                          &                         &  875 & 150 & 0.7  & 0.125 \\\cline{3-6}
  Sulfate                 & Midlatitude summer      &  600 & 200 & 0.45 & 0.05  \\\cline{3-6}
                          &                         &  200 & 100 & 0.3  & 0.03  \\\hline
  \end{tabular}
  \caption{Parameters used to construct the test aerosol profiles.}
  \label{tab:Test.Profile.aerosol_parameters}
\end{table}

\begin{figure}[htp]
  \centering
  \includegraphics[scale=1.0]{graphics/Test.Profile1.eps}
  \caption{Cloud and Aerosol data for Test Profile 1. Tropical climatology used for atmosphere.
    \textbf{(Upper panels)} Cloud profiles \textit{(a1)} Cloud water content \textit{(a2)} Cloud particle effective radius.
    \textbf{(Lower panels)} Aerosol profiles \textit{(b1)} Aerosol concentration \textit{(b2)} Aerosol particle effective radius.}
  \label{fig:Test.Profile1}
\end{figure}

\begin{figure}[htp]
  \centering
  \includegraphics[scale=1.0]{graphics/Test.Profile2.eps}
  \caption{Cloud and Aerosol data for Test Profile 2. Subarctic summer used for atmosphere.
    \textbf{(Upper panels)} Cloud profiles \textit{(a1)} Cloud water content \textit{(a2)} Cloud particle effective radius.
    \textbf{(Lower panels)} Aerosol profiles \textit{(b1)} Aerosol concentration \textit{(b2)} Aerosol particle effective radius.}
  \label{fig:Test.Profile2}
\end{figure}

\begin{figure}[htp]
  \centering
  \includegraphics[scale=1.0]{graphics/Test.Profile3.eps}
  \caption{Cloud and Aerosol data for Test Profile 3. U.S. Standard Atmosphere used for atmosphere.
    \textbf{(Upper panels)} Cloud profiles \textit{(a1)} Cloud water content \textit{(a2)} Cloud particle effective radius.
    \textbf{(Lower panels)} Aerosol profiles \textit{(b1)} Aerosol concentration \textit{(b2)} Aerosol particle effective radius.}
  \label{fig:Test.Profile3}
\end{figure}

\begin{figure}[htp]
  \centering
  \includegraphics[scale=1.0]{graphics/Test.Profile4.eps}
  \caption{Cloud and Aerosol data for Test Profile 4. Midlatitude winter used for atmosphere.
    \textbf{(Upper panels)} Cloud profiles \textit{(a1)} Cloud water content \textit{(a2)} Cloud particle effective radius.
    \textbf{(Lower panels)} Aerosol profiles \textit{(b1)} Aerosol concentration \textit{(b2)} Aerosol particle effective radius.}
  \label{fig:Test.Profile4}
\end{figure}

\begin{figure}[htp]
  \centering
  \includegraphics[scale=1.0]{graphics/Test.Profile5.eps}
  \caption{Cloud and Aerosol data for Test Profile 5. Subarctic winter used for atmosphere.
    \textbf{(Upper panels)} Cloud profiles \textit{(a1)} Cloud water content \textit{(a2)} Cloud particle effective radius.
    \textbf{(Lower panels)} Aerosol profiles \textit{(b1)} Aerosol concentration \textit{(b2)} Aerosol particle effective radius.}
  \label{fig:Test.Profile5}
\end{figure}

\begin{figure}[htp]
  \centering
  \includegraphics[scale=1.0]{graphics/Test.Profile6.eps}
  \caption{Cloud and Aerosol data for Test Profile 6. Midlatitude summer used for atmosphere.
    \textbf{(Upper panels)} Cloud profiles \textit{(a1)} Cloud water content \textit{(a2)} Cloud particle effective radius.
    \textbf{(Lower panels)} Aerosol profiles \textit{(b1)} Aerosol concentration \textit{(b2)} Aerosol particle effective radius.}
  \label{fig:Test.Profile6}
\end{figure}

\section{Forward Model Impact}
%=============================

\subsection{Clouds}
%------------------
The brightness temperature residuals that result from using different schemes for interpolation of the cloud optical properties for the test cloud profiles for NOAA-18 HIRS/4, NOAA-18 AMSU-A, and Band 1 of MetOp-A IASI are shown in figures \ref{fig:hirs4_n18.Differences.FWD.Cloud}, \ref{fig:amsua_n18.Differences.FWD.Cloud}, and \ref{fig:iasiB1_metop-a.Differences.FWD.Cloud} respectively.

Comparison of the clear-cloudy residuals to the interpolation residuals indicates that the latter is a small fraction of the former, but the magnitudes of the interpolation residuals can be relatively large, e.g. HIRS/4 and IASI hail and rain cloud cases, and AMSU-A ch.15 for most cloud cases. In addition, primarily for the infrared instruments, the differences between interpolation schemes alone can be relatively large which suggests that the residuals are overly influenced by anomalous interpolation due to low LUT data density. However, the interpretation of the interpolation residuals is somewhat complicated by the fact that the cloud optical properties used in the current CRTM version do not cover all the ranges of input data (e.g. effective radii, temperatures). This is discussed further in section  \ref{sec:Insufficient.LUT.range.Cloud}.
 
\begin{figure}[htp]
  \centering
  \textsf{\textbf{NOAA-18 HIRS/4}}\vspace{1.5ex}
  \includegraphics[bb=70 130 540 646,clip,scale=1.0]{graphics/Cloud/FWD/hirs4_n18.Differences.eps}
  \caption{Cloudy brightness temperature residuals for NOAA-18 HIRS/4. \textbf{(Left column)} Clear - Cloudy brightness temperature residuals. \textbf{(Centre column)} Cloudy calculation residuals due to difference in interpolated cloud optical properties from linear and cubic interpolation schemes. \textbf{(Right column)} Cloudy calculation residuals due to difference in interpolated cloud optical properties from linear and averaged quadratic interpolation schemes.}
  \label{fig:hirs4_n18.Differences.FWD.Cloud}
\end{figure}

\begin{figure}[htp]
  \centering
  \textsf{\textbf{NOAA-18 AMSU-A}}\vspace{1.5ex}
  \includegraphics[bb=70 130 540 646,clip,scale=1.0]{graphics/Cloud/FWD/amsua_n18.Differences.eps}
  \caption{Cloudy brightness temperature residuals for NOAA-18 AMSU-A. \textbf{(Left column)} Clear - Cloudy brightness temperature residuals. \textbf{(Centre column)} Cloudy calculation residuals due to difference in interpolated cloud optical properties from linear and cubic interpolation schemes. \textbf{(Right column)} Cloudy calculation residuals due to difference in interpolated cloud optical properties from linear and averaged quadratic interpolation schemes.}
  \label{fig:amsua_n18.Differences.FWD.Cloud}
\end{figure}

\begin{figure}[htp]
  \centering
  \textsf{\textbf{MetOp-A IASI (Band 1)}}\vspace{1.5ex}
  \includegraphics[bb=70 130 540 646,clip,scale=1.0]{graphics/Cloud/FWD/iasiB1_metop-a.Differences.eps}
  \caption{Cloudy brightness temperature residuals for MetOp-A IASI Band 1. \textbf{(Left column)} Clear - Cloudy brightness temperature residuals. \textbf{(Centre column)} Cloudy calculation residuals due to difference in interpolated cloud optical properties from linear and cubic  interpolation schemes. \textbf{(Right column)} Cloudy calculation residuals due to difference in interpolated cloud optical properties from linear and averaged quadratic interpolation schemes.}
  \label{fig:iasiB1_metop-a.Differences.FWD.Cloud}
\end{figure}

\subsection{Aerosols}
%--------------------
The brightness temperature residuals that result from different schemes for interpolation of the aerosol optical properties for the test aerosol profiles for NOAA-18 HIRS/4 and Band 1 of MetOp-A IASI are shown in figures \ref{fig:hirs4_n18.Differences.FWD.Aerosol} and \ref{fig:iasiB1_metop-a.Differences.FWD.Aerosol} respectively.

Compared to the cloudy interpolation residuals, the magnitudes of the same for aerosol optical properties is negligible. This could be a result of using a too-low aerosol burden in the test profiles. Alternatively, it could also indicate that the aerosol optical properties are either smoother in general with respect to the independent variables, or that they are simply more effectively represented in the LUT. The similarity between the residuals regardless of the interpolation scheme suggests the latter.

\begin{figure}[htp]
  \centering
  \textsf{\textbf{NOAA-18 HIRS/4}}\vspace{1.5ex}
  \includegraphics[bb=70 130 540 646,clip,scale=1.0]{graphics/Aerosol/FWD/hirs4_n18.Differences.eps}
  \caption{Aerosol brightness temperature residuals for NOAA-18 HIRS/4. \textbf{(Left column)} Clear - Aerosol brightness temperature residuals. \textbf{(Centre column)} Aerosol calculation residuals due to difference in interpolated aerosol optical properties from linear and cubic interpolation schemes. \textbf{(Right column)} Aerosol calculation residuals due to difference in interpolated aerosol optical properties from linear and averaged quadratic interpolation schemes.}
  \label{fig:hirs4_n18.Differences.FWD.Aerosol}
\end{figure}

\begin{figure}[htp]
  \centering
  \textsf{\textbf{MetOp-A IASI (Band 1)}}\vspace{1.5ex}
  \includegraphics[bb=70 130 540 646,clip,scale=1.0]{graphics/Aerosol/FWD/iasiB1_metop-a.Differences.eps}
  \caption{Aerosol brightness temperature residuals for MetOp-A IASI Band 1. \textbf{(Left column)} Clear - Aerosol brightness temperature residuals. \textbf{(Centre column)} Aerosol calculation residuals due to difference in interpolated aerosol optical properties from linear and cubic  interpolation schemes. \textbf{(Right column)} Aerosol calculation residuals due to difference in interpolated aerosol optical properties from linear and averaged quadratic interpolation schemes.}
  \label{fig:iasiB1_metop-a.Differences.FWD.Aerosol}
\end{figure}


\section{Forward/Tangent-Linear Model Tests}
%===========================================
Before discussing the results of the forward/tangent-linear model tests (FWD/TL test), a short description of the test itself is warranted. The FWD/TL test is not a ``pass-or-fail'' type of test, but is performed to allow assessment of the behaviour of the forward and tangent-linear model over a range of perturbations to the model variables. The input variables (temperature, cloud particle effective radius, and cloud water content for the CloudScatter test; aerosol particule effective radius and aerosol concentration for the AersosolScatter test) are perturbed 15 times decreasing from a maximum fraction of 0.1, with each subsequent perturbation being half of the previous one. Thus the final perturbation applied is approximately $6\times 10^{-6}$.
 
 
\subsection{CloudScatter Module}
%-------------------------------
\subsubsection{Insufficient range in LUT}
%........................................
\label{sec:Insufficient.LUT.range.Cloud}
Cases arise where the input cloud properties (temperature and effective radius) fall outside the range of data covered in the cloud optical properties LUT. In the forward model case, when this happens no extrapolation is performed - the LUT extrema values are simply returned. In the tangent-linear model case, the returned result is always zero.

A comparison of test output for the water cloud case for AMSU-A ch.8 (55.5GHz) inspecting the variation of the optical depth as a function of temperature is shown in figure \ref{fig:amsua_n18.ch8.WATER.dOd_dT}. The cloud optical property LUT temperature range is (currently) 263.16-300K. The layer cross-section shown in figure \ref{fig:amsua_n18.ch8.WATER.dOd_dT} is at 695hPa which, for the tropical climatology, has a temperature of 282.14K. The $\pm$0.1 perturbation fraction for the temperature yields 253.9K (-0.1) and 310.3K (+0.1). Because these values are outside the LUT temperature range, the forward model simply returns the 263.16 and 300K values for the -0.1 and +0.1 temperature perturbation respectively and that leads to the ``kinks'' in the non-linear response of figure \ref{fig:amsua_n18.ch8.WATER.dOd_dT}. Note that the feature occurs irrespective of the interpolation method. The tangent-linear result is not impacted in this particular case because the tangent-linear optical depth is not directly affected by the temperature perturbation, but only by the dependence of the mass extinction coefficient on the temperature perturbation - which, for temperatures outside the LUT range, is zero.

\begin{figure}[htp]
  \centering
  \begin{tabular}{c c c}
    \multicolumn{3}{c}{\qquad\sffamily\textbf{NOAA-18 AMSU-A ch.8}}\\
    \multicolumn{3}{c}{\qquad\sffamily\textbf{Water cloud test case}}\\
    \qquad\textsf{(a)} & \qquad\textsf{(b)}  & \qquad\textsf{(c)} \\
    \qquad\textsf{Linear} & \qquad\textsf{Cubic}  & \qquad\textsf{Averaged Quadratic} \\
    \includegraphics[bb=90 400 300 540,clip,scale=0.7]{graphics/Cloud/TL/amsua_n18.ch8.WATER.NLIN.dOd_dT.eps} &
    \includegraphics[bb=90 400 300 540,clip,scale=0.7]{graphics/Cloud/TL/amsua_n18.ch8.WATER.NCUBIC.dOd_dT.eps}  &
    \includegraphics[bb=90 400 300 540,clip,scale=0.7]{graphics/Cloud/TL/amsua_n18.ch8.WATER.AVGQUAD.dOd_dT.eps}
  \end{tabular}
  \caption{Effect of insufficient range in the cloud optical property LUT. Comparison of forward, non-linear (red) and tangent-linear (black) model optical depth variation with respect to temperature at 695hPa for the NOAA-18 AMSU-A ch.8 water cloud case using \textbf{(a)} linear, \textbf{(b)} cubic, and \textbf{(c)} averaged quadratic interpolation. The deviations in the non-linear response for the larger perturbations is due to the input cloud temperature data extending beyond that defined in the LUT. Symbol positions indicate the perturbation fractions at which the calculations were performed.}
  \label{fig:amsua_n18.ch8.WATER.dOd_dT}
\end{figure}

\subsubsection{Discontinuous Derivatives}
%........................................
\label{sec:Discontinuous.derivatives.Cloud}
The biggest problem for the simpler polynomial interpolation schemes being tested here is the fact that the interpolating function derivatives are discontinuous across LUT hingepoints. The effects of this appear as regular failures of linear interpolation and occasional failures of cubic interpolation. The following documents the character of this failures.

A comparison of test output for the snow cloud case for AMSU-A ch.8 (55.5GHz) inspecting the variation of the optical depth as a function of effective radius at 400hPa is shown in figure \ref{fig:amsua_n18.ch8.SNOW.dOd_dReff.TL}. Here, linear interpolation clearly highlights the derivative discontinuity when crossing over hinge-points in the LUT data. Inspection of the LUT dimension vectors lists the effective radii for the microwave cases as [10, 50, 250, 500, 750, 1000] and the test profile parameters for the snow cloud case shown in table \ref{tab:Test.Profile.cloud_parameters} indicate the maximum effective radius is 500\micron. Thus, as the forward model effective radius is perturbed beyond 500\micron, any subsequent linear interpolation of the LUT data will yield discontinuous derivatives. The linear case, figure \ref{fig:amsua_n18.ch8.SNOW.dOd_dReff.TL}(a), shows this most clearly: as soon as the effective radius changes from $<$500\micron{} to $>$500\micron, there is an abrupt change in the slope of the forward, non-linear result. This effect is not apparent in the corresponding plot for the cubic interpolation case but, as is shown later, that is due to a combination of the scale of the plot (i.e. the discontinuity is there, just not visible) and serendipity (i.e. it just so happens that the derivatives of the interpolating polynomials are nearly equal across the LUT hingepoint). The result using averaged quadratic interpolation, figure \ref{fig:amsua_n18.ch8.SNOW.dOd_dReff.TL}(c), appears similar to that for cubic interpolation, but with slightly different response curve slopes.

\begin{figure}[htp]
  \centering
  \begin{tabular}{c c c}
    \multicolumn{3}{c}{\qquad\sffamily\textbf{NOAA-18 AMSU-A ch.8}}\\
    \multicolumn{3}{c}{\qquad\sffamily\textbf{Snow cloud test case}}\\
    \qquad\textsf{(a)} & \qquad\textsf{(b)}  & \qquad\textsf{(c)} \\
    \qquad\textsf{Linear} & \qquad\textsf{Cubic}  & \qquad\textsf{Averaged Quadratic} \\
    \includegraphics[bb=90 400 300 540,clip,scale=0.7]{graphics/Cloud/TL/amsua_n18.ch8.SNOW.NLIN.dOd_dReff.eps} &
    \includegraphics[bb=90 400 300 540,clip,scale=0.7]{graphics/Cloud/TL/amsua_n18.ch8.SNOW.NCUBIC.dOd_dReff.eps} &
    \includegraphics[bb=90 400 300 540,clip,scale=0.7]{graphics/Cloud/TL/amsua_n18.ch8.SNOW.AVGQUAD.dOd_dReff.eps} 
  \end{tabular}
  \caption{Effect of discontinuous derivatives in the FWD/TL test. Comparison of forward, non-linear (red) and tangent-linear (black) model optical depth variation with respect to particle effective radius at 400hPa for the NOAA-18 AMSU-A ch.8 snow cloud case using different interpolation schemes. See figure \ref{fig:Test.Profile4} for the snow cloud water content and effective radius profiles.  \textbf{(a)} Linear interpolation. The abrubt change in the non-linear result slope occurs as the perturbations cross a LUT hingepoint (see text for details). \textbf{(b)} Cubic interpolation of the LUT data in this case does not lead to any noticable discontinuity. \textbf{(c)} Averaged quadratic interpolation preserves the derivatives across LUT higepoints, but note the slopes are slightly different. Symbol positions indicate the perturbation fractions at which the calculations were performed.}
  \label{fig:amsua_n18.ch8.SNOW.dOd_dReff.TL}
\end{figure}

A similar result is obtained in the snow cloud case for HIRS/4 ch.8 (900\invcm), but in this case the difference between cubic and averaged quadratic interpolation is more pronounced. Figure \ref{fig:hirs4_n18.ch8.SNOW.dg_dReff.TL} shows the variation of the asymmetry parameter with respect to effective radius at the 217hPa layer pressure (near the top of the snow cloud). For the linear interpolation case, we see the characteristic abrupt change in the non-linear slope across a LUT hinge-point, whereas for the cubic interpolation case the non-linear response is better behaved. The averaged quadratic interpolation result is quite different from the cubic interpolation case, with a significantly different tangent-linear slope, and a non-linear response that is more pronounced and with opposite curvature with respect to increasing perturbations.

\begin{figure}[htp]
  \centering
  \begin{tabular}{c c c}
    \multicolumn{3}{c}{\qquad\sffamily\textbf{NOAA-18 HIRS/4 ch.8}}\\
    \multicolumn{3}{c}{\qquad\sffamily\textbf{Snow cloud test case}}\\
    \qquad\textsf{(a)} & \qquad\textsf{(b)}  & \qquad\textsf{(c)} \\
    \qquad\textsf{Linear} & \qquad\textsf{Cubic}  & \qquad\textsf{Averaged Quadratic} \\
    \includegraphics[bb=90 400 300 540,clip,scale=0.7]{graphics/Cloud/TL/hirs4_n18.ch8.SNOW.NLIN.dg_dReff.eps} &
    \includegraphics[bb=90 400 300 540,clip,scale=0.7]{graphics/Cloud/TL/hirs4_n18.ch8.SNOW.NCUBIC.dg_dReff.eps} &
    \includegraphics[bb=90 400 300 540,clip,scale=0.7]{graphics/Cloud/TL/hirs4_n18.ch8.SNOW.AVGQUAD.dg_dReff.eps} 
  \end{tabular}
  \caption{Effect of discontinuous derivatives in the FWD/TL test. Comparison of forward, non-linear (red) and tangent-linear (black) model asymmetry parameter variation with respect to particle effective radius at 217hPa for the NOAA-18 HIRS/4 ch.8 snow cloud case using different interpolation schemes. See figure \ref{fig:Test.Profile4} for the snow cloud water content and effective radius profiles. \textbf{(a)} Linear interpolation. The abrubt change in the non-linear result slope occurs as the perturbations cross a LUT hingepoint (see text for details). \textbf{(b)} Cubic interpolation of the LUT data in this case does not lead to any noticeable discontinuity. \textbf{(c)} Averaged quadratic interpolation preserves the derivatives across LUT higepoints, but note the tangent-linear slope and character of the non-linar response are quite different from the cubic interpoaltion case. Symbol positions indicate the perturbation fractions at which the calculations were performed.}
  \label{fig:hirs4_n18.ch8.SNOW.dg_dReff.TL}
\end{figure}

A comparison of test results for the rain cloud case for AMSU-A ch.15 (89GHz), inspecting the variation of the optical depth as a function of effective radius at 918hPa, is shown in figure \ref{fig:amsua_n18.ch15.RAIN.dOd_dReff}. This case is mentioned because changing the interpolation scheme yields results that differ in sign: the linear interpolation scheme produces a negative tangent-linear slope, while the cubic and averaged quadratic produce a positive slope. Similarly for the non-linear response near zero perturbation.

\begin{figure}[htp]
  \centering
  \begin{tabular}{c c c}
    \multicolumn{3}{c}{\qquad\sffamily\textbf{NOAA-18 AMSU-A ch.15}}\\
    \multicolumn{3}{c}{\qquad\sffamily\textbf{Rain cloud test case}}\\
    \qquad\textsf{(a)} & \qquad\textsf{(b)}  & \qquad\textsf{(c)} \\
    \qquad\textsf{Linear} & \qquad\textsf{Cubic}  & \qquad\textsf{Averaged Quadratic} \\
    \includegraphics[bb=90 400 300 540,clip,scale=0.7]{graphics/Cloud/TL/amsua_n18.ch15.RAIN.NLIN.dOd_dReff.eps} &
    \includegraphics[bb=90 400 300 540,clip,scale=0.7]{graphics/Cloud/TL/amsua_n18.ch15.RAIN.NCUBIC.dOd_dReff.eps} &
    \includegraphics[bb=90 400 300 540,clip,scale=0.7]{graphics/Cloud/TL/amsua_n18.ch15.RAIN.AVGQUAD.dOd_dReff.eps} 
  \end{tabular}
  \caption{Impact of interpolation scheme on response slope. Comparison of forward, non-linear (red) and tangent-linear (black) model optical depth variation with respect to particle effective radius at 918hPa for the NOAA-18 AMSU-A ch.15 rain cloud case using different interpolation schemes. See figure \ref{fig:Test.Profile3} for the rain cloud water content and effective radius profiles. \textbf{(a)} Linear interpolation. Tangent-linear response, and non-linear response as $\delta x \rightarrow\pm 0$, have negative slope. \textbf{(b)} Cubic interpolation. Tangent-linear and non-linear response now have a positive slope. \textbf{(c)} Averaged quadratic interpolation produces a result similar to that using cubic interpolation but with a slightly different slope.
Symbol positions indicate the perturbation fractions at which the calculations were performed.}
  \label{fig:amsua_n18.ch15.RAIN.dOd_dReff}
\end{figure}

All of the previous FWD/TL test output has indicated that while linear interpolation produces spurious results in general, cubic interpolation works relatively well. However, there are many cases where simple polynomial interpolation does not produce good results, regardless of the order. Figure \ref{fig:hirs4_n18.ch8.WATER.dg_dReff.TL} shows the FWD/TL asymmetry parameter perturbation profiles due to effective radius perturbations for NOAA-18 HIRS/4 channel 8 water cloud test case at 695hPa. As the effective radius increases beyond a hinge-point, the linear interpolation case (figure \ref{fig:hirs4_n18.ch8.WATER.dg_dReff.TL}(a)) shows the characteristic abrubt change in the non-linear response. However, it also occurs for the cubic interpolation case (figure \ref{fig:hirs4_n18.ch8.WATER.dg_dReff.TL}(b)). Because the averaged quadratic interpolation scheme preserves derivative values across LUT hingepoint, the non-linear response is well-behaved about the hinge-point. Note also that the slope in the averaged quadratic case has the same sign as the linear case, and opposite to the cubic case. 

\begin{figure}[htp]
  \centering
  \begin{tabular}{c c c}
    \multicolumn{3}{c}{\qquad\sffamily\textbf{NOAA-18 HIRS/4 ch.8}}\\
    \multicolumn{3}{c}{\qquad\sffamily\textbf{Water cloud test case}}\\
    \qquad\textsf{(a)} & \qquad\textsf{(b)}  & \qquad\textsf{(c)} \\
    \qquad\textsf{Linear} & \qquad\textsf{Cubic}  & \qquad\textsf{Averaged Quadratic} \\
    \includegraphics[bb=90 400 300 540,clip,scale=0.7]{graphics/Cloud/TL/hirs4_n18.ch8.WATER.NLIN.dg_dReff.eps} &
    \includegraphics[bb=90 400 300 540,clip,scale=0.7]{graphics/Cloud/TL/hirs4_n18.ch8.WATER.NCUBIC.dg_dReff.eps} &
    \includegraphics[bb=90 400 300 540,clip,scale=0.7]{graphics/Cloud/TL/hirs4_n18.ch8.WATER.AVGQUAD.dg_dReff.eps} 
  \end{tabular}
  \caption{Demonstration that cubic interpolation also suffers from the discontinuous derivative problem. Comparison of forward, non-linear (red) and tangent-linear (black) model asymmetry parameter variation with respect to particle effective radius for the NOAA-18 HIRS/4 ch.8 water cloud case using different interpolation schemes. See figure \ref{fig:Test.Profile1} for the water cloud water content and effective radius profiles. \textbf{(a)} Linear interpolation. The abrubt change in the non-linear result slope occurs as the perturbations cross a LUT hingepoint. \textbf{(b)} Cubic interpolation also exhibits the abrupt change in the non-linear response across a LUT hingepoint. \textbf{(c)} Averaged quadratic interpolation preserves derivatives across LUT hingepoints so the non-linear response is well-behaved. Note the change in the sign of the response curve slopes for the different interpolation schemes. Symbol positions indicate the perturbation fractions at which the calculations were performed.}
  \label{fig:hirs4_n18.ch8.WATER.dg_dReff.TL}
\end{figure}

Closer inspection of the actual LUT data indicates why the cubic interpolation scheme performs so poorly with respect to the non-linear response in this case: the data are distributed in such a fashion as to produce large discontinuities in the derivatives across an interpolation higepoint. Figure \ref{fig:g_IR.Reff.898cm-1}(a) shows the water cloud asymmetry parameter LUT data plotted as a function of effective radius for 898\invcm{} (approximately the central frequency of HIRS channel 8) with the cubic and averaged quadratic interpolates superimposed. For the perturbation shown in figure \ref{fig:hirs4_n18.ch8.WATER.dg_dReff.TL}, the effective radius varies from about 18 to 22\micron, with the zero perturbation value for the selected pressure layer being around 18.5\micron. Returning to figure \ref{fig:g_IR.Reff.898cm-1}(a), as the effective radius passes the hinge-point at 20\micron{} (labeled point \#3), the cubic interpolation switches from the red curve to the green curve, which have very different slopes close to the 20\micron{} hingepoint, leading to the corresponding change in the non-linear response seen in figure \ref{fig:hirs4_n18.ch8.WATER.dg_dReff.TL}. For averaged quadratic interpolation, crossing the 20\micron{} hingepoint changes the interpolation from the cyan to the magenta curve which by definition have the same slope at the hingepoint. This is clearly shown in figure \ref{fig:g_IR.Reff.898cm-1}(b) where the derivative of the cubic polynomial interpolates are clearly very different at 20\micron{}, whereas those for the averaged quadratic interpolates are piecewise continuous.

\begin{figure}[htp]
  \centering
  \includegraphics[scale=0.8]{graphics/Cloud/g_IR.Reff.898cm-1.eps}
  \caption{Comparison of interpolation schemes across a cloud optical properties LUT hingepoint. \textbf{(a)} The asymmetry factor is interpolated as a function of effective radius for a single infrared frequency, 898\invcm. The interpolation is being performed in the specified range of effective radii about the 20\micron{} radius hingepoint. \textbf{(b)} The derivatives of the cubic and averaged quadratic interpolation schemes shown in (a). The derivatives of the adjacent cubic polynomials are very different (i.e. discontinuous) at the LUT hingepoint ($R_{eff}=20$\micron), whereas those for the averaged quadratic interpolation are piecewise continuous.}
  \label{fig:g_IR.Reff.898cm-1}
\end{figure}

A series of tests were performed to see if transforming the variables that are interpolated would make a different in this particular case. Both the dependent and independent variables were transformed individually and together. The dependent variable transforms\cite{Purser_Variable_Transform} used were,

\parbox{10cm}{\begin{eqnarray*}
                y & = & \frac{2g-1}{\sqrt{1-(2g-1)^{2}}}\\
                g & = & \frac{1}{2}+\frac{y}{2\sqrt{1+y^{2}}}
              \end{eqnarray*}}\hfill
\parbox{1cm}{\begin{eqnarray}\end{eqnarray}}

and the independent variable transform\cite{Purser_Variable_Transform} used was simply,
\begin{equation}
  x = \frac{1}{r}
  \label{eqn:r_transform}
\end{equation}
Transforming just the asymmetry parameter before interpolating produced a result similar to that for no transformation except that the magnitude of the excursions in the points 1-4 higher order interpolations were slightly reduced from that of figure \ref{fig:g_IR.Reff.898cm-1}.

Transforming the independent variable prior to interpolation produced a slightly better result as seen in figure \ref{fig:g_IR.Reff.898cm-1.r_transformed}(a). The point 1-4 cubic interpolation curve provides a qualitatively better representation of the LUT data although the difference between it and the point 2-5 interpolation curve is still evident. The excursion now seen for the point 2-5 interpolation curve between points 2 and 3 is never used for interpolation, but its appearance does highlight the potential for poor interpolates due to low LUT data density. The interpolate derivatives, shown in figure \ref{fig:g_IR.Reff.898cm-1.r_transformed}(b), behave similarly to the untransformed case.

\begin{figure}[htp]
  \centering
  \includegraphics[scale=0.8]{graphics/Cloud/g_IR.Reff.898cm-1.r_transformed.eps} \\
  \caption{Comparison of interpolation schemes across a cloud optical properties LUT hingepoint when the independent variable is transformed according to eqn.\ref{eqn:r_transform}. Compare with figure \ref{fig:g_IR.Reff.898cm-1}.}
  \label{fig:g_IR.Reff.898cm-1.r_transformed}
\end{figure}

Transforming both the dependent and independent variables prior to interpolation produced results similar to that for just the independent variable transform shown in figure \ref{fig:g_IR.Reff.898cm-1.r_transformed}(a) but with a larger excursion for the points 2-5 interpolation curve between points 2 and 3.

None of the results for the interpolations shown in figures \ref{fig:g_IR.Reff.898cm-1}(a) or \ref{fig:g_IR.Reff.898cm-1.r_transformed}(a) are particularly satisfactory. It is clear that better representation of the cloud optical properties is needed by increasing the LUT data density. 


\subsection{AerosolScatter Module}
%---------------------------------
\subsubsection{Insufficient range in LUT}
%........................................
\label{sec:Insufficient.LUT.range.Aerosol}
As with the CloudScatter tests described in section \ref{sec:Insufficient.LUT.range.Cloud}, the AerosolScatter LUT interpolation produces similar results when the input data lies outside the range of the LUT data. Figure \ref{fig:hirs4_n18.ch8.SSAM.dw_dReff} shows the impact of this effect for single scatter albedo as a function of effective radius for NOAA-18 HIRS/4 ch.8 for the sea salt (SSAM) aerosol test case. Comparison with figure \ref{fig:amsua_n18.ch8.WATER.dOd_dT} shows the same characteristic discontinuity.

\begin{figure}[htp]
  \centering
  \begin{tabular}{c c c}
    \multicolumn{3}{c}{\qquad\sffamily\textbf{NOAA-18 HIRS/4 ch.8}}\\
    \multicolumn{3}{c}{\qquad\sffamily\textbf{Sea Salt (SSAM) test case}}\\
    \qquad\textsf{(a)} & \qquad\textsf{(b)}  & \qquad\textsf{(c)} \\
    \qquad\textsf{Linear} & \qquad\textsf{Cubic}  & \qquad\textsf{Averaged Quadratic} \\
    \includegraphics[bb=90 400 300 540,clip,scale=0.7]{graphics/Aerosol/TL/hirs4_n18.ch8.SSAM.NLIN.dw_dReff.eps} &
    \includegraphics[bb=90 400 300 540,clip,scale=0.7]{graphics/Aerosol/TL/hirs4_n18.ch8.SSAM.NCUBIC.dw_dReff.eps} &
    \includegraphics[bb=90 400 300 540,clip,scale=0.7]{graphics/Aerosol/TL/hirs4_n18.ch8.SSAM.AVGQUAD.dw_dReff.eps} 
  \end{tabular}
  \caption{Effect of insufficient range in the aerosol optical property LUT. Comparison of forward, non-linear (red) and tangent-linear (black) model single scatter albedo variation with respect to effective radius at 718hPa for the NOAA-18 HIRS/4 ch.8 sea salt (SSAM) case using \textbf{(a)} linear, \textbf{(b)} cubic, and \textbf{(c)} averaged quadratic interpolation. The deviations in the non-linear response for the larger perturbations is due to the input aerosol effective radius data extending beyond that defined in the LUT. Symbol positions indicate the perturbation fractions at which the calculations were performed. See figure \ref{fig:Test.Profile2} for the sea salt (SSAM) aerosol concentration and effective radius profiles.}
  \label{fig:hirs4_n18.ch8.SSAM.dw_dReff}
\end{figure}


\subsubsection{Discontinous derivatives and discretised LUT data}
%................................................................
As with the CloudScatter tests described in section \ref{sec:Discontinuous.derivatives.Cloud}, the use of a simple polynomial interpolating function does not preserve the continuity of derivatives across LUT hingepoints. This effect is exacerbated in the aerosol optical property interpolation by discretisation of the data. Figure \ref{fig:hirs4_n18.ch8.SSCM.dg_dReff} shows the impact of this effect for the asymmetry factor as a function of effective radius for NOAA-18 HIRS/4 ch.8 for the sea salt (SSCM) aerosol test case. Note that the typical abrupt change is seen in the reponse plots, but the position where it occurs changes with the interpolation method used, and the non-linear response is quite poor in general for the entire range of perturbations; something that was not seen in the CloudScatter interpolations. For the CloudScatter case we saw the data in question was not represented at a high enough data density for interpolation to perform well; in this case, it appears the slight discretisation of the LUT data along the y-axis coupled with the irregular spacing in the x-axis is the cause of poor non-linear response for this case, as shown in figure \ref{fig:g.Reff.SSCM.900cm-1}. Figure \ref{fig:g.Reff.SSCM.900cm-1}(b) shows the significant difference between the interpolating functions either side of a LUT hingepoint (in this case $\sim$11.41\micron.)

\begin{figure}[htp]
  \centering
  \begin{tabular}{c c c}
    \multicolumn{3}{c}{\qquad\sffamily\textbf{NOAA-18 HIRS/4 ch.8}}\\
    \multicolumn{3}{c}{\qquad\sffamily\textbf{Sea Salt (SSCM) test case}}\\
    \qquad\textsf{(a)} & \qquad\textsf{(b)}  & \qquad\textsf{(c)} \\
    \qquad\textsf{Linear} & \qquad\textsf{Cubic}  & \qquad\textsf{Averaged Quadratic} \\
    \includegraphics[bb=90 400 300 540,clip,scale=0.7]{graphics/Aerosol/TL/hirs4_n18.ch8.SSCM.NLIN.dg_dReff.eps} &
    \includegraphics[bb=90 400 300 540,clip,scale=0.7]{graphics/Aerosol/TL/hirs4_n18.ch8.SSCM.NCUBIC.dg_dReff.eps} &
    \includegraphics[bb=90 400 300 540,clip,scale=0.7]{graphics/Aerosol/TL/hirs4_n18.ch8.SSCM.AVGQUAD.dg_dReff.eps} 
  \end{tabular}
  \caption{Effect of discretised data when interpolating LUT aerosol optical property data. Comparison of forward, non-linear (red) and tangent-linear (black) model asymmetry parameter variation with respect to effective radius at 972hPa for the NOAA-18 HIRS/4 ch.8 sea salt (SSAM) case using \textbf{(a)} linear, \textbf{(b)} cubic, and \textbf{(c)} averaged quadratic interpolation. Symbol positions indicate the perturbation fractions at which the calculations were performed. See figure \ref{fig:Test.Profile5} for the sea salt (SSCM) aerosol concentration and effective radius profiles.}
  \label{fig:hirs4_n18.ch8.SSCM.dg_dReff}
\end{figure}

\begin{figure}[htp]
  \centering
  \includegraphics[scale=0.8]{graphics/Aerosol/g.Reff.SSCM.900cm-1.eps}
  \caption{Comparison of interpolation across a aerosol optical property LUT hingepoint where the data is partially discretised. \textbf{(a)} The sea salt (SSCM) asymmetry factor as a function of effective radius for a single infrared frequency of 900\invcm. \textbf{(b)} Zoom of a portion of the plot in (a) showing the respective interpolation curves for interpolation being performed about the $\sim$11.4\micron{} radius hingepoint. The character of the cubic interpolating function generated using points 1-4 (red curve) is different from that generated using points 2-5 (green curve). The averaged quadratic interpolations are more well behaved about the LUT hingepoint.}
  \label{fig:g.Reff.SSCM.900cm-1}
\end{figure}

\section{Tangent-Linear/Adjoint Model Tests}
%===========================================
The tangent-linear/adjoint (TL/AD) test is a simpler one than the FWD/TL test. In this test both the CloudScatter and AerosolScatter tangent-linear and adjoint models are run with successive inputs given a value of 1.0. The subsequent TL output and transpose of the AD output should agree to within numerical precision. This should be true regardless of the LUT interpolation scheme used, and it was found to be so. The results shown here are simply an additional documentation of the difference between the adjoint model outputs that are due to the interpolation method.

\subsection{CloudScatter Module}
%-------------------------------
Using the snow cloud case for AMSU-A ch.8 again, the differences between the three interpolation methods  in the adjoint model is shown in figure \ref{fig:amsua_n18.ch8.SNOW.dOd_dReff.AD}. The Jacobian profile for the linear interpolation case, figure \ref{fig:amsua_n18.ch8.SNOW.dOd_dReff.AD}(a), is significantly different in shape and magnitude compared to either the cubic or averaged quadratic interpolation results. The differences between the cubic and averaged quadratic interpolation results are more subtle with the latter having a slightly larger peak value than the former.
\begin{figure}[htp]
  \centering
  \begin{tabular}{c c c}
    \multicolumn{3}{c}{\qquad\sffamily\textbf{NOAA-18 AMSU-A ch.8}}\\
    \multicolumn{3}{c}{\qquad\sffamily\textbf{Snow cloud test case}}\\
    \qquad\textsf{(a)} & \qquad\textsf{(b)}  & \qquad\textsf{(c)} \\
    \qquad\textsf{Linear} & \qquad\textsf{Cubic}  & \qquad\textsf{Averaged Quadratic} \\
    \includegraphics[bb=90 400 300 540,clip,scale=0.7]{graphics/Cloud/AD/amsua_n18.ch8.SNOW.NLIN.dOd_dReff.eps} &
    \includegraphics[bb=90 400 300 540,clip,scale=0.7]{graphics/Cloud/AD/amsua_n18.ch8.SNOW.NCUBIC.dOd_dReff.eps} &
    \includegraphics[bb=90 400 300 540,clip,scale=0.7]{graphics/Cloud/AD/amsua_n18.ch8.SNOW.AVGQUAD.dOd_dReff.eps} \\
  \end{tabular}
  \caption{Comparison of the tangent-linear (black) and adjoint (red) model optical depth variation with respect to effective radius profile for the NOAA-18 AMSU-A ch.8 snow cloud case using \textbf{(a)} linear, \textbf{(b)} cubic, and \textbf{(c)} averaged quadratic interpolation. See figure \ref{fig:Test.Profile4} for the snow cloud water content and effective radius profiles.}
  \label{fig:amsua_n18.ch8.SNOW.dOd_dReff.AD}
\end{figure}

A similar comparison for the rain cloud case for AMSU-A channel 15 is shown in figure \ref{fig:amsua_n18.ch15.RAIN.dw_dReff.AD}. In this example, the case of figure \ref{fig:amsua_n18.ch15.RAIN.dw_dReff.AD}(a) clearly shows the shortcomings of using linear interpolation with the current cloud optical property LUT data. Comparison of the cubic and averaged quadratic interpolation results of figure \ref{fig:amsua_n18.ch15.RAIN.dw_dReff.AD}(b) and (c) respectively again shows subtle, but noticeable, differences in both the Jacobian shape and peak magnitudes.
\begin{figure}[htp]
  \centering
  \begin{tabular}{c c c}
    \multicolumn{3}{c}{\qquad\sffamily\textbf{NOAA-18 AMSU-A ch.15}}\\
    \multicolumn{3}{c}{\qquad\sffamily\textbf{Rain cloud test case}}\\
    \qquad\textsf{(a)} & \qquad\textsf{(b)}  & \qquad\textsf{(c)} \\
    \qquad\textsf{Linear} & \qquad\textsf{Cubic}  & \qquad\textsf{Averaged Quadratic} \\
    \includegraphics[bb=90 400 300 540,clip,scale=0.7]{graphics/Cloud/AD/amsua_n18.ch15.RAIN.NLIN.dw_dReff.eps} &
    \includegraphics[bb=90 400 300 540,clip,scale=0.7]{graphics/Cloud/AD/amsua_n18.ch15.RAIN.NCUBIC.dw_dReff.eps}  &
    \includegraphics[bb=90 400 300 540,clip,scale=0.7]{graphics/Cloud/AD/amsua_n18.ch15.RAIN.AVGQUAD.dw_dReff.eps} \\
  \end{tabular}
  \caption{Comparison of the tangent-linear (black) and adjoint (red) model single scatter albedo variation with respect to effective radius profile for the NOAA-18 AMSU-A ch.15 rain cloud case using \textbf{(a)} linear, \textbf{(b)} cubic, and \textbf{(c)} averaged quadratic interpolation. See figure \ref{fig:Test.Profile3} for the rain cloud water content and effective radius profiles.}
  \label{fig:amsua_n18.ch15.RAIN.dw_dReff.AD}
\end{figure}

 
\subsection{AerosolScatter Module}
%---------------------------------
The NOAA-18 HIRS/4 sea salt (SSAM) test case discussed in the FWD/TL section, as well as a dust aerosol case, are shown here. The results for single scatter albedo and asymmetry parameter Jacobian profiles for the three interpolation schemes are shown in figures \ref{fig:hirs4_n18.ch8.SSAM.dw_dReff.AD} and \ref{fig:hirs4_n18.ch8.DUST.dg_dReff.AD} for the sea salt (SSAM) and dust aerosol cases respectively. The differences between the results due to the interpolation is not as marked for the aerosol cases as for the cloud cases - probably due to a combination of a smaller radiometric effect in general for aerosols, and a higher LUT data density that tends to minimise interpolation errors.

\begin{figure}[htp]
  \centering
  \begin{tabular}{c c c}
    \multicolumn{3}{c}{\qquad\sffamily\textbf{NOAA-18 HIRS/4}}\\
    \multicolumn{3}{c}{\qquad\sffamily\textbf{Sea salt (SSAM) test case}}\\
    \qquad\textsf{(a)} & \qquad\textsf{(b)}  & \qquad\textsf{(c)} \\
    \qquad\textsf{Linear} & \qquad\textsf{Cubic}  & \qquad\textsf{Averaged Quadratic} \\
    \includegraphics[bb=90 400 300 540,clip,scale=0.7]{graphics/Aerosol/AD/hirs4_n18.ch8.SSAM.NLIN.dw_dReff.eps} &
    \includegraphics[bb=90 400 300 540,clip,scale=0.7]{graphics/Aerosol/AD/hirs4_n18.ch8.SSAM.NCUBIC.dw_dReff.eps} &
    \includegraphics[bb=90 400 300 540,clip,scale=0.7]{graphics/Aerosol/AD/hirs4_n18.ch8.SSAM.AVGQUAD.dw_dReff.eps}
  \end{tabular}
  \caption{Comparison of the tangent-linear (black) and adjoint (red) model single scatter albedo variation with respect to effective radius profile for the NOAA-18 HIRS/4 ch.8 sea salt (SSAM) aerosol case using \textbf{(a)} linear, \textbf{(b)} cubic, and \textbf{(c)} averaged quadratic interpolation. See figure \ref{fig:Test.Profile2} for the sea salt (SSAM) aerosol concentration and effective radius profiles.}
  \label{fig:hirs4_n18.ch8.SSAM.dw_dReff.AD}
\end{figure}

\begin{figure}[htp]
  \centering
  \begin{tabular}{c c c}
    \multicolumn{3}{c}{\qquad\sffamily\textbf{NOAA-18 HIRS/4}}\\
    \multicolumn{3}{c}{\qquad\sffamily\textbf{Dust test case}}\\
    \qquad\textsf{(a)} & \qquad\textsf{(b)}  & \qquad\textsf{(c)} \\
    \qquad\textsf{Linear} & \qquad\textsf{Cubic}  & \qquad\textsf{Averaged Quadratic} \\
    \includegraphics[bb=90 400 300 540,clip,scale=0.7]{graphics/Aerosol/AD/hirs4_n18.ch8.DUST.NLIN.dg_dReff.eps} &
    \includegraphics[bb=90 400 300 540,clip,scale=0.7]{graphics/Aerosol/AD/hirs4_n18.ch8.DUST.NCUBIC.dg_dReff.eps} &
    \includegraphics[bb=90 400 300 540,clip,scale=0.7]{graphics/Aerosol/AD/hirs4_n18.ch8.DUST.NCUBIC.dg_dReff.eps} \\
  \end{tabular}
  \caption{Comparison of the tangent-linear (black) and adjoint (red) model asymmetry parameter variation with respect to effective radius profile for the NOAA-18 HIRS/4 ch.8 dust aerosol case using \textbf{(a)} linear, \textbf{(b)} cubic, and \textbf{(c)} averaged quadratic interpolation. See figure \ref{fig:Test.Profile1} for the dust aerosol concentration and effective radius profiles.}
  \label{fig:hirs4_n18.ch8.DUST.dg_dReff.AD}
\end{figure}



%More stuff
%- Timing tests


\section{Conclusions}
%====================
While not a new observation, it is quite clear that the use of a simple polynomial for interpolation is not sufficient when comparing forward, tangent-linear, and adjoint model output. An interpolation scheme that preserves the continuity of derivatives across interpolation hingepoints, such as the averaged quadratic scheme discussed in this document, is required to ensure the adjoint model output of the CloudScatter and AerosolScatter modules of the CRTM are useful in the context of data assimilation. It is more than likely that the impact of the LUT interpolation schemes on the cloud and aerosol property Jacobians will not be large (if detactable at all), but the CRTM uses these interpolating procedures in several other modules (e.g. surface emissivity LUTs) so doing it correctly affects more than just the scattering codes in the CRTM.

In addition, the construction of the LUT data needs to be done more carefully. Insufficient data ranges for the parameters in question (e.g. effective radii, cloud temperatures) are more likely to be an issue even if the interpolation was perfect. Also, ensuring the LUT data are well represented in both the independent and dependent data directions is necessary to prevent spurious excursions in the interpolated data.

% The references section
%=======================
\bibliographystyle{plain}
\bibliography{bibliography}


%% The appendices section
%%=======================
%\begin{appendix}
%
%\end{appendix}


\end{document}


  \caption{Schematic flowcharts of the REL-1.1 CRTM Forward, Tangent-linear, and Adjoint routines used to interpolate the cloud and aerosol optical properties from the CloudCoeff and AerosolCoeff lookup tables. The forward component of the interpolation is recomputed in the tangent-linear and adjoint routines.}
  \label{fig:AtmScatter_Interpolation}
\end{figure}

The CRTM already saves the intermediate forward model results for its components in private structures, i.e. the structure definition is public, but the internals are private. We call these structures the internal variables for the particular CRTM component (e.g. AtmAbsorption, CloudScatter, AerosolScatter, etc).

To avoid the unnecessary forward model recalculations for the cloud and aerosol property LUT interpolations, separate structures were created to retain the intermediate LUT interpolation results, and added to the main internal variable structure. The structure used for cloudy atmospheres is shown in figure \ref{fig:CSinterp_structure} and its usage in the CloudScatter internal variable structure is shown in figure \ref{fig:CSVariables_structure}. Similar, the aerosol structure is shown in figure \ref{fig:ASinterp_structure}, and its usage in the AerosolScatter internal variable structure is shown in figure \ref{fig:ASVariables_structure}. Because the cloud optical properties in the microwave are dependent upon an additional parameter (temperature), a different structure is defined for cloud and aerosol computations to minimise memory usage.

\begin{figure}[htp]
  \centering
  \doublebox{
  \begin{minipage}[b]{6.5in}
    \begin{ttfamily}
      \begin{verbatim}
      
  TYPE :: CSinterp_type                              
    ! The interpolating polynomials                  
    TYPE(LPoly_type) :: wlp  ! Frequency             
    TYPE(LPoly_type) :: xlp  ! Effective radius      
    TYPE(LPoly_type) :: ylp  ! Temperature           
    ! The LUT interpolation indices                  
    INTEGER :: i1, i2        ! Frequency             
    INTEGER :: j1, j2        ! Effective radius      
    INTEGER :: k1, k2        ! Temperature           
    ! The LUT interpolation boundary check           
    LOGICAL :: f_outbound    ! Frequency             
    LOGICAL :: r_outbound    ! Effective radius      
    LOGICAL :: t_outbound    ! Temperature           
    ! The interpolation input                        
    REAL(fp) :: f_int        ! Frequency             
    REAL(fp) :: r_int        ! Effective radius      
    REAL(fp) :: t_int        ! Temperature           
    ! The data to be interpolated                    
    REAL(fp) :: f(NPTS)      ! Frequency             
    REAL(fp) :: r(NPTS)      ! Effective radius      
    REAL(fp) :: t(NPTS)      ! Temperature           
  END TYPE CSinterp_type                             
      \end{verbatim}
    \end{ttfamily}
  \end{minipage}
  }
  \caption{The structure definition to hold the forward model cloud optical properties lookup table  interpolation results.}
  \label{fig:CSinterp_structure}
\end{figure}

\begin{figure}[htp]
  \centering
  \doublebox{
  \begin{minipage}[b]{6.5in}
    \begin{ttfamily}
      \begin{alltt}
      
  TYPE :: CRTM_CSVariables_type
    PRIVATE
    ! The interpolation data
    \hyperref[fig:CSinterp_structure]{TYPE(CSinterp_type) :: csi(MAX_N_LAYERS, MAX_N_CLOUDS)}
    ! The interpolation results
    REAL(fp) :: ke(MAX_N_LAYERS, MAX_N_CLOUDS)  ! Mass extinction coefficient
    REAL(fp) :: w(MAX_N_LAYERS, MAX_N_CLOUDS)   ! Single scatter albedo
    REAL(fp) :: g(MAX_N_LAYERS, MAX_N_CLOUDS)   ! Asymmetry factor
    REAL(fp) :: pcoeff(0:MAX_N_LEGENDRE_TERMS,&
                       MAX_N_PHASE_ELEMENTS,  &
                       MAX_N_LAYERS,          &
                       MAX_N_CLOUDS           ) ! Phase coefficient
    ! The accumulated scattering coefficient
    REAL(fp) :: Total_bs(MAX_N_LAYERS)          ! Volume scattering coefficient
  END TYPE CRTM_CSVariables_type
      \end{alltt}
    \end{ttfamily}
  \end{minipage}
  }
  \caption{The structure definition to hold all the forward model CloudScatter results showing the added \texttt{csi} structure array used to hold the intermediate interpolation results. The array dimensions are declared in the \texttt{CRTM\_Parameters} module.}
  \label{fig:CSVariables_structure}
\end{figure}

\begin{figure}[htp]
  \centering
  \doublebox{
  \begin{minipage}[b]{6.5in}
    \begin{ttfamily}
      \begin{verbatim}
      
  TYPE :: ASinterp_type
    ! The interpolating polynomials
    TYPE(LPoly_type) :: wlp  ! Frequency
    TYPE(LPoly_type) :: xlp  ! Effective radius
    ! The LUT interpolation indices
    INTEGER :: i1, i2        ! Frequency
    INTEGER :: j1, j2        ! Effective radius
    ! The LUT interpolation boundary check
    LOGICAL :: f_outbound    ! Frequency
    LOGICAL :: r_outbound    ! Effective radius
    ! The interpolation input
    REAL(fp) :: f_int        ! Frequency
    REAL(fp) :: r_int        ! Effective radius
    ! The data to be interpolated
    REAL(fp) :: f(NPTS)      ! Frequency
    REAL(fp) :: r(NPTS)      ! Effective radius
  END TYPE ASinterp_type
      \end{verbatim}
    \end{ttfamily}
  \end{minipage}
  }
  \caption{The structure definition to hold the forward model aerosol optical properties lookup table  interpolation results.}
  \label{fig:ASinterp_structure}
\end{figure}

\begin{figure}[htp]
  \centering
  \doublebox{
  \begin{minipage}[b]{6.5in}
    \begin{ttfamily}
      \begin{alltt}
      
  TYPE :: CRTM_ASVariables_type
    PRIVATE
    ! The interpolation data
    \hyperref[fig:ASinterp_structure]{TYPE(ASinterp_type) :: asi(MAX_N_LAYERS, MAX_N_AEROSOLS)}
    ! The interpolation result
    REAL(fp) :: ke(MAX_N_LAYERS, MAX_N_AEROSOLS)  ! Mass extinction coefficient
    REAL(fp) :: w(MAX_N_LAYERS, MAX_N_AEROSOLS)   ! Single scatter albedo
    REAL(fp) :: g(MAX_N_LAYERS, MAX_N_AEROSOLS)   ! Asymmetry factor
    REAL(fp) :: pcoeff(0:MAX_N_LEGENDRE_TERMS,&
                       MAX_N_PHASE_ELEMENTS,  &
                       MAX_N_LAYERS,          &
                       MAX_N_AEROSOLS         )   ! Phase coefficient
    ! The accumulated scattering coefficient
    REAL(fp) :: Total_bs(MAX_N_LAYERS)            ! Volume scattering coefficient
  END TYPE CRTM_ASVariables_type
      \end{alltt}
    \end{ttfamily}
  \end{minipage}
  }
  \caption{The structure definition to hold all the forward model AerosolScatter results showing the added \texttt{asi} structure array used to hold the intermediate interpolation results. The array dimensions are declared in the \texttt{CRTM\_Parameters} module.}
  \label{fig:ASVariables_structure}
\end{figure}

With the intermediate interpolation results now available for reuse, the tangent-linear and adjoint routines now avoid the forward model recalculations altogether, as shown schematically in figure \ref{fig:AtmScatter_Interpolation.new}.

\begin{figure}[htp]
  \centering
  \input{graphics/Flowcharts/AtmScatter_Interpolation.new.pstex_t}
  \caption{Schematic flowcharts of the REL-1.2 CRTM Forward, Tangent-linear, and Adjoint routines used to interpolate the cloud and aerosol optical properties from the CloudCoeff and AerosolCoeff lookup tables. The forward component of the interpolation is now saved in a structure variable and reused in the tangent-linear and adjoint routines.}
  \label{fig:AtmScatter_Interpolation.new}
\end{figure}

The timing data that follows compares the CloudScatter and AerosolScatter component tests for a baseline (revision 2757) and modified (revision 2787) version of the CRTM library from the RB-1.2 branch.


\section{Interface Description}
%==============================
The model is called from the \texttt{Compute\_MW\_SfcOptics()} functions in the \texttt{CRTM\_MW\_Water\_SfcOptics} module. The main source module is \texttt{CRTM\_LowFrequency\_MWSSEM} and it contains the public entities shown in table \ref{tab:main_procedures_list}. Note that the internal variable structure is usable but not accessible outside the \texttt{CRTM\_LowFrequency\_MWSSEM} module.
\begin{table}[htp]
  \centering
  \begin{tabular}{|c|c|}
    \hline
    \textbf{Name} & \textbf{Description} \\
    \hline\hline
    \multicolumn{2}{|c|}{\textbf{Data types}}\\
    \hline
    \texttt{iVar\_type}               & Internal variable struture \\
    \hline
    \multicolumn{2}{|c|}{\textbf{Subroutines}}\\
    \hline
    \texttt{LowFrequency\_MWSSEM}     & Forward model \\
    \texttt{LowFrequency\_MWSSEM\_TL} & Tangent-linear model \\
    \texttt{LowFrequency\_MWSSEM\_AD} & Adjoint model \\
    \hline
  \end{tabular}
  \caption{List of public procedures in the \texttt{CRTM\_LowFrequency\_MWSSEM} module}
  \label{tab:main_procedures_list}
\end{table}

The interface and argument descriptions for the forward model are shown in figure \ref{fig:fwd_interface}.

The interface and argument descriptions for the tangent-linear model are shown in figure \ref{fig:tl_interface}. Note that the ``internal variable'' argument, \texttt{iVar}, is now an input as this structure contains intermediate forward model variables computed within the forward model. Also note that there are no frequency and zenith angle tangent-linear inputs. This model does not compute sensitivities of the emissivity to those quantities.

The interface and argument descriptions for the adjoint model are shown in figure \ref{fig:ad_interface}. As with the tangent-linear interface, the internal variable argument, \texttt{iVar}, is an input. Note that if an argument is an input in the tangent-linear model, its corresponding adjoint argument is an output. Similarly, adjoint input arguments correspond with forward model output arguments. Note that although the adjoint emissivity is an input to the model, upon exiting the adjoint subroutine it is set to zero.

The temperature, salinity, and wind speed adjoints are all summed over the number of Stokes vector components as shown below,
\begin{eqnarray*}
  \textrm{Temperature\_AD} &=& \sum^{N}_i \frac{\partial e_i}{\partial T}\\
  \textrm{Salinity\_AD}    &=& \sum^{N}_i \frac{\partial e_i}{\partial S}\\
  \textrm{Wind\_Speed\_AD} &=& \sum^{N}_i \frac{\partial e_i}{\partial W}
\end{eqnarray*}
where $N$ is the number of Stokes vector components. Currently, this is fixed at 2 (vertical and horizontal polarisations only).
 
\begin{figure}[htp]
  \centering
  \doublebox{
  \begin{minipage}[b]{6.5in}
    \begin{ttfamily}
      \begin{verbatim}
  SUBROUTINE LowFrequency_MWSSEM( Frequency   , &  ! Input
                                  Zenith_Angle, &  ! Input
                                  Temperature , &  ! Input
                                  Salinity    , &  ! Input
                                  Wind_Speed  , &  ! Input
                                  Emissivity  , &  ! Output
                                  iVar          )  ! Internal variable output
    REAL(fp),        INTENT(IN)     :: Frequency
    REAL(fp),        INTENT(IN)     :: Zenith_Angle
    REAL(fp),        INTENT(IN)     :: Temperature
    REAL(fp),        INTENT(IN)     :: Salinity
    REAL(fp),        INTENT(IN)     :: Wind_Speed
    REAL(fp),        INTENT(OUT)    :: Emissivity(:)
    TYPE(iVar_type), INTENT(IN OUT) :: iVar
      \end{verbatim}
    \end{ttfamily}
    \centering
    \begin{tabular}{c|c|c|c}
      \textbf{Argument} & \textbf{Description}                    & \textbf{Units}   & \textbf{Intent} \\
      \hline\hline
      Frequency          & Microwave frequency                    & GHz              & IN      \\
      \hline
                         & Satellite zenith angle                 &                  &         \\
      \rb{Zenith\_Angle} & at the sea surface                     & \rb{Degrees}     & \rb{IN} \\ 
      \hline
      Temperature        & Sea surface temperature                & Kelvin           & IN      \\
      \hline
      Salinity           & Salinity of sea water                  & \textperthousand & IN      \\
      \hline
      Wind\_Speed        & Sea surface wind speed                 & m.s$^{-1}$       & IN      \\
      \hline
                         & The surface emissivity at vertical     &                  &         \\
      \rb{Emissivity}    & and horizontal polarizations           & \rb{N/A}         & \rb{OUT}\\
      \hline
                         & Structure containing internal          &                  &         \\
      iVar               & variables required for subsequent      & N/A              & OUT     \\
                         & tangent-linear or adjoint model calls. &                  &         \\
    \end{tabular}
  \end{minipage}
  }
  \caption{Forward model interface and argument description for the low frequency microwave sea surface emissivity model.}
  \label{fig:fwd_interface}
\end{figure}

\begin{figure}[htp]
  \centering
  \doublebox{
  \begin{minipage}[b]{6.5in}
    \begin{ttfamily}
      \begin{verbatim}
  SUBROUTINE LowFrequency_MWSSEM_TL( Frequency     , &  ! Input
                                     Zenith_Angle  , &  ! Input
                                     Temperature   , &  ! FWD Input
                                     Salinity      , &  ! FWD Input
                                     Wind_Speed    , &  ! FWD Input
                                     Temperature_TL, &  ! TL  Input
                                     Salinity_TL   , &  ! TL  Input
                                     Wind_Speed_TL , &  ! TL  Input
                                     Emissivity_TL , &  ! TL  Output
                                     iVar            )  ! Internal variable input
    REAL(fp),        INTENT(IN)  :: Frequency
    REAL(fp),        INTENT(IN)  :: Zenith_Angle
    REAL(fp),        INTENT(IN)  :: Temperature
    REAL(fp),        INTENT(IN)  :: Salinity
    REAL(fp),        INTENT(IN)  :: Wind_Speed
    REAL(fp),        INTENT(IN)  :: Temperature_TL
    REAL(fp),        INTENT(IN)  :: Salinity_TL
    REAL(fp),        INTENT(IN)  :: Wind_Speed_TL
    REAL(fp),        INTENT(OUT) :: Emissivity_TL(:)
    TYPE(iVar_type), INTENT(IN)  :: iVar
      \end{verbatim}
    \end{ttfamily}
    \centering
    \begin{tabular}{c|c|c|c}
      \textbf{Argument} & \textbf{Description}                    & \textbf{Units}   & \textbf{Intent} \\
      \hline\hline
      Frequency          & Microwave frequency                    & GHz              & IN      \\
      \hline
                         & Satellite zenith angle                 &                  &         \\
      \rb{Zenith\_Angle} & at the sea surface                     & \rb{Degrees}     & \rb{IN} \\ 
      \hline
      Temperature        & Sea surface temperature                & Kelvin           & IN      \\
      \hline
      Salinity           & Salinity of sea water                  & \textperthousand & IN      \\
      \hline
      Wind\_Speed        & Sea surface wind speed                 & m.s$^{-1}$       & IN      \\
      \hline
      Temperature\_TL    & Sea surface temperature perturbation   & Kelvin           & IN      \\
      \hline
      Salinity\_TL       & Salinity of sea water perturbation     & \textperthousand & IN      \\
      \hline
      Wind\_Speed\_TL    & Sea surface wind speed perturbation    & m.s$^{-1}$       & IN      \\
      \hline
                         & The surface emissivity perturbation at &                  &         \\
      \rb{Emissivity\_TL}& vertical and horizontal polarizations  & \rb{N/A}         & \rb{OUT}\\
      \hline
                         & Structure containing internal            &                &         \\
      \rb{iVar}          & variables. Output from the forward model.& \rb{N/A}       & \rb{IN} \\ 
    \end{tabular}
  \end{minipage}
  }
  \caption{Tangent-linear model interface and argument description for the low frequency microwave sea surface emissivity model.}
  \label{fig:tl_interface}
\end{figure}

\begin{figure}[htp]
  \centering
  \doublebox{
  \begin{minipage}[b]{6.5in}
    \begin{ttfamily}
      \begin{verbatim}
  SUBROUTINE LowFrequency_MWSSEM_AD( Frequency     , &  ! Input
                                     Zenith_Angle  , &  ! Input
                                     Temperature   , &  ! FWD Input
                                     Salinity      , &  ! FWD Input
                                     Wind_Speed    , &  ! FWD Input
                                     Emissivity_AD , &  ! AD  Input
                                     Temperature_AD, &  ! AD  Output
                                     Salinity_AD   , &  ! AD  Output
                                     Wind_Speed_AD , &  ! AD  Output
                                     iVar            )  ! Internal variable input
    REAL(fp),        INTENT(IN)     :: Frequency
    REAL(fp),        INTENT(IN)     :: Zenith_Angle
    REAL(fp),        INTENT(IN)     :: Temperature
    REAL(fp),        INTENT(IN)     :: Salinity
    REAL(fp),        INTENT(IN)     :: Wind_Speed
    REAL(fp),        INTENT(IN OUT) :: Emissivity_AD(:)
    REAL(fp),        INTENT(IN OUT) :: Temperature_AD
    REAL(fp),        INTENT(IN OUT) :: Salinity_AD
    REAL(fp),        INTENT(IN OUT) :: Wind_Speed_AD
    TYPE(iVar_type), INTENT(IN)     :: iVar
      \end{verbatim}
    \end{ttfamily}
    \centering
    \begin{tabular}{c|c|c|c}
      \textbf{Argument} & \textbf{Description}                    & \textbf{Units}   & \textbf{Intent} \\
      \hline\hline
      Frequency          & Microwave frequency                    & GHz              & IN      \\
      \hline
                         & Satellite zenith angle                 &                  &         \\
      \rb{Zenith\_Angle} & at the sea surface                     & \rb{Degrees}     & \rb{IN} \\ 
      \hline
      Temperature        & Sea surface temperature                & Kelvin           & IN      \\
      \hline
      Salinity           & Salinity of sea water                  & \textperthousand & IN      \\
      \hline
      Wind\_Speed        & Sea surface wind speed                 & m.s$^{-1}$       & IN      \\
      \hline
                         & The surface emissivity adjoint at      &                  &         \\
      \rb{Emissivity\_AD}& vertical and horizontal polarizations  & \rb{N/A}         & \rb{IN OUT}\\
      \hline
      Temperature\_AD    & Sea surface temperature adjoint        & (Kelvin)$^{-1}$  & IN OUT \\
      \hline
      Salinity\_AD       & Salinity of sea water adjoint          & (\textperthousand)$^{-1}$ & IN OUT  \\
      \hline
      Wind\_Speed\_AD    & Sea surface wind speed adjoint         & (m.s$^{-1}$)$^{-1}$       & IN OUT  \\
      \hline
                         & Structure containing internal variables. &                &         \\
      \rb{iVar}          & Output from the forward model.           & \rb{N/A}       & \rb{IN} \\ 
    \end{tabular}
  \end{minipage}
  }
  \caption{Adjoint model interface and argument description for the low frequency microwave sea surface emissivity model.}
  \label{fig:ad_interface}
\end{figure}

\section{Model Test}
%===================

This section details the forward/tangent-linear and tangent-linear/adjoint tests performed on the main subroutines: \texttt{LowFrequency\_MWSSEM()}, \texttt{LowFrequency\_MWSSEM\_TL()}, and \texttt{LowFrequency\_MWSSEM\_AD()}. The number of input test values and their ranges of the input forward variables are shown in table \ref{tab:main_input_range}. The actual values used are evenly distributed between the minimum and maximum, inclusively. 
\begin{table}[htp]
  \centering
  \begin{tabular}{| c | c | r@{.}l@{ - }r@{.}l | c |}
    \hline
    \textbf{Quantity} & \textbf{\# of Values} & \multicolumn{4}{c|}{\textbf{Range}} & \textbf{Units} \\
    \hline\hline
    Frequency    & 16 &   5&0 &  20&0 & GHz \\
    Zenith angle &  7 &   0&0 &  60&0 & Deg. \\
    Temperature  & 11 & 273&0 & 303&0 & K \\
    Salinity     &  5 &  20&0 &  40&0 & \textperthousand \\
    Wind speed   & 21 &   2&0 &  19&0 & m.s$^{-1}$ \\
    \hline
  \end{tabular}
  \caption{Range of test input data to main LF MWSSEM routines}
  \label{tab:main_input_range}
\end{table}
The number of frequency and wind speed values were chosen such that there was not always correspondence with the hinge points of the ocean height variance lookup table (LUT). This ensures that the interpolation of the LUT data is included in the testing.

Forward model results for two test frequencies (7.0GHz and 19.0GHz)\footnote{AMSR-E channel 1 and 3 frequencies are 6.925GHz and  18.7GHz respectively.}, one zenith angle (30$^\circ$), and one salinity value (35\textperthousand) are shown in figure \ref{fig:fwd_emissivity}. As mentioned in section \ref{sec:model_description}, for wind speeds greater than 7.0ms$^{-1}$ the Fresnel reflectivities are modified to account for surface foam and this shows up in the forward results as a discontinuity between 7.0 and 8.0ms$^{-1}$ (most evident in figure \ref{fig:fwd_emissivity}(b)). 
\begin{figure}[htp]
  \centering
  \begin{tabular}{c c}
    \multicolumn{2}{c}{\sffamily\textbf{\textbfm{f}=7.0GHz, \textbfm{\theta_i}=30$^\circ$, S=35\textperthousand}}\\
    \textsf{(a) Vertical Polarisation} &
    \textsf{(b) Horizontal Polarization} \\
    \includegraphics[bb=110 240 508 540,clip,scale=0.5]{graphics/Model/ev_s35.0ppt_z30.0_7.00GHz.eps} &
    \includegraphics[bb=110 240 508 540,clip,scale=0.5]{graphics/Model/eh_s35.0ppt_z30.0_7.00GHz.eps} \\\\

    \multicolumn{2}{c}{\sffamily\textbf{\textbfm{f}=19.0GHz, \textbfm{\theta_i}=30$^\circ$, S=35\textperthousand}}\\
    \textsf{(c) Vertical Polarisation} &
    \textsf{(d) Horizontal Polarization} \\
    \includegraphics[bb=110 240 508 540,clip,scale=0.5]{graphics/Model/ev_s35.0ppt_z30.0_19.00GHz.eps} &
    \includegraphics[bb=110 240 508 540,clip,scale=0.5]{graphics/Model/eh_s35.0ppt_z30.0_19.00GHz.eps}
  \end{tabular}
  \caption{Computed vertical and horizontal polarised emissivities at two frequencies $<$ 20GHz, a zenith angle of 30$^\circ$, a salinity of 35\textperthousand, and for a range of ocean surface wind speeds and temperatures. The feature seen at 7m.s$^{-1}$ is due to the modification of the reflectivity due to foam cover.}
  \label{fig:fwd_emissivity}
\end{figure}


\subsection{FWD/TL Test Results}
%-------------------------------
The description of the FWD/TL tests for routines with real valued output is given in section \ref{sec:fwdtl_test}. Some representative results are shown in figure \ref{fig:fwdtl_a0.1000_7.00GHz_emissivity} for 7.0GHz and an alpha value of 0.1. The aforementioned discontinuity seen at wind speeds of 7.0ms$^{-1}$ is quite evident in the non-linear and tangent-linear responses (figures \ref{fig:fwdtl_a0.1000_7.00GHz_emissivity}(a)-(d)). The relatively large value of alpha means the perturbation is also relatively large and as such, the test residuals of figures \ref{fig:fwdtl_a0.1000_7.00GHz_emissivity}(e) and (f) still exhibit some functional characteristics. When the value of alpha is decreased to 0.0001, while the responses themselves appear similar, the test residuals decrease to the point where it appears calculation ``noise'' predominates, as shown in figure \ref{fig:fwdtl_a0.0001_7.00GHz_emissivity}. This is expected as the perturbation applied is much smaller and thus the forward model response is correspondingly more linear. The maximum tolerance residual for each value of alpha is shown in table \ref{tab:fwdtl_alpha}. As expected, as alpha decreases so do the tolerance residuals since.
\begin{table}[htp]
  \centering
  \begin{tabular}{| c | c |}
    \hline
    \boldmath$\alpha$\unboldmath & \textbf{Tolerance residual,} \boldmath$t_r$\unboldmath \\
    \hline\hline
    0.1    & 2.0e-06 \\
    0.01   & 2.0e-07 \\
    0.001  & 2.0e-08 \\
    0.0001 & 2.0e-09 \\
    \hline
  \end{tabular}
  \caption{Maximum tolerance residuals for the emissivity FWD/TL tests.}
  \label{tab:fwdtl_alpha}
\end{table}

Test results for an alpha value of 0.1 but for a frequency of 19.0GHz are shown in figure \ref{fig:fwdtl_a0.1000_19.00GHz_emissivity}. The character of the non-linear and tangent-linear responses is very different to that seen for the 7.0GHz case. The unevenness seen along the wind speed dimension is due to the small scale correction applied for frequencies greater than 15GHz. Smaller residuals, but with the same characteristics spikes, were seen for the  19.0GHz case but with an alpha value of 0.0001, as shown in figure \ref{fig:fwdtl_a0.0001_19.00GHz_emissivity}.

As described in section \ref{sec:small_scale_correction}, the ocean height variance is used in the small-scale reflectivity correction. Figure \ref{fig:sdd_wind_speed_spectra}(a) shows the ocean height variance as a function of wind speed for various frequencies. Although they appear relatively smmoth, removal of the mean slope, as shown in figure \ref{fig:sdd_wind_speed_spectra}(b), shows how noisy the data is, which translates to the perturbation surfaces of figure \ref{fig:fwdtl_a0.1000_19.00GHz_emissivity}. Additionally, the large spikes in the test residuals of figures \ref{fig:fwdtl_a0.1000_19.00GHz_emissivity}(e) and (f) occur at wind speeds of 2.0, 10.5, and 19.0ms$^{-1}$ which are all hingepoints in the ocean height variance LUT.

To determine if the noisy height displacement data is the cause of these wind speed hingepoint spikes, the data was smoothed using a Savitsky-Golay filter (see chapter 14 of \citet{NumericalRecipes_Fortran}) of width 6ms$^{-1}$ and 40GHz in the wind speed and frequency dimensions respectively. The smoothed height variance mean difference wind speed spectra are shown in figure \ref{fig:sddsmthd_wind_speed_spectra}. The FWD/TL residuals using this smoothed data are shown in figure \ref{fig:smthdfwdtl_a0.0001_19.00GHz_emissivity} where they are approximately 2-10 times less than those using the original data, but still exhibit the anomalous peaks at the wind speed hingepoints.

Repeating the tests for different wind speed grids such that interpolation was performed primarily between, and not across, hingepoints led to the residuals shown in figure \ref{fig:fwdtl_wtest_a0.0001_19.00GHz_emissivity}. Only the vertically polarised results are shown. Two tests were run: one where the edge value wind speeds were selected to not coincide with a LUT hingepoint but an intermediate value of 10.5ms$^{-1}$ did, as seen in figure \ref{fig:fwdtl_wtest_a0.0001_19.00GHz_emissivity}(a); and one where there were no test wind speed values near LUT hingepoints, as seen in figure \ref{fig:fwdtl_wtest_a0.0001_19.00GHz_emissivity}(b). It appears that the forward model is particularly sensitive to perturbations about the LUT wind speed hingepoints, even when using the smoothed data.

Because perturbations along the temperature dimension do not exhibit the same behaviour, it suggests the integrations done on the ocean wave spectra of \citet{BjerkaasRiedel_1979} to derive the various height variance values, $\zeta^2_R$, should be recomputed. Since the residuals are of the order of 0.05\% it may be unnecessary, but it does make objective validation of the tangent-linear model difficult.

\begin{figure}[htp]
  \centering
  \begin{tabular}{c c}
    \multicolumn{2}{c}{\sffamily\textbf{Non-linear difference}}\\
    \textsf{(a)} $\Delta e_v$ &
    \textsf{(b)} $\Delta e_h$ \\
    \includegraphics[bb=110 240 508 540,clip,scale=0.5]{graphics/Model/FWDTL/Initial/FWDdev_a0.1000_s35.0ppt_z30.0_7.00GHz.eps} &
    \includegraphics[bb=110 240 508 540,clip,scale=0.5]{graphics/Model/FWDTL/Initial/FWDdeh_a0.1000_s35.0ppt_z30.0_7.00GHz.eps} \\\\
    \multicolumn{2}{c}{\sffamily\textbf{Tangent-linear response}}\\
    \textsf{(c)} $\delta e_v$ &
    \textsf{(d)} $\delta e_h$ \\
    \includegraphics[bb=110 240 508 540,clip,scale=0.5]{graphics/Model/FWDTL/Initial/TLdev_a0.1000_s35.0ppt_z30.0_7.00GHz.eps} &
    \includegraphics[bb=110 240 508 540,clip,scale=0.5]{graphics/Model/FWDTL/Initial/TLdeh_a0.1000_s35.0ppt_z30.0_7.00GHz.eps} \\\\
    \multicolumn{2}{c}{\sffamily\textbf{Forward/tangent-linear test result}}\\
    \textsf{(e)} $|\Delta e_v - \delta e_v|$ &
    \textsf{(f)} $|\Delta e_h - \delta e_h|$ \\
    \includegraphics[bb=110 240 508 540,clip,scale=0.5]{graphics/Model/FWDTL/Initial/FWDTLtestev_a0.1000_s35.0ppt_z30.0_7.00GHz.eps} & 
    \includegraphics[bb=110 240 508 540,clip,scale=0.5]{graphics/Model/FWDTL/Initial/FWDTLtesteh_a0.1000_s35.0ppt_z30.0_7.00GHz.eps}
  \end{tabular}
  \caption{Computed emissivities at 7GHz for the forward/tangent-linear test with $\alpha$=0.1. \textbf{(a)} Vertically polarised non-linear difference.  \textbf{(b)} Horizontally polarised non-linear difference. \textbf{(c)} Vertically polarised tangent-linear response. \textbf{(d)} Horizontally polarised tangent-linear response. \textbf{(e)} Vertically polarised test residual. \textbf{(f)} Horizontally polarised test residual.}
  \label{fig:fwdtl_a0.1000_7.00GHz_emissivity}
\end{figure}

\begin{figure}[htp]
  \centering
  \begin{tabular}{c c}
    \multicolumn{2}{c}{\sffamily\textbf{Forward/tangent-linear test result}}\\
    \textsf{(a)} $|\Delta e_v - \delta e_v|$ &
    \textsf{(b)} $|\Delta e_h - \delta e_h|$ \\
    \includegraphics[bb=110 240 508 540,clip,scale=0.5]{graphics/Model/FWDTL/Initial/FWDTLtestev_a0.0001_s35.0ppt_z30.0_7.00GHz.eps} & 
    \includegraphics[bb=110 240 508 540,clip,scale=0.5]{graphics/Model/FWDTL/Initial/FWDTLtesteh_a0.0001_s35.0ppt_z30.0_7.00GHz.eps}
  \end{tabular}
  \caption{Forward/tangent-linear test residuals at 7GHz for $\alpha$=0.0001. \textbf{(a)} Vertically polarised test residual (compare with figure \ref{fig:fwdtl_a0.1000_7.00GHz_emissivity}(e)). \textbf{(b)} Horizontally polarised test residual (compare with figure \ref{fig:fwdtl_a0.1000_7.00GHz_emissivity}(f)).}
  \label{fig:fwdtl_a0.0001_7.00GHz_emissivity}
\end{figure}

\begin{figure}[htp]
  \centering
  \begin{tabular}{c c}
    \multicolumn{2}{c}{\sffamily\textbf{Non-linear difference}}\\
    \textsf{(a)} $\Delta e_v$ &
    \textsf{(b)} $\Delta e_h$ \\
    \includegraphics[bb=110 240 508 540,clip,scale=0.5]{graphics/Model/FWDTL/Initial/FWDdev_a0.1000_s35.0ppt_z30.0_19.00GHz.eps} &
    \includegraphics[bb=110 240 508 540,clip,scale=0.5]{graphics/Model/FWDTL/Initial/FWDdeh_a0.1000_s35.0ppt_z30.0_19.00GHz.eps} \\\\
    \multicolumn{2}{c}{\sffamily\textbf{Tangent-linear response}}\\
    \textsf{(c)} $\delta e_v$ &
    \textsf{(d)} $\delta e_h$ \\
    \includegraphics[bb=110 240 508 540,clip,scale=0.5]{graphics/Model/FWDTL/Initial/TLdev_a0.1000_s35.0ppt_z30.0_19.00GHz.eps} &
    \includegraphics[bb=110 240 508 540,clip,scale=0.5]{graphics/Model/FWDTL/Initial/TLdeh_a0.1000_s35.0ppt_z30.0_19.00GHz.eps} \\\\
    \multicolumn{2}{c}{\sffamily\textbf{Forward/tangent-linear test result}}\\
    \textsf{(e)} $|\Delta e_v - \delta e_v|$ &
    \textsf{(f)} $|\Delta e_h - \delta e_h|$ \\
    \includegraphics[bb=110 240 508 540,clip,scale=0.5]{graphics/Model/FWDTL/Initial/FWDTLtestev_a0.1000_s35.0ppt_z30.0_19.00GHz.eps} & 
    \includegraphics[bb=110 240 508 540,clip,scale=0.5]{graphics/Model/FWDTL/Initial/FWDTLtesteh_a0.1000_s35.0ppt_z30.0_19.00GHz.eps}
  \end{tabular}
  \caption{Computed emissivities at 19GHz for the forward/tangent-linear test with $\alpha$=0.1. \textbf{(a)} Vertically polarised non-linear difference.  \textbf{(b)} Horizontally polarised non-linear difference. \textbf{(c)} Vertically polarised tangent-linear response. \textbf{(d)} Horizontally polarised tangent-linear response. \textbf{(e)} Vertically polarised test residual. \textbf{(f)} Horizontally polarised test residual.}
  \label{fig:fwdtl_a0.1000_19.00GHz_emissivity}
\end{figure}

\begin{figure}[htp]
  \centering
  \begin{tabular}{c c}
    \multicolumn{2}{c}{\sffamily\textbf{Forward/tangent-linear test result}}\\
    \textsf{(a)} $|\Delta e_v - \delta e_v|$ &
    \textsf{(b)} $|\Delta e_h - \delta e_h|$ \\
    \includegraphics[bb=110 240 508 540,clip,scale=0.5]{graphics/Model/FWDTL/Initial/FWDTLtestev_a0.0001_s35.0ppt_z30.0_19.00GHz.eps} & 
    \includegraphics[bb=110 240 508 540,clip,scale=0.5]{graphics/Model/FWDTL/Initial/FWDTLtesteh_a0.0001_s35.0ppt_z30.0_19.00GHz.eps}
  \end{tabular}
  \caption{Forward/tangent-linear test residuals at 19GHz for $\alpha$=0.0001. \textbf{(a)} Vertically polarised test residual (compare with figure \ref{fig:fwdtl_a0.1000_19.00GHz_emissivity}(e)). \textbf{(b)} Horizontally polarised test residual (compare with figure \ref{fig:fwdtl_a0.1000_19.00GHz_emissivity}(f)).}
  \label{fig:fwdtl_a0.0001_19.00GHz_emissivity}
\end{figure}

\begin{figure}[htp]
  \centering
  \begin{tabular}{c}
    \textsf{(a) Ocean height variance $(4k^2\zeta^2_R)$ wind speed spectra}\\
    \includegraphics[bb=85 400 540 560,clip,scale=0.9]{graphics/LUT/sdd_wind_speed_spectra.eps}\\
    \textsf{(b) Ocean height variance $(4k^2\zeta^2_R)$ mean difference wind speed spectra}\\
    \includegraphics[bb=85 225 540 384,clip,scale=0.9]{graphics/LUT/sdd_wind_speed_spectra.eps}
  \end{tabular}
  \caption{Ocean height variance wind speed spectra used for the small-scale reflectivity correction. \textbf{(a)} Actual height variance spectra in the LUT. Dashed black line is the linear fit to the average for all frequencies. \textbf{(b)} Height variance mean difference spectra obtained by subtracted the mean value and slope from the data, highlighting the noisiness in the LUT data.}
  \label{fig:sdd_wind_speed_spectra}
\end{figure}

\begin{figure}[htp]
  \centering
  \includegraphics[bb=85 225 540 384,clip,scale=0.9]{graphics/LUT/sddsmthd_wind_speed_spectra.eps}
  \caption{Ocean height variance mean difference spectra obtained from smoothed data. Original data was smoothed using a Savitzky-Golay filter in both the wind speed and frequency dimensions. Compare with figure \ref{fig:sdd_wind_speed_spectra}(b).}
  \label{fig:sddsmthd_wind_speed_spectra}
\end{figure}

\begin{figure}[htp]
  \centering
  \begin{tabular}{c c}
    \multicolumn{2}{c}{\sffamily\textbf{Forward/tangent-linear test result}}\\
    \textsf{(a)} $|\Delta e_v - \delta e_v|$ &
    \textsf{(b)} $|\Delta e_h - \delta e_h|$ \\
    \includegraphics[bb=110 240 508 540,clip,scale=0.5]{graphics/Model/FWDTL/Smoothed/FWDTLtestev_a0.1000_s35.0ppt_z30.0_19.00GHz.eps} & 
    \includegraphics[bb=110 240 508 540,clip,scale=0.5]{graphics/Model/FWDTL/Smoothed/FWDTLtesteh_a0.1000_s35.0ppt_z30.0_19.00GHz.eps}
  \end{tabular}
  \caption{Forward/tangent-linear test residuals at 19GHz for $\alpha$=0.1 using the smoothed ocean height variance spectra. Peak residuals are $\sim$2-10 times less than those using the original height variance data. \textbf{(a)} Vertically polarised test residual (compare with figure \ref{fig:fwdtl_a0.1000_19.00GHz_emissivity}(e)). \textbf{(b)} Horizontally polarised test residual (compare with figure \ref{fig:fwdtl_a0.1000_19.00GHz_emissivity}(f)).}
  \label{fig:smthdfwdtl_a0.0001_19.00GHz_emissivity}
\end{figure}

\begin{figure}[htp]
  \centering
  \begin{tabular}{c c}
    \multicolumn{2}{c}{\sffamily\textbf{Forward/tangent-linear test result}}\\
    \textsf{(a) Wind speed test gridpoint at} &
    \textsf{(b) No wind speed test gridpoint}  \\
    \textsf{LUT hingepoint of 10ms\textbfm{^{\textsf{-1}}}} &
    \textsf{corresponds with LUT hingepoints}  \\
    \includegraphics[bb=110 240 508 540,clip,scale=0.5]{graphics/Model/FWDTL/Initial/FWDTLtestev_wtest1_a0.0001_s35.0ppt_z30.0_19.00GHz.eps} & 
    \includegraphics[bb=110 240 508 540,clip,scale=0.5]{graphics/Model/FWDTL/Initial/FWDTLtestev_wtest2_a0.0001_s35.0ppt_z30.0_19.00GHz.eps}
  \end{tabular}
  \caption{Forward/tangent-linear vertically polarised test residuals at 19GHz for $\alpha$=0.0001 for different wind speed grid spacings. Compare with figure \ref{fig:fwdtl_a0.0001_19.00GHz_emissivity}(a). \textbf{(a)} Edge wind speed values no longer correspond with LUT hingepoints, but centre value at 10ms\textbfm{^{\textsf{-1}}} does . \textbf{(b)} No wind speed values correspond with LUT hingepoints.}
  \label{fig:fwdtl_wtest_a0.0001_19.00GHz_emissivity}
\end{figure}


\subsection{TL/AD Test Results}
%------------------------------
Following the description of the TL/AD test in section \ref{sec:tlad_test} for routines with both real valued input and output, the TL/AD test performed for the model was,
\begin{equation}
  \underbrace{\left[\delta e_v^2 + \delta e_h^2\right]}_{\mathbf{TL}^{T}\mathbf{TL}} - \underbrace{\left[\delta{T}.\dstar T + \delta{S}.\dstar S + \delta{W}.\dstar W\right]}_{\mathbf{\delta x}^{T}\mathbf{AD}(TL)} = 0
  \label{eqn:tlad_model}
\end{equation}
where T, S, and W are the sea surface temperature, salinity, and surface wind speed respectively (TL inputs are set to 0.1 in all cases); and $e_v$ and $e_h$ are the vertically and horizontally polarised sea surface emissivities. Examples of the intermediate quanitites and test residual used in this test are shown in figure \ref{fig:tlad_s35.0ppt_z30.0_7.00GHz} for $f$ = 7.0GHz and figure \ref{fig:tlad_s35.0ppt_z30.0_19.00GHz} for  $f$ = 19.0GHz, both for salinities of 35\textperthousand. In both cases, the residual differences are within numerical preicsion. Additionally, these results are typical for other frequencies and salinities tested.

\begin{figure}[htp]
  \centering
  \begin{tabular}{c c}
    \multicolumn{2}{c}{\sffamily\textbf{Tangent-linear emissivities}}\\
    \textsf{(a)} $\delta r_v$ &
    \textsf{(b)} $\delta r_h$ \\
    \includegraphics[bb=120 240 508 540,clip,scale=0.5]{graphics/Model/TLAD/ev_TL_s35.0ppt_z30.0_7.00GHz.eps} &
    \includegraphics[bb=120 240 508 540,clip,scale=0.5]{graphics/Model/TLAD/eh_TL_s35.0ppt_z30.0_7.00GHz.eps} \\\\
    \multicolumn{2}{c}{\sffamily\textbf{Adjoint temperature and salinity}}\\
    \textsf{(c)} $\dstar T$ &
    \textsf{(d)} $\dstar S$ \\
    \includegraphics[bb=115 240 508 540,clip,scale=0.5]{graphics/Model/TLAD/t_AD_s35.0ppt_z30.0_7.00GHz.eps} &
    \includegraphics[bb=110 240 508 540,clip,scale=0.5]{graphics/Model/TLAD/s_AD_s35.0ppt_z30.0_7.00GHz.eps} \\\\
    {\sffamily\textbf{Adjoint wind speed}} & {\sffamily\textbf{Test residual}} \\
    \textsf{(e)} $\dstar T$ &
    \textsf{(f)} $\mathbf{TL}^{T}\mathbf{TL} - \mathbf{\delta x}^{T}\mathbf{AD}(TL)$ \\
    \includegraphics[bb=110 240 508 540,clip,scale=0.5]{graphics/Model/TLAD/w_AD_s35.0ppt_z30.0_7.00GHz.eps} & 
    \includegraphics[bb=110 240 508 540,clip,scale=0.5]{graphics/Model/TLAD/TLtTL-dxtAD_s35.0ppt_z30.0_7.00GHz.eps}
  \end{tabular}
  \caption{Example of quantities used to test the TL/AD routines for $\delta{T}$, $\delta{S}$, and $\delta{W}$ inputs of 0.1, a salinity of 35\textperthousand, an incidence angle of 30$^{\circ}$ at a frequency of 7.0GHz. \textbf{(a)} Tangent-linear vertical emissivity. \textbf{(b)} Tangent-linear horizontal emissivity. \textbf{(c)} Adjoint temperature.  \textbf{(d)} Adjoint salinity. \textbf{(e)} Adjoint wind speed. \textbf{(f)} Test residual (see eqn.\ref{eqn:tlad_model}).}
  \label{fig:tlad_s35.0ppt_z30.0_7.00GHz}
\end{figure}

\begin{figure}[htp]
  \centering
  \begin{tabular}{c c}
    \multicolumn{2}{c}{\sffamily\textbf{Tangent-linear emissivities}}\\
    \textsf{(a)} $\delta r_v$ &
    \textsf{(b)} $\delta r_h$ \\
    \includegraphics[bb=120 240 508 540,clip,scale=0.5]{graphics/Model/TLAD/ev_TL_s35.0ppt_z30.0_19.00GHz.eps} &
    \includegraphics[bb=120 240 508 540,clip,scale=0.5]{graphics/Model/TLAD/eh_TL_s35.0ppt_z30.0_19.00GHz.eps} \\\\
    \multicolumn{2}{c}{\sffamily\textbf{Adjoint temperature and salinity}}\\
    \textsf{(c)} $\dstar T$ &
    \textsf{(d)} $\dstar S$ \\
    \includegraphics[bb=115 240 508 540,clip,scale=0.5]{graphics/Model/TLAD/t_AD_s35.0ppt_z30.0_19.00GHz.eps} &
    \includegraphics[bb=110 240 508 540,clip,scale=0.5]{graphics/Model/TLAD/s_AD_s35.0ppt_z30.0_19.00GHz.eps} \\\\
    {\sffamily\textbf{Adjoint wind speed}} & {\sffamily\textbf{Test residual}} \\
    \textsf{(e)} $\dstar T$ &
    \textsf{(f)} $\mathbf{TL}^{T}\mathbf{TL} - \mathbf{\delta x}^{T}\mathbf{AD}(TL)$ \\
    \includegraphics[bb=110 240 508 540,clip,scale=0.5]{graphics/Model/TLAD/w_AD_s35.0ppt_z30.0_19.00GHz.eps} & 
    \includegraphics[bb=110 240 508 540,clip,scale=0.5]{graphics/Model/TLAD/TLtTL-dxtAD_s35.0ppt_z30.0_19.00GHz.eps}
  \end{tabular}
  \caption{Example of quantities used to test the TL/AD routines for $\delta{T}$, $\delta{S}$, and $\delta{W}$ inputs of 0.1, a salinity of 35\textperthousand, an incidence angle of 30$^{\circ}$ at a frequency of 19.0GHz. \textbf{(a)} Tangent-linear vertical emissivity. \textbf{(b)} Tangent-linear horizontal emissivity. \textbf{(c)} Adjoint temperature.  \textbf{(d)} Adjoint salinity. \textbf{(e)} Adjoint wind speed. \textbf{(f)} Test residual (see eqn.\ref{eqn:tlad_model}).}
  \label{fig:tlad_s35.0ppt_z30.0_19.00GHz}
\end{figure}

\section{Component Tests}
%========================

\subsection{Guillou Ocean Permittivity}
%--------------------------------------
The Guillou ocean permittivity model is taken from \citet{Guillou_1998} where temperature and salinity dependent polynomimal fits for the conductivity, $\sigma$, the static permittivity, \es, the high frequency permittivity, \einf, and the Debye relaxation time, $\tau$, are used to produce complex permittivity values according to the Debye model. The emissivity model invokes the Guillou permittivity procedures only for frequencies less than 20GHz.

This section details the forward/tangent-linear and tangent-linear/adjoint tests performed on the Guillou ocean permittivity procedures. The number and range of input quantities used in the tests are shown in table \ref{tab:guillou_input_range}. A selection of computed forward model Guillou permittivities at different frequencies are shown in figure \ref{fig:guillou_permittivity}. 

\begin{table}[htp]
  \centering
  \begin{tabular}{| c | c | r@{.}l@{ - }r@{.}l | c |}
    \hline
    \textbf{Quantity} & \textbf{\# of Values} & \multicolumn{4}{c|}{\textbf{Range}} & \textbf{Units} \\
    \hline\hline
    Frequency   & 21 &   5&0 &  20&0 & GHz \\
    Salinity    & 21 &  20&0 &  40&0 & \textperthousand \\
    Temperature & 21 & 273&0 & 303&0 & K \\
    \hline
  \end{tabular}
  \caption{Range of test input data to the Guillou ocean permittivity procedures.}
  \label{tab:guillou_input_range}
\end{table}

\begin{figure}[htp]
  \centering
  \begin{tabular}{c c}
    \multicolumn{2}{c}{\sffamily\textbf{5.0GHz}}\\
    \textsf{(a) Real part} &
    \textsf{(b) Imaginary part} \\
    \includegraphics[bb=135 240 508 540,clip,scale=0.5]{graphics/Guillou/e_re_5.00GHz.eps} &
    \includegraphics[bb=135 240 508 540,clip,scale=0.5]{graphics/Guillou/e_im_5.00GHz.eps} \\\\

    \multicolumn{2}{c}{\sffamily\textbf{10.25GHz}}\\
    \textsf{(c) Real part} &
    \textsf{(d) Imaginary part} \\
    \includegraphics[bb=135 240 508 540,clip,scale=0.5]{graphics/Guillou/e_re_10.25GHz.eps} &
    \includegraphics[bb=135 240 508 540,clip,scale=0.5]{graphics/Guillou/e_im_10.25GHz.eps} \\\\

    \multicolumn{2}{c}{\sffamily\textbf{20.0GHz}}\\
    \textsf{(e) Real part} &
    \textsf{(f) Imaginary part} \\
    \includegraphics[bb=135 240 508 540,clip,scale=0.5]{graphics/Guillou/e_re_20.00GHz.eps} &
    \includegraphics[bb=135 240 508 540,clip,scale=0.5]{graphics/Guillou/e_im_20.00GHz.eps}
  \end{tabular}
  \caption{Real and imaginary parts of the computed Guillou permittivity as a function of temperature and salinity for three frequencies $\le$ 20GHz}
  \label{fig:guillou_permittivity}
\end{figure}


\subsubsection{FWD/TL Test Results}
%..................................
\label{sec:fwdtl_guillou}
The description of the FWD/TL tests for routines with complex valued output was given in section \ref{sec:fwdtl_test}. Some representative results for the Guillou permittivity routines are shown in figure \ref{fig:fwdtl_a0.1000_guillou} for 7.25GHz and an alpha value of 0.1, and in figure \ref{fig:fwdtl_a0.0001_guillou} for 16.25GHz and an alpha value of 0.0001. The maximum tolerance residual for each value of alpha is shown in table \ref{tab:fwdtl_guillou_alpha}.
\begin{table}[htp]
  \centering
  \begin{tabular}{| c | c |}
    \hline
    \boldmath$\alpha$\unboldmath & \textbf{Tolerance residual,} \boldmath$t_r$\unboldmath \\
    \hline\hline
    0.1    & 6.0e-08 \\
    0.01   & 6.0e-10 \\
    0.001  & 5.0e-11 \\
    0.0001 & 4.0e-10 \\
    \hline
  \end{tabular}
  \caption{Maximum tolerance residuals for the Guillou permittivity FWD/TL tests.}
  \label{tab:fwdtl_guillou_alpha}
\end{table}
As would be expected, as alpha decreases so do the tolerance residuals since, for smaller and smaller perturbations the forward model reponse becomes more linear, i.e. the residuals are more due to noise than non-linearity. This is quite evident when one compares the residual surfaces of figure \ref{fig:fwdtl_a0.1000_guillou}(e) and (f) with those in figure \ref{fig:fwdtl_a0.0001_guillou}(e) and (f). As the alpha value decreases, the residuals contain less information about the polynomial dependencies of the Guillou permittivity on temperature (higher orders) and salinity (linear). It it surmised that the $O(10^{-10})$ tolerance limit for the FWD/TL tests is due to the propagation of precision errors in the model parameterisation. In any case, the results are well below the precision of the measurements used in generating the fit coefficients as reported in \citet{Guillou_1998}.

\begin{figure}[htp]
  \centering
  \begin{tabular}{c c}
    \multicolumn{2}{c}{\sffamily\textbf{Non-linear difference}}\\
    \textsf{(a)} $\Delta\Re\{\epsilon\}$ &
    \textsf{(b)} $\Delta\Im\{\epsilon\}$ \\
    \hspace{1.0em}\includegraphics[bb=125 240 508 540,clip,scale=0.5]{graphics/Guillou/FWDTL/FWDde_a0.1000_re_7.25GHz.eps} &
    \includegraphics[bb=125 240 508 540,clip,scale=0.5]{graphics/Guillou/FWDTL/FWDde_a0.1000_im_7.25GHz.eps} \\\\
    \multicolumn{2}{c}{\sffamily\textbf{Tangent-linear response}}\\
    \textsf{(c)} $\Re\{\de\}$ &
    \textsf{(d)} $\Im\{\de\}$ \\
    \hspace{1.0em}\includegraphics[bb=125 240 508 540,clip,scale=0.5]{graphics/Guillou/FWDTL/TLde_a0.1000_re_7.25GHz.eps} &
    \includegraphics[bb=125 240 508 540,clip,scale=0.5]{graphics/Guillou/FWDTL/TLde_a0.1000_im_7.25GHz.eps} \\\\
    \multicolumn{2}{c}{\sffamily\textbf{Forward/tangent-linear test result}}\\
    \textsf{(e)} $|\Delta\Re\{\epsilon\} - \Re\{\de\}|$ &
    \textsf{(f)} $|\Delta\Im\{\epsilon\} - \Im\{\de\}|$ \\
    \includegraphics[bb=110 240 508 540,clip,scale=0.5]{graphics/Guillou/FWDTL/FWDTLtest_a0.1000_re_7.25GHz.eps} & 
    \includegraphics[bb=120 240 508 540,clip,scale=0.5]{graphics/Guillou/FWDTL/FWDTLtest_a0.1000_im_7.25GHz.eps}
  \end{tabular}
  \caption{Real and imaginary parts of the computed Guillou complex permittivities at 7.25GHz for the forward/tangent-linear test with $\alpha$=0.1. \textbf{(a)} Real component non-linear difference.  \textbf{(b)} Imaginary component non-linear difference. \textbf{(c)} Real component tangent-linear response. \textbf{(d)} Imaginary component tangent-linear response. \textbf{(e)} Real component test residual. \textbf{(f)} Imaginary component test residual.}
  \label{fig:fwdtl_a0.1000_guillou}
\end{figure}

\begin{figure}[htp]
  \centering
  \begin{tabular}{c c}
    \multicolumn{2}{c}{\sffamily\textbf{Non-linear difference}}\\
    \textsf{(a)} $\Delta\Re\{\epsilon\}$ &
    \textsf{(b)} $\Delta\Im\{\epsilon\}$ \\
    \hspace{1.0em}\includegraphics[bb=127 240 508 540,clip,scale=0.5]{graphics/Guillou/FWDTL/FWDde_a0.0001_re_16.25GHz.eps} &
    \hspace{1.0em}\includegraphics[bb=127 240 508 540,clip,scale=0.5]{graphics/Guillou/FWDTL/FWDde_a0.0001_im_16.25GHz.eps} \\\\
    \multicolumn{2}{c}{\sffamily\textbf{Tangent-linear response}}\\
    \textsf{(c)} $\Re\{\de\}$ &
    \textsf{(d)} $\Im\{\de\}$ \\
    \hspace{1.0em}\includegraphics[bb=127 240 508 540,clip,scale=0.5]{graphics/Guillou/FWDTL/TLde_a0.0001_re_16.25GHz.eps} &
    \hspace{1.0em}\includegraphics[bb=127 240 508 540,clip,scale=0.5]{graphics/Guillou/FWDTL/TLde_a0.0001_im_16.25GHz.eps} \\\\
    \multicolumn{2}{c}{\sffamily\textbf{Forward/tangent-linear test result}}\\
    \textsf{(e)} $|\Delta\Re\{\epsilon\} - \Re\{\de\}|$ &
    \textsf{(f)} $|\Delta\Im\{\epsilon\} - \Im\{\de\}|$ \\
    \includegraphics[bb=115 240 508 540,clip,scale=0.5]{graphics/Guillou/FWDTL/FWDTLtest_a0.0001_re_16.25GHz.eps} & 
    \includegraphics[bb=109 240 508 540,clip,scale=0.5]{graphics/Guillou/FWDTL/FWDTLtest_a0.0001_im_16.25GHz.eps}
  \end{tabular}
  \caption{Real and imaginary parts of the computed Guillou complex permittivities at 16.25GHz for the forward/tangent-linear test with $\alpha$=0.0001. \textbf{(a)} Real component non-linear difference.  \textbf{(b)} Imaginary component non-linear difference. \textbf{(c)} Real component tangent-linear response. \textbf{(d)} Imaginary component tangent-linear response. \textbf{(e)} Real component test residual. \textbf{(f)} Imaginary component test residual.}
  \label{fig:fwdtl_a0.0001_guillou}
\end{figure}


\subsubsection{TL/AD Test Results}
%.................................
\label{sec:tlad_guillou}
Following the description of the TL/AD tests in section \ref{sec:tlad_test} for routines with real valued input and complex valued output, the TL/AD test performed for the Guillou permittivity routines was,
\begin{equation}
  \underbrace{\left[\Re\{\de\}^{2} + \Im\{\de\}^{2}\right]}_{\mathbf{TL}^{T}\mathbf{TL}} - \underbrace{\left[\delta{T}.\dstar {T} + \delta{S}.\dstar {S}\right]}_{\mathbf{\delta x}^{T}\mathbf{AD}(TL)} = 0
  \label{eqn:tlad_guillou}
\end{equation}
where $T$ and $S$ are the sea surface temperature and salinity respectively (TL inputs are set to 0.1 in both cases), and $\epsilon$ is the complex permittivity. Examples of the intermediate and final quantities used in this test are shown in figure \ref{fig:tlad_7.25GHz_guillou} for $f = 7.25$GHz and \ref{fig:tlad_16.25GHz_guillou} for $f = 16.25$GHz. The differences between the values represented in figures \ref{fig:tlad_7.25GHz_guillou}(e) and (f) and \ref{fig:tlad_16.25GHz_guillou}(e) and (f) are shown in figure \ref{fig:tlad_test_guillou}. In both cases, the differences were within numerical precision. These results are typical of the other frequencies tested.

\begin{figure}[htp]
  \centering
  \begin{tabular}{c c}
    \multicolumn{2}{c}{\sffamily\textbf{Tangent-linear permittivity}}\\
    \textsf{(a)} $\Re\{\de\}$ &
    \textsf{(b)} $\Im\{\de\}$ \\
    \includegraphics[bb=125 240 508 540,clip,scale=0.5]{graphics/Guillou/TLAD/e_TL_re_7.25GHz.eps} &
    \includegraphics[bb=125 240 508 540,clip,scale=0.5]{graphics/Guillou/TLAD/e_TL_im_7.25GHz.eps} \\\\
    \multicolumn{2}{c}{\sffamily\textbf{Adjoint temperature and salinity}}\\
    \textsf{(c)} $\dstar {T}$ &
    \textsf{(d)} $\dstar {S}$ \\
    \includegraphics[bb=125 240 508 540,clip,scale=0.5]{graphics/Guillou/TLAD/t_AD_7.25GHz.eps} &
    \includegraphics[bb=120 240 508 540,clip,scale=0.5]{graphics/Guillou/TLAD/s_AD_7.25GHz.eps} \\\\
    \multicolumn{2}{c}{\sffamily\textbf{Test quantities}}\\
    \textsf{(e)} $\mathbf{TL}^{T}\mathbf{TL}$ &
    \textsf{(f)} $\mathbf{\delta x}^{T}\mathbf{AD}(TL)$ \\
    \includegraphics[bb=120 240 508 540,clip,scale=0.5]{graphics/Guillou/TLAD/TLtTL_7.25GHz.eps} & 
    \includegraphics[bb=120 240 508 540,clip,scale=0.5]{graphics/Guillou/TLAD/dxtAD_7.25GHz.eps}
  \end{tabular}
  \caption{Example of quantities used to test the TL/AD Guillou permittivity routines for $\delta{T}$ and $\delta{S}$ inputs of 0.1 at 7.25GHz. \textbf{(a)} Real component of the tangent-linear permittivity.  \textbf{(b)} Imaginary component of the tangent-linear permittivity. \textbf{(c)} Temperature adjoint. \textbf{(d)} Salinity adjoint. \textbf{(e)} Tangent-linear test result (see eqn.\ref{eqn:tlad_guillou}). \textbf{(f)} Adjoint test result (see eqn.\ref{eqn:tlad_guillou}).}
  \label{fig:tlad_7.25GHz_guillou}
\end{figure}

\begin{figure}[htp]
  \centering
  \begin{tabular}{c c}
    \multicolumn{2}{c}{\sffamily\textbf{Tangent-linear permittivity}}\\
    \textsf{(a)} $\Re\{\de\}$ &
    \textsf{(b)} $\Im\{\de\}$ \\
    \includegraphics[bb=130 240 508 540,clip,scale=0.5]{graphics/Guillou/TLAD/e_TL_re_16.25GHz.eps} &
    \includegraphics[bb=125 240 508 540,clip,scale=0.5]{graphics/Guillou/TLAD/e_TL_im_16.25GHz.eps} \\\\
    \multicolumn{2}{c}{\sffamily\textbf{Adjoint temperature and salinity}}\\
    \textsf{(c)} $\dstar {T}$ &
    \textsf{(d)} $\dstar {S}$ \\
    \includegraphics[bb=130 240 508 540,clip,scale=0.5]{graphics/Guillou/TLAD/t_AD_16.25GHz.eps} &
    \includegraphics[bb=120 240 508 540,clip,scale=0.5]{graphics/Guillou/TLAD/s_AD_16.25GHz.eps} \\\\
    \multicolumn{2}{c}{\sffamily\textbf{Test quantities}}\\
    \textsf{(e)} $\mathbf{TL}^{T}\mathbf{TL}$ &
    \textsf{(f)} $\mathbf{\delta x}^{T}\mathbf{AD}(TL)$ \\
    \includegraphics[bb=120 240 508 540,clip,scale=0.5]{graphics/Guillou/TLAD/TLtTL_16.25GHz.eps} & 
    \includegraphics[bb=120 240 508 540,clip,scale=0.5]{graphics/Guillou/TLAD/dxtAD_16.25GHz.eps}
  \end{tabular}
  \caption{Example of quantities used to test the TL/AD Guillou permittivity routines for $\delta{T}$ and $\delta{S}$ inputs of 0.1 at 16.25GHz. \textbf{(a)} Real component of the tangent-linear permittivity.  \textbf{(b)} Imaginary component of the tangent-linear permittivity. \textbf{(c)} Temperature adjoint. \textbf{(d)} Salinity adjoint. \textbf{(e)} Tangent-linear test result (see eqn.\ref{eqn:tlad_guillou}). \textbf{(f)} Adjoint test result (see eqn.\ref{eqn:tlad_guillou}).}
  \label{fig:tlad_16.25GHz_guillou}
\end{figure}

\begin{figure}[htp]
  \centering
  \begin{tabular}{c}
    \textsf{(a) $\mathbf{TL}^{T}\mathbf{TL} - \mathbf{\delta x}^{T}\mathbf{AD}(TL)$ for $f$=7.25GHz}\\
    \includegraphics[bb=115 240 508 525,clip,scale=0.8]{graphics/Guillou/TLAD/TLtTL-dxtAD_7.25GHz.eps}\\\\\\
    \textsf{(b) $\mathbf{TL}^{T}\mathbf{TL} - \mathbf{\delta x}^{T}\mathbf{AD}(TL)$ for $f$=16.25GHz}\\
    \includegraphics[bb=115 240 508 525,clip,scale=0.8]{graphics/Guillou/TLAD/TLtTL-dxtAD_16.25GHz.eps}\\\\
  \end{tabular}
  \caption{Guillou permittivity model TL/AD test results for the two test frequencies indicating TL/AD agreement to numerical precision. \textbf{(a)} Result for 7.25GHz (See figure \ref{fig:tlad_7.25GHz_guillou}). \textbf{(b)} Result for 16.25GHz (See figure \ref{fig:tlad_16.25GHz_guillou}).}
  \label{fig:tlad_test_guillou}
\end{figure}



\subsection{Fresnel Reflectivity}
%--------------------------------
The derivations of the Fresnel reflectivity equations used in the model are given in appendix \ref{sec:fresnel_equations}. This section details the forward/tangent-linear and tangent-linear/adjoint tests performed on the Fresnel reflectivity procedures. The number and range of input quantities used in the tests are shown in table \ref{tab:fresnel_input_range}. A selection of computed forward model reflectivities at different incidence angles are shown in figure \ref{fig:fresnel_reflectivity}.
\begin{table}[htp]
  \centering
  \begin{tabular}{| c | c | r@{.}l@{ - }r@{.}l | c |}
    \hline
    \textbf{Quantity} & \textbf{\# of Values} & \multicolumn{4}{c|}{\textbf{Range}} & \textbf{Units} \\
    \hline\hline
    Angle, $\theta_i$ &  7 &  0&0 &  60&0 & degrees \\
    $\Re\{\epsilon\}$ & 21 &  5&0 &  75&0 & F.m$^{-1}$ (?) \\
    $\Im\{\epsilon\}$ & 21 & -5&0 & -31&0 & F.m$^{-1}$ (?) \\
    \hline
  \end{tabular}
  \caption{Range of test input data to the Fresnel reflectivity procedures.}
  \label{tab:fresnel_input_range}
\end{table}

\begin{figure}[htp]
  \centering
  \begin{tabular}{c c}
    \multicolumn{2}{c}{\boldmath$\theta_i$\unboldmath\sffamily\textbf{=0.0}\boldmath$^\circ$\unboldmath}\\
    \textsf{(a) Vertical} &
    \textsf{(b) Horizontal} \\
    \includegraphics[bb=135 240 508 540,clip,scale=0.5]{graphics/Fresnel/rv_z0.0.eps} &
    \includegraphics[bb=135 240 508 540,clip,scale=0.5]{graphics/Fresnel/rh_z0.0.eps} \\\\

    \multicolumn{2}{c}{\boldmath$\theta_i$\unboldmath\sffamily\textbf{=30.0}\boldmath$^\circ$\unboldmath}\\
    \textsf{(c) Vertical} &
    \textsf{(d) Horizontal} \\
    \includegraphics[bb=135 240 508 540,clip,scale=0.5]{graphics/Fresnel/rv_z30.0.eps} &
    \includegraphics[bb=135 240 508 540,clip,scale=0.5]{graphics/Fresnel/rh_z30.0.eps} \\\\

    \multicolumn{2}{c}{\boldmath$\theta_i$\unboldmath\sffamily\textbf{=60.0}\boldmath$^\circ$\unboldmath}\\
    \textsf{(e) Vertical} &
    \textsf{(f) Horizontal} \\
    \includegraphics[bb=135 240 508 540,clip,scale=0.5]{graphics/Fresnel/rv_z60.0.eps} &
    \includegraphics[bb=135 240 508 540,clip,scale=0.5]{graphics/Fresnel/rh_z60.0.eps}
  \end{tabular}
  \caption{Vertical and horizontal Fresnel reflectivities as a function of the real and imaginary part of the permittivity for three incidence angles.}
  \label{fig:fresnel_reflectivity}
\end{figure}


\subsubsection{FWD/TL Test Results}
%..................................
The description of the FWD/TL tests for routines with reall valued output was given in section \ref{sec:fwdtl_test}. Some representative results for the Fresnel reflectivity FWD/TL tests are shown in figure \ref{fig:fwdtl_a0.1000_fresnel} for an incidence angle of $20^{\circ}$ and an alpha value of 0.1 and in figure \ref{fig:fwdtl_a0.0001_fresnel} for an incidence angle of $40^{\circ}$ and an alpha value of 0.0001. The maximum tolerance residual for each value of alpha is shown in table \ref{tab:fwdtl_fresnel_alpha}.
\begin{table}[htp]
  \centering
  \begin{tabular}{| c | c |}
    \hline
    \boldmath$\alpha$\unboldmath & \textbf{Tolerance residual,} \boldmath$t_r$\unboldmath \\
    \hline\hline
    0.1    & 7.0e-09 \\
    0.01   & 7.0e-11 \\
    0.001  & 7.0e-13 \\
    0.0001 & 3.0e-12 \\
    \hline
  \end{tabular}
  \caption{Maximum tolerance residuals for the Fresnel reflectivity FWD/TL tests.}
  \label{tab:fwdtl_fresnel_alpha}
\end{table}
As with the permittivity FWD/TL test, as the alpha value decreases so does the tolerance residual. Interestingly, the tolerance residual for the smallest alpha value follows the same pattern as for the permittivity in that it is an order of magnitude larger than that for the next larger alpha value case. Additional tests showed that as alpha is decreased even further, the associated tolerance residual steadily increases indicating the results are at the precision limit.

\begin{figure}[htp]
  \centering
  \begin{tabular}{c c}
    \multicolumn{2}{c}{\sffamily\textbf{Non-linear difference}}\\
    \textsf{(a)} $\Delta r_v$ &
    \textsf{(b)} $\Delta r_h$ \\
    \includegraphics[bb=120 240 508 540,clip,scale=0.5]{graphics/Fresnel/FWDTL/FWDdrv_a0.1000_z20.0.eps} &
    \includegraphics[bb=120 240 508 540,clip,scale=0.5]{graphics/Fresnel/FWDTL/FWDdrh_a0.1000_z20.0.eps} \\\\
    \multicolumn{2}{c}{\sffamily\textbf{Tangent-linear response}}\\
    \textsf{(c)} $\delta r_v$ &
    \textsf{(d)} $\delta r_h$ \\
    \includegraphics[bb=120 240 508 540,clip,scale=0.5]{graphics/Fresnel/FWDTL/TLdrv_a0.1000_z20.0.eps} &
    \includegraphics[bb=120 240 508 540,clip,scale=0.5]{graphics/Fresnel/FWDTL/TLdrh_a0.1000_z20.0.eps} \\\\
    \multicolumn{2}{c}{\sffamily\textbf{Forward/tangent-linear test result}}\\
    \textsf{(e)} $|\Delta r_v - \delta r_v|$ &
    \textsf{(f)} $|\Delta r_h - \delta r_h|$ \\
    \includegraphics[bb=120 240 508 540,clip,scale=0.5]{graphics/Fresnel/FWDTL/FWDTLtestrv_a0.1000_z20.0.eps} & 
    \includegraphics[bb=120 240 508 540,clip,scale=0.5]{graphics/Fresnel/FWDTL/FWDTLtestrh_a0.1000_z20.0.eps}
  \end{tabular}
  \caption{Vertical and horizontal Fresnel reflectivities at $\theta_i$=20.0$^\circ$ for the forward/tangent-linear test with $\alpha$=0.1. \textbf{(a)} Vertical component non-linear difference.  \textbf{(b)} Horizontal component non-linear difference. \textbf{(c)} Vertical component tangent-linear response. \textbf{(d)} Horizontal component tangent-linear response. \textbf{(e)} Vertical component test residual. \textbf{(f)} Horizontal component test residual.}
  \label{fig:fwdtl_a0.1000_fresnel}
\end{figure}

\begin{figure}[htp]
  \centering
  \begin{tabular}{c c}
    \multicolumn{2}{c}{\sffamily\textbf{Non-linear difference}}\\
    \textsf{(a)} $\Delta r_v$ &
    \textsf{(b)} $\Delta r_h$ \\
    \includegraphics[bb=115 240 508 540,clip,scale=0.5]{graphics/Fresnel/FWDTL/FWDdrv_a0.0001_z40.0.eps} &
    \includegraphics[bb=120 240 508 540,clip,scale=0.5]{graphics/Fresnel/FWDTL/FWDdrh_a0.0001_z40.0.eps} \\\\
    \multicolumn{2}{c}{\sffamily\textbf{Tangent-linear response}}\\
    \textsf{(c)} $\delta r_v$ &
    \textsf{(d)} $\delta r_h$ \\
    \includegraphics[bb=115 240 508 540,clip,scale=0.5]{graphics/Fresnel/FWDTL/TLdrv_a0.0001_z40.0.eps} &
    \includegraphics[bb=120 240 508 540,clip,scale=0.5]{graphics/Fresnel/FWDTL/TLdrh_a0.0001_z40.0.eps} \\\\
    \multicolumn{2}{c}{\sffamily\textbf{Forward/tangent-linear test result}}\\
    \textsf{(e)} $|\Delta r_v - \delta r_v|$ &
    \textsf{(f)} $|\Delta r_h - \delta r_h|$ \\
    \includegraphics[bb=110 240 508 540,clip,scale=0.5]{graphics/Fresnel/FWDTL/FWDTLtestrv_a0.0001_z40.0.eps} & 
    \includegraphics[bb=110 240 508 540,clip,scale=0.5]{graphics/Fresnel/FWDTL/FWDTLtestrh_a0.0001_z40.0.eps}
  \end{tabular}
  \caption{Vertical and horizontal Fresnel reflectivities at $\theta_i$=40.0$^\circ$ for the forward/tangent-linear test with $\alpha$=0.0001. \textbf{(a)} Vertical component non-linear difference.  \textbf{(b)} Horizontal component non-linear difference. \textbf{(c)} Vertical component tangent-linear response. \textbf{(d)} Horizontal component tangent-linear response. \textbf{(e)} Vertical component test residual. \textbf{(f)} Horizontal component test residual.}
  \label{fig:fwdtl_a0.0001_fresnel}
\end{figure}


\subsubsection{TL/AD Test Results}
%.................................
Following the description of the TL/AD test in section \ref{sec:tlad_test} for routines with complex valued input and real valued output, the TL/AD test performed for the Fresnel reflectivity routines was,
\begin{equation}
  \underbrace{\left[\delta r_{v}^2 + \delta r_{h}^2\right]}_{\mathbf{TL}^{T}\mathbf{TL}} - \underbrace{\left[\Re\{\de\}.\Re\{\dstar \epsilon\} + \Im\{\de\}.\Im\{\dstar \epsilon\}\right]}_{\mathbf{\delta x}^{T}\mathbf{AD}(TL)} = 0
  \label{eqn:tlad_fresnel}
\end{equation}
where $\epsilon$ is the complex permittivity (TL inputs set to 0.1), and $r_v$ and $r_h$ are the vertical and horizontal reflectivities respectively. Examples of the intermediate and final quantities used in this test are shown in figure \ref{fig:tlad_z20.0_fresnel} for $\theta_{i} = 20^{\circ}$, and figure \ref{fig:tlad_z40.0_fresnel} for $\theta_{i} = 40^{\circ}$. The differences between figure \ref{fig:tlad_z20.0_fresnel}(e) and (f), and figure \ref{fig:tlad_z40.0_fresnel}(e) and (f) are shown in figure \ref{fig:tlad_test_fresnel}. In both cases, the differences are within numerical precision. These results are typical of the other incidence angles tested.

\begin{figure}[htp]
  \centering
  \begin{tabular}{c c}
    \multicolumn{2}{c}{\sffamily\textbf{Tangent-linear reflectivites}}\\
    \textsf{(a)} $\delta r_v$ &
    \textsf{(b)} $\delta r_h$ \\
    \includegraphics[bb=120 240 508 540,clip,scale=0.5]{graphics/Fresnel/TLAD/rv_TL_z20.0.eps} &
    \includegraphics[bb=120 240 508 540,clip,scale=0.5]{graphics/Fresnel/TLAD/rv_TL_z20.0.eps} \\\\
    \multicolumn{2}{c}{\sffamily\textbf{Adjoint permittivities}}\\
    \textsf{(c)} $\Re\{\dstar \epsilon\}$ &
    \textsf{(d)} $\Im\{\dstar \epsilon\}$ \\
    \includegraphics[bb=115 240 508 540,clip,scale=0.5]{graphics/Fresnel/TLAD/Re_e_AD_z20.0.eps} &
    \includegraphics[bb=110 240 508 540,clip,scale=0.5]{graphics/Fresnel/TLAD/Im_e_AD_z20.0.eps} \\\\
    \multicolumn{2}{c}{\sffamily\textbf{Test quantities}}\\
    \textsf{(e)} $\mathbf{TL}^{T}\mathbf{TL}$ &
    \textsf{(f)} $\mathbf{\delta x}^{T}\mathbf{AD}(TL)$ \\
    \includegraphics[bb=110 240 508 540,clip,scale=0.5]{graphics/Fresnel/TLAD/TLtTL_z20.0.eps} & 
    \includegraphics[bb=110 240 508 540,clip,scale=0.5]{graphics/Fresnel/TLAD/dxtAD_z20.0.eps}
  \end{tabular}
  \caption{Example of quantities used to test the TL/AD Fresnel reflectivity routines for $\Re\{\de\}$ and $\Im\{\de\}$ inputs of 0.1 at an incidence angle of 20$^{\circ}$. \textbf{(a)} Tangent-linear vertical reflectivity. \textbf{(b)} Tangent-linear horizontal reflectivity. \textbf{(c)} Real component of the adjoint permittivity.  \textbf{(d)} Imaginary component of the adjoint permittivity. \textbf{(e)} Tangent-linear test result (see eqn.\ref{eqn:tlad_fresnel}). \textbf{(f)} Adjoint test result (see eqn.\ref{eqn:tlad_fresnel}).}
  \label{fig:tlad_z20.0_fresnel}
\end{figure}

\begin{figure}[htp]
  \centering
  \begin{tabular}{c c}
    \multicolumn{2}{c}{\sffamily\textbf{Tangent-linear reflectivites}}\\
    \textsf{(a)} $\delta r_v$ &
    \textsf{(b)} $\delta r_h$ \\
    \includegraphics[bb=115 240 508 540,clip,scale=0.5]{graphics/Fresnel/TLAD/rv_TL_z40.0.eps} &
    \includegraphics[bb=115 240 508 540,clip,scale=0.5]{graphics/Fresnel/TLAD/rv_TL_z40.0.eps} \\\\
    \multicolumn{2}{c}{\sffamily\textbf{Adjoint permittivities}}\\
    \textsf{(c)} $\Re\{\dstar \epsilon\}$ &
    \textsf{(d)} $\Im\{\dstar \epsilon\}$ \\
    \includegraphics[bb=115 240 508 540,clip,scale=0.5]{graphics/Fresnel/TLAD/Re_e_AD_z40.0.eps} &
    \includegraphics[bb=115 240 508 540,clip,scale=0.5]{graphics/Fresnel/TLAD/Im_e_AD_z40.0.eps} \\\\
    \multicolumn{2}{c}{\sffamily\textbf{Test quantities}}\\
    \textsf{(e)} $\mathbf{TL}^{T}\mathbf{TL}$ &
    \textsf{(f)} $\mathbf{\delta x}^{T}\mathbf{AD}(TL)$ \\
    \includegraphics[bb=110 240 508 540,clip,scale=0.5]{graphics/Fresnel/TLAD/TLtTL_z40.0.eps} & 
    \includegraphics[bb=110 240 508 540,clip,scale=0.5]{graphics/Fresnel/TLAD/dxtAD_z40.0.eps}
  \end{tabular}
  \caption{Example of quantities used to test the TL/AD Fresnel reflectivity routines for $\Re\{\de\}$ and $\Im\{\de\}$ inputs of 0.1 at an incidence angle of 40$^{\circ}$. \textbf{(a)} Tangent-linear vertical reflectivity. \textbf{(b)} Tangent-linear horizontal reflectivity. \textbf{(c)} Real component of the adjoint permittivity.  \textbf{(d)} Imaginary component of the adjoint permittivity. \textbf{(e)} Tangent-linear test result (see eqn.\ref{eqn:tlad_fresnel}). \textbf{(f)} Adjoint test result (see eqn.\ref{eqn:tlad_fresnel}).}
  \label{fig:tlad_z40.0_fresnel}
\end{figure}

\begin{figure}[htp]
  \centering
  \begin{tabular}{c}
    \textsf{(a) $\mathbf{TL}^{T}\mathbf{TL} - \mathbf{\delta x}^{T}\mathbf{AD}(TL)$ for $\theta_i$=20.0$^\circ$}\\
    \hspace{1em}\includegraphics[bb=115 240 508 525,clip,scale=0.8]{graphics/Fresnel/TLAD/TLtTL-dxtAD_z20.0.eps}\\\\\\
    \textsf{(b) $\mathbf{TL}^{T}\mathbf{TL} - \mathbf{\delta x}^{T}\mathbf{AD}(TL)$ for $\theta_i$=40.0$^\circ$}\\
    \includegraphics[bb=105 240 508 525,clip,scale=0.8]{graphics/Fresnel/TLAD/TLtTL-dxtAD_z40.0.eps}\\\\
  \end{tabular}
  \caption{Fresnel reflectivity model TL/AD test results for the two test incidence angles indicating TL/AD agreement to numerical precision. \textbf{(a)} Result for 20$^{\circ}$ (See figure \ref{fig:tlad_z20.0_fresnel}). \textbf{(b)} Result for 40$^{\circ}$ (See figure \ref{fig:tlad_z40.0_fresnel}).}
  \label{fig:tlad_test_fresnel}
\end{figure}



\section{Conclusions}
%====================
The low frequency microwave sea surface emissivity model has been shown to be internally consistent across its forward, tangent-linear, and adjoint forms.

Validating the forward/tangent-linear model consistency for frequencies greater than 15GHz proved slightly difficult due to the impact that the noisy ocean height variance LUT data had on the resultant emissivities due to the applied small-scale correction. Smoothing the ocean height variance data did decrease the test residuals, but did not eliminate particular features associated with interpolation across LUT hingepoints. Visual inspection of a selection of FWD/TL residuals for various temperatures, salinities, and wind speeds was required to verify the tests. No rigorous objective method was found that could be applied successfully for all combinations of inputs.

Validation of the tangent-linear/adjoint model was comparatively easy in that all test residuals could be objectively compared to within numerical precision. All model components passed this test. 

The impact of the updated microwave sea surface emissivity model on computed brightness temperatures in the CRTM can be quite large, 20-30K, for those channels that are senstive to the surface. The largest portion of the change is due to the emissivity model in the current v1.1 release of the CRTM being Fastem1 which is known to not handle low frequencies very well. However, for the very lowest frequencies tested, the low frequency model still produces an additional brightness temperature difference from Fastem3 of the order of 4-8K. 




% The references section
%=======================
\bibliographystyle{plain}
\bibliography{bibliography}	


% The appendices section
%=======================
\begin{appendix}
  \section{Fresnel Reflectivity Derivation}
%========================================
\label{sec:fresnel_equations}
This section merely derives the Fresnel reflectivity equations used in the CRTM microwave sea surface emissivity from the more typical equations.

As defined in section 1.5.2 in \citet{BornWolf_1999}, the complex amplitudes of the reflected waves parallel (vertical) and perpendicular (horizontal) to the plane of incidence of an air/ocean water interface are given by,
\begin{equation}
  R_{\parallel} = \frac{n_{2}\cos\theta_{i} - n_{1}\cos\theta_{t}}{n_{2}\cos\theta_{i} + n_{1}\cos\theta_{t}}.A_{\parallel}
  \label{eqn:Rv_amplitude}
\end{equation}
and
\begin{equation}
  R_{\perp} = \frac{n_{1}\cos\theta_{i} - n_{2}\cos\theta_{t}}{n_{1}\cos\theta_{i} + n_{2}\cos\theta_{t}}.A_{\perp}
  \label{eqn:Rh_amplitude}
\end{equation}
where $n_{1}$ and $n_{2}$ are the refractive indices of air and ocean water respectively, $\theta_{i}$ and $\theta_{t}$ the angles of incidence and transmission respectively, and $A$ represents the complex amplitude of the incident wave. Additionally, the reflectivity is given by the ratio,
\begin{equation}
  r = \frac{|R|^{2}}{|A|^{2}}
  \label{eqn:reflectivity}
\end{equation}

The refractive index of a medium can be expressed via Maxwell's formula,
\begin{equation*}
  n = \sqrt{\epsilon\mu}
\end{equation*}
where we can assume the magnetic permeability of ocean water is unity, such that
\begin{equation}
  n_{2} = \sqrt{\epsilon}
  \label{eqn:maxwells}
\end{equation}
Setting the refractive index of air to a value of 1.0, we can use equation \ref{eqn:maxwells} in Snell's law to obtain a substitution for $\cos\theta_{t}$,
\begin{eqnarray}
  \sqrt{\epsilon}\,\sin\theta_{t}  &=& \sin\theta_{i}\nonumber\\
  \textrm{i.e. }\sin^{2}\theta_{t} &=& \frac{\sin^{2}\theta_{i}}{\epsilon}\nonumber\\
  \cos^{2}\theta_{t}               &=& 1 - \frac{1 - \cos^{2}\theta_{i}}{\epsilon}\nonumber\\
                                   &=& \frac{\epsilon - 1 + \cos^{2}\theta_{i}}{\epsilon}\nonumber\\
  \therefore\:\cos\theta_{t}       &=& \sqrt{\frac{\epsilon - 1 + \cos^{2}\theta_{i}}{\epsilon}}
  \label{eqn:cost_substitution}
\end{eqnarray}
Substituting equations \ref{eqn:maxwells} and \ref{eqn:cost_substitution} into equation \ref{eqn:Rv_amplitude} and multiplying by $\sqrt{\epsilon}/\sqrt{\epsilon}$ we get,
\begin{eqnarray}
  R_{\parallel} &=& \frac{\sqrt{\epsilon}\,\cos\theta_{i} - \sqrt{\displaystyle\frac{\epsilon - 1 + \cos^{2}\theta_{i}}{\epsilon}}}{\sqrt{\epsilon}\,\cos\theta_{i} + \sqrt{\displaystyle\frac{\epsilon - 1 + \cos^{2}\theta_{i}}{\epsilon}}}.\frac{\sqrt{\epsilon}}{\sqrt{\epsilon}}.A_{\parallel}\nonumber\\\nonumber\\
                &=& \frac{\epsilon\,\cos\theta_{i} - \sqrt{\epsilon - 1 + \cos^{2}\theta_{i}}}{\epsilon\,\cos\theta_{i} + \sqrt{\epsilon - 1 + \cos^{2}\theta_{i}}}.A_{\parallel}
  \label{eqn:Rv_amplitude2}
\end{eqnarray}
with the reflectivity given by equation \ref{eqn:reflectivity}
\begin{equation}
  r_{\parallel} = \left|\frac{\epsilon\,\cos\theta_{i} - \sqrt{\epsilon - 1 + \cos^{2}\theta_{i}}}{\epsilon\,\cos\theta_{i} + \sqrt{\epsilon - 1 + \cos^{2}\theta_{i}}}\right|^{2}
  \label{eqn:Rv_reflectivity}
\end{equation}
Similarly for the equation \ref{eqn:Rh_amplitude},
\begin{eqnarray}
  R_{\perp} &=& \frac{\cos\theta_{i} - \sqrt{\epsilon}\,\sqrt{\displaystyle\frac{\epsilon - 1 + \cos^{2}\theta_{i}}{\epsilon}}}{\cos\theta_{i} + \sqrt{\epsilon}\,\sqrt{\displaystyle\frac{\epsilon - 1 + \cos^{2}\theta_{i}}{\epsilon}}}.A_{\perp}\nonumber\\\nonumber\\
                &=& \frac{\cos\theta_{i} - \sqrt{\epsilon - 1 + \cos^{2}\theta_{i}}}{\cos\theta_{i} + \sqrt{\epsilon - 1 + \cos^{2}\theta_{i}}}.A_{\perp}
  \label{eqn:Rh_amplitude2}
\end{eqnarray}
with reflectivity,
\begin{equation}
  r_{\perp} = \left|\frac{\cos\theta_{i} - \sqrt{\epsilon - 1 + \cos^{2}\theta_{i}}}{\cos\theta_{i} + \sqrt{\epsilon - 1 + \cos^{2}\theta_{i}}}\right|^{2}
  \label{eqn:Rh_reflectivity}
\end{equation}


  \section{Impact on computational speed due to polynomial calculations}
%=====================================================================
\label{sec:Poly_Routine_Speed.appendix}
Both the \citet{Ellison_2003} and \citet{Guillou_1998} emissivity models use polynomial fits to data to repesent various components of the complex ocean surface permittivity. Forward, tangent-linear, and adjoint polynomial computation routines were written to simplify their evaluation.

The permittivity codes were profiled with the polynomial calculations performed inline, and using the polynomial routines with the profiler outputs. The results are shown in figures \ref{fig:ellison_nopoly_profile} and \ref{fig:ellison_poly_profile} for the Ellison code, and figures \ref{fig:guillou_nopoly_profile} and \ref{fig:guillou_poly_profile} for the Guillou code. On average, the inline polynomial evaluation results in the test code running approximately two (Ellison) to three (Guillou) times faster than if evaluation routines were used. The larger impact for the Guillou code is likely due to there being more polynomials to evaluate, many with six coefficients.

Note that the 2-3$\times$ factor from the profiling tests does not necessarily translate into wall-clock time.

\begin{figure}[htp]
  \centering
  \doublebox{
  \begin{minipage}[b]{6.5in}
    \begin{ttfamily}
      \begin{verbatim}
Each sample counts as 0.01 seconds.
  %   cumulative  self                     
 time   seconds  seconds    calls  name    
 31.43    2.36    2.36          1  MAIN__
 22.17    4.03    1.67   16669800  unit_test_MOD_fp_equal_within_scalar
 21.97    5.68    1.65    9724050  ocean_permittivity_MOD_ellison_ocean_permittivity
 19.04    7.11    1.43    9261000  ocean_permittivity_MOD_ellison_ocean_permittivity_tl
  3.46    7.37    0.26    1852200  ocean_permittivity_MOD_ellison_ocean_permittivity_ad
  0.87    7.43    0.07   16669800  unit_test_MOD_last_test_failed
  0.73    7.49    0.06   16669800  unit_test_MOD_test_passed
  0.33    7.51    0.03             unit_test_MOD_fp_equal_within_rank1
      \end{verbatim}
    \end{ttfamily}
  \end{minipage}
  }
  \caption{Profile results for Ellison permittivity tests with polynomial evaluation performed inline.}
  \label{fig:ellison_nopoly_profile}
\end{figure}

\begin{figure}[htp]
  \centering
  \doublebox{
  \begin{minipage}[b]{6.5in}
  \begin{ttfamily}
    \begin{verbatim}
Each sample counts as 0.01 seconds.
  %   cumulative  self                      
 time   seconds  seconds    calls  name    
 28.67    4.09    4.09   38896200  ocean_permittivity_MOD_poly
 18.28    6.69    2.61   37044000  ocean_permittivity_MOD_poly_tl
 14.11    8.70    2.01          1  MAIN__
 12.32   10.46    1.76    9724050  ocean_permittivity_MOD_ellison_ocean_permittivity
 10.77   11.99    1.54   16669800  unit_test_MOD_fp_equal_within_scalar
  8.00   13.13    1.14    9261000  ocean_permittivity_MOD_ellison_ocean_permittivity_tl
  3.47   13.63    0.50    7408800  ocean_permittivity_MOD_poly_ad
  2.84   14.03    0.41    1852200  ocean_permittivity_MOD_ellison_ocean_permittivity_ad
  0.84   14.15    0.12   16669800  unit_test_MOD_test_passed
  0.56   14.23    0.08   16669800  unit_test_MOD_last_test_failed
  0.07   14.24    0.01             unit_test_MOD_test_failed
  0.04   14.25    0.01          1  timing_utility_MOD_begin_timing
    \end{verbatim}
  \end{ttfamily}
  \end{minipage}
  }
  \caption{Profile results for Ellison permittivity tests using polynomial evaluation subroutines}
  \label{fig:ellison_poly_profile}
\end{figure}

\begin{figure}[htp]
  \centering
  \doublebox{
  \begin{minipage}[b]{6.5in}
    \begin{ttfamily}
      \begin{verbatim}
Each sample counts as 0.01 seconds.
  %   cumulative   self                      
 time   seconds   seconds    calls   name    
 25.66    2.15     2.15          1   MAIN__
 24.94    4.24     2.09    9261000   ocean_permittivity_MOD_guillou_ocean_permittivity_tl
 24.46    6.29     2.05    9724050   ocean_permittivity_MOD_guillou_ocean_permittivity
 16.77    7.70     1.41   16669800   unit_test_MOD_fp_equal_within_scalar
  5.01    8.12     0.42    1852200   ocean_permittivity_MOD_guillou_ocean_permittivity_ad
  2.09    8.29     0.18   16669800   unit_test_MOD_test_passed
  0.90    8.37     0.08   14817600   unit_test_MOD_last_test_failed
  0.12    8.38     0.01              unit_test_MOD_test_failed
      \end{verbatim}
    \end{ttfamily}
  \end{minipage}
  }
  \caption{Profile results for Guillou permittivity tests with polynomial evaluation performed inline.}
  \label{fig:guillou_nopoly_profile}
\end{figure}

\begin{figure}[htp]
  \centering
  \doublebox{
  \begin{minipage}[b]{6.5in}
  \begin{ttfamily}
    \begin{verbatim}
Each sample counts as 0.01 seconds.
  %   cumulative   self                      
 time   seconds   seconds    calls   name    
 38.64    9.85     9.85   68068350   ocean_permittivity_MOD_poly
 21.76   15.39     5.55   64827000   ocean_permittivity_MOD_poly_tl
  9.50   17.81     2.42    9724051   ocean_permittivity_MOD_guillou_ocean_permittivity
  9.34   20.19     2.38              MAIN__
  7.10   22.00     1.81   16669800   unit_test_MOD_fp_equal_within_scalar
  5.89   23.50     1.50    9261000   ocean_permittivity_MOD_guillou_ocean_permittivity_tl
  5.30   24.85     1.35   12965400   ocean_permittivity_MOD_poly_ad
  1.37   25.20     0.35    1852200   ocean_permittivity_MOD_guillou_ocean_permittivity_ad
  0.47   25.32     0.12   16669800   unit_test_MOD_test_passed
  0.31   25.40     0.08   14817600   unit_test_MOD_last_test_failed
  0.12   25.43     0.03              unit_test_MOD_fp_equal_within_rank1
  0.12   25.46     0.03              unit_test_MOD_test_failed
  0.08   25.48     0.02              ocean_permittivity_MOD_ellison_ocean_permittivity_ad
    \end{verbatim}
  \end{ttfamily}
  \end{minipage}
  }
  \caption{Profile results for Guillou permittivity tests using polynomial evaluation subroutines}
  \label{fig:guillou_poly_profile}
\end{figure}


\end{appendix}

\end{document}

