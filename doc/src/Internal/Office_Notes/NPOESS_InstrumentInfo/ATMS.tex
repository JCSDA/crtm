\section{ATMS (Advanced Technology Microwave Sounder)}
%=====================================================

\subsection{Channel type definitions}
%------------------------------------
In generating the line-by-line transmittances for microwave instruments, the ``intermediate frequencies'' are used to define the sensor responses. The ATMS will be treated similarly to heterodyne receivers like the AMSU-A instruments where the received radio frequency (RF) is downconverted to a lower intermediate frequency (IF). The relationship between the intermediate and measurement frequencies is shown in figure \ref{fig:broadband_frequency_translation}. Note that this treatment is purely schematic for explanatory purposes only - it should not be construed as a description of the construction or behaviour of the ATMS itself.
\begin{figure}[htp]
  \centering
  \input{graphics/broadband_frequency_translation.pstex_t}
  \caption{Frequency translation in heterodyne reception for a broadband signal. Adapted from fig.1.9b in \citet{Janssen_1993}}
  \label{fig:broadband_frequency_translation}
\end{figure}

For modeling the ATMS, the channels are divided into three types,
\begin{itemize}
  \item{Single passband channels (1-5,7-10,16,17)}
  \item{Double sideband channels (6,11,18-22)}
  \item{Quadruple sideband channels (12-15)}
\end{itemize}
Single passband channels are defined as those whose bandwidth span the channel centre frequency, as shown in figure \ref{fig:single_passband}. Typically for these channels stopbands are specified to reduce the effects of local oscillator noise, although no information is currently available on whetrher or not the ATMS single passband channels have stopbands. Double sideband channels are shown schematically in figure \ref{fig:double_sideband}. These channels are also referred to as folded passbands with the lower frequency sideband referred to as the lower sideband and the higher frequency sideband being the upper sideband. Quadruple sideband channels are shown schematically in figure \ref{fig:quadruple_sideband}. Note that all of the schematic channel definitions in figures \ref{fig:double_sideband} and \ref{fig:quadruple_sideband} assume bandwidth symmetry about the central and first sideband offset frequencies.
\begin{figure}[htp]
  \centering
  \input{graphics/single_passband.pstex_t}
  \caption{Schematic illustration of a single passband microwave channel}
  \label{fig:single_passband}
\end{figure}
\begin{figure}[htp]
  \centering
  \input{graphics/double_sideband.pstex_t}
  \caption{Schematic illustration of a double sideband microwave channel}
  \label{fig:double_sideband}
\end{figure}
\begin{figure}[htp]
  \centering
  \input{graphics/quadruple_sideband.pstex_t}
  \caption{Schematic illustration of a quadruple sideband microwave channel}
  \label{fig:quadruple_sideband}
\end{figure}

\subsection{Channel Frequencies}
%-------------------------------
The central and sideband offset frequencies listed here are taken from \citet{Muth_2004}, and the bandwidth frequencies are taken from \cite{ATMS_PFM_CalLog}; these data are shown in table \ref{tab:atms_fo_sb_and_df}.
\begin{table}[htp]
  \centering
  \begin{tabular}{|c|c|c|c|c|}
    \hline
                     & \textbf{Central Frequency}\superscript{a} & \textbf{Sideband 1 Offset}\superscript{a} & \textbf{Sideband 2 Offset}\superscript{a} & \textbf{Bandwidth}\superscript{b} \\
    \textbf{Channel} & \bfrequency{0}             & \sideband{1}               & \sideband{2}               & $\Delta f$         \\
                     & (GHz)                      & (GHz)                      & (GHz)                      & (MHz)              \\
    \hline\hline
            1        &           23.800000        & 0.0                        & 0.0                        & 0.258         \\
            2        &           31.400000        & 0.0                        & 0.0                        & 0.172         \\
            3        &           50.300000        & 0.0                        & 0.0                        & 0.173         \\
            4        &           51.760000        & 0.0                        & 0.0                        & 0.381         \\
            5        &           52.800000        & 0.0                        & 0.0                        & 0.366         \\
            6        &           53.596000        & 0.115                      & 0.0                        & 0.1587,0.1648\superscript{c} \\
            7        &           54.400000        & 0.0                        & 0.0                        & 0.387         \\
            8        &           54.940000        & 0.0                        & 0.0                        & 0.387         \\
            9        &           55.500000        & 0.0                        & 0.0                        & 0.317         \\
           10        &           57.290344        & 0.0                        & 0.0                        & 0.151         \\
           11        &           57.290344        & 0.217                      & 0.0                        & 0.0763        \\
           12        &           57.290344        & 0.3222                     & 0.048                      & 0.0351        \\
           13        &           57.290344        & 0.3222                     & 0.022                      & 0.01547       \\
           14        &           57.290344        & 0.3222                     & 0.010                      & 0.0078,0.0079\superscript{c} \\
           15        &           57.290344        & 0.3222                     & 0.0045                     & 0.0029        \\
           16        &           88.200000        & 0.0                        & 0.0                        & 1.9282        \\
           17        &          165.500000        & 0.0                        & 0.0                        & 1.1251        \\
           18        &          183.310000        & 7.0                        & 0.0                        & 1.9302        \\
           19        &          183.310000        & 4.5                        & 0.0                        & 1.9519        \\
           20        &          183.310000        & 3.0                        & 0.0                        & 0.9799        \\
           21        &          183.310000        & 1.8                        & 0.0                        & 0.9823        \\
           22        &          183.310000        & 1.0                        & 0.0                        & 0.4940        \\
    \hline
  \end{tabular}
  \caption{Central, sideband offset, and bandwidth frequencies for ATMS. \superscript{a}Data from \citet{Muth_2004}. \superscript{b}Data from \cite{ATMS_PFM_CalLog}. \superscript{c}Different lower and upper sideband widths reported. }
  \label{tab:atms_fo_sb_and_df}
\end{table}
Typically, it is assumed the upper and lower sideband bandwidths are the same for the double and quadruple sideband channels. However, for ATMS channels 6 and 14, different bandwidth for the sidebands were reported (e.g. the lower and upper sidebands for channel 6 differ by about 3.7\%). Thus, rather than determining the intermediate frequency ranges (under the assumption of bandwidth symmetry about the central frequency or sideband offsets), the band edge frequencies relative to the channel central frequencies will be specified, as shown in figure \ref{fig:broadband_frequency_translation}. For the single passband channels, assuming no stopbands), the frequency range $f_1 \rightarrow f_2$ is simply given by,
\begin{eqnarray*}
  f_1 & = & f_0 - \frac{\Delta f}{2} \\
  f_2 & = & f_0 + \frac{\Delta f}{2}
\end{eqnarray*}
The band edge frequency ranges for the ATMS single passband channels are shown in table \ref{tab:atms_single_f}.
\begin{table}[htp]
  \centering
  \begin{tabular}{|c|c|}
    \hline
    \textbf{Channel} & \bfrequency{1}$\rightarrow$\bfrequency{2} \\
                     & (GHz) \\
    \hline\hline
    1   &    23.671000 - 23.929000  \\  
    2   &    31.314000 - 31.486000  \\  
    3   &    50.213500 - 50.386500  \\  
    4   &    51.569500 - 51.950500  \\  
    5   &    52.617000 - 52.983000  \\  
    7   &    54.206500 - 54.593500  \\  
    8   &    54.746500 - 55.133500  \\  
    9   &    55.341500 - 55.658500  \\  
    10  &    57.214844 - 57.365844  \\  
    16  &    87.235900 - 89.164100  \\  
    17  &    164.93745 - 166.06255  \\
    \hline
  \end{tabular}
  \caption{Computed band edge frequencies for the ATMS single passband channels}
  \label{tab:atms_single_f}
\end{table}

The frequency ranges for the double sideband channels are computed assuming the bandwidths for the lower and upper sidebands (as defined in figure \ref{fig:double_sideband}) are different,
\begin{eqnarray*}
  f_{L1} & = & f_0 - df_1 - \frac{\Delta f_L}{2} \\
  f_{L2} & = & f_0 - df_1 + \frac{\Delta f_L}{2} \\
  f_{U1} & = & f_0 + df_1 - \frac{\Delta f_U}{2} \\
  f_{U2} & = & f_0 + df_1 + \frac{\Delta f_U}{2}
\end{eqnarray*}
The band edge frequency ranges for the ATMS double sideband channels are shown in table \ref{tab:atms_double_f}.
\begin{table}[htp]
  \centering
  \begin{tabular}{|c|c|c|}
    \hline
    \textbf{Channel} & \bfrequency{L1}$\rightarrow$\bfrequency{L2} & \bfrequency{U1}$\rightarrow$\bfrequency{U2} \\
                     & (GHz) & (GHz) \\
    \hline\hline
    6   &    53.401650 - 53.560350   &   53.628600 - 53.793400 \\
    11  &    57.035194 - 57.111494   &   57.469194 - 57.545494 \\
    18  &    175.34490 - 177.27510   &   189.34490 - 191.27510 \\
    19  &    177.83405 - 179.78595   &   186.83405 - 188.78595 \\
    20  &    179.82005 - 180.79995   &   185.82005 - 186.79995 \\
    21  &    181.01885 - 182.00115   &   184.61885 - 185.60115 \\
    22  &    182.06300 - 182.55700   &   184.06300 - 184.55700 \\
    \hline
  \end{tabular}
  \caption{Computed band edge frequencies for the ATMS double sideband channels}
  \label{tab:atms_double_f}
\end{table}

For the quadruple sideband channels, we begin to run into nomenclature issues. For the purposes of this document, it will be assumed that the lower and upper sidebands shown in figure \ref{fig:quadruple_sideband} are symmetric about the central frequency, $f_0$. That is, the bandwidths of the outermost (furtherest from $f_0$) sidebands of figure \ref{fig:quadruple_sideband} are both given by $\Delta f_U$, and the bandwidths of the innermost (closest to $f_0$) sidebands of figure \ref{fig:quadruple_sideband} are both given by $\Delta f_L$. To distinguish between sidebands less than and greater than $f_0$, the subscripts - and + shall be used respectively. The frequency ranges for the quadruple sideband channels are then given by,
\begin{eqnarray*}
  f_{U1-} & = & f_0 - df_1 - df_2 - \frac{\Delta f_U}{2} \\
  f_{U2-} & = & f_0 - df_1 - df_2 + \frac{\Delta f_U}{2} \\
  f_{L1-} & = & f_0 - df_1 + df_2 - \frac{\Delta f_L}{2} \\
  f_{L2-} & = & f_0 - df_1 + df_2 + \frac{\Delta f_L}{2} \\
  f_{L1+} & = & f_0 + df_1 + df_2 - \frac{\Delta f_L}{2} \\
  f_{L2+} & = & f_0 + df_1 + df_2 + \frac{\Delta f_L}{2} \\
  f_{U1-} & = & f_0 + df_1 - df_2 - \frac{\Delta f_U}{2} \\
  f_{U2-} & = & f_0 + df_1 - df_2 + \frac{\Delta f_U}{2}   
\end{eqnarray*}
The band edge frequency ranges for the ATMS quadruple sideband channels are shown in table \ref{tab:atms_quadruple_f}.
\begin{table}[htp]
  \centering
  \begin{tabular}{|c|c|c|c|c|}
    \hline
    \textbf{Channel} & \bfrequency{U1-}$\rightarrow$\bfrequency{U2-} & \bfrequency{L1-}$\rightarrow$\bfrequency{L2-} & \bfrequency{L1+}$\rightarrow$\bfrequency{L2+} & \bfrequency{U1+}$\rightarrow$\bfrequency{U2+} \\
       & (GHz)     & (GHz)     & (GHz)     & (GHz) \\
    \hline\hline
    12 & 56.902594 - 56.937694 & 56.998594 - 57.033694 & 57.546994 - 57.582094 & 57.642994 - 57.678094 \\
    13 & 56.938409 - 56.953879 & 56.982409 - 56.997879 & 57.582809 - 57.598279 & 57.626809 - 57.642279 \\
    14 & 56.954194 - 56.962094 & 56.974244 - 56.982044 & 57.598644 - 57.606444 & 57.618594 - 57.626494 \\
    15 & 56.962194 - 56.965094 & 56.971194 - 56.974094 & 57.606594 - 57.609494 & 57.615594 - 57.618494 \\
    \hline
  \end{tabular}
  \caption{Computed band edge frequencies for the ATMS quadruple sideband channels}
  \label{tab:atms_quadruple_f}
\end{table}

\subsection{Measured ATMS Response Functions}
Response data in loss of decibels was obtained from \cite{Muth_etal_2004}. We convert the response data in loss of decibels to relative responses by, 
\begin{equation}
  r=10^{\textbf{-0.1*Loss(DB)}}    
\end{equation}
We calculate central frequencies for the passbands by taking the first moment of the relative response data for each passband. These passband central frequencies will hereafter be referred to as the "official" calculated passband central frequencies. For comparison purposes only we calculated passband central frequencies by taking the average of the two half maximum frequency points that define the full width half maximum "FWHM" of each passband. These passband central frequencies will hereafter be referred to as the "FWHM" calculations. 

%We calculate two sets of channel and band central frequencies for the response data provided. One set of band and channel central frequencies are 
%calculated using the full width half maximum points of the channels. For this set the IF central frequency of 

%For channels 1 and 2 response data is provided at the measurement frequencies. We calculate two sets of central frequencies for these single passband
%channels. One set of channel central frequencies are calculated using the full width half maximum points of the single passbands. For this set the central frequency of a single passband is calculated by taking an average of the full width half maximum points associated with the passbnd. A second set of central frequencies for these channels is calculated by taking the first moment of the relative response data. The differences between documented channel central frequencies and calculated channel central frequencies for 
%are shown in table \ref{tab:atms_measured_diff} for both sets of calculations.
 
For channels 3-22 the relative response data are provided at intermediate frequencies. 
To construct the measurement frequencies from intermediate frequencies we assign channels 3-22 to two categories. 

\begin{itemize}
  \item{Unfolded IF Channels (3-10, and 16-17)}
  \item{Folded IF Channels (11-15, and 18-22)}  
\end{itemize}

For unfolded IF channels the relationships between measurement frequencies and intermediate frequencies are shown in
figures 2.1 and 2.2. Note that for the unfolded IF channels the relationships between the measurement frequencies
and intermediate frequencies are dependent on both the IF channel central frequencies and the RF channel central frequencies.
With the exception of channel 6 all the unfolded channels are single passband. In the particular case of channel 6 the FWHM calculation
is the average of the half maximum frequency points that are adjacent to the IF channel central frequency. 
Differences between the documented and official IF passband central frequencies, and between the documented and FWHM IF passband central frequencies
are shown in Table \ref{tab:atms_single_Unfolded} for the unfolded single passband channels. Differences between the documented and official IF channel central frequencies, and between the documented and FWHM central frequencies are shown in Table \ref{tab:atms_double_Unfolded} for channel 6 which is an unfolded double passband channel.

\begin{table}[htp]
  \centering
  \begin{tabular}{|c|c|c|c|}
    \hline
    \textbf{Channel} & $\textbf{Documented }$\bfrequency{o} & $\textbf{FWHM Average } ${\textbfm{\Delta}}\bfrequency{o} & $\textbf{Official } ${\textbfm{\Delta}}\bfrequency{o}  \\
    & (GHz)  & (GHz)   & (GHz) \\               
    \hline\hline 
    3   &    6.990344   &   0.000392  &   0.000363 \\  
    4   &    5.530344   &   0.001315  &   0.003213 \\  
    5   &    4.490344   &  -0.001025  &  -0.001701 \\  
    7   &    2.890344   &   0.001093  &   0.002429 \\ 
    8   &    2.350344   &   0.000927  &   0.002215 \\ 
    9   &    1.790344   &   0.001194  &   0.002537 \\ 
    10  &    0.087500   &   0.000059  &   0.003473 \\ 
    16  &    5.450000   &  -0.001355  &  -0.000477 \\ 
    17  &    0.925000   &   0.000477  &   0.000858 \\
    \hline
  \end{tabular}
  \caption{Differences between documented intermediate passband central frequencies and calculated intermediate passband central frequencies.}
  \label{tab:atms_single_Unfolded}
\end{table}

\begin{table}[htp]
  \centering
  \begin{tabular}{|c|c|c|c|}
    \hline
    \textbf{Channel} & $\textbf{Documented }$\bfrequency{o} & $\textbf{FWHM Average } ${\textbfm{\Delta}}\bfrequency{o} & $\textbf{Official } ${\textbfm{\Delta}}\bfrequency{o}  \\
    & (GHz)  & (GHz)   & (GHz) \\               
    \hline\hline 
    6   &    0.000000   &   0.00000  &   0.000000 \\ 
    \hline
  \end{tabular}
  \caption{Differences between documented intermediate passband central frequencies and calculated intermediate passband central frequencies.}
  \label{tab:atms_double_Unfolded}
\end{table}

For folded IF channels the relationships between measurement frequencies and intermediate frequencies are shown in
figures 2.3 and 2.4. Note that for the folded IF channels the relationships between the measurement frequencies and intermediate frequencies are only dependent on the RF channel central frequencies. The RF channel central frequencies are obtained from [source 2] and cannot be corroborated for channels 3-22 since measured RF data is not available. For folded double passband channels the calculated IF passband central frequency is the sideband 1 offset. For folded quadruple passband channels the central frequency of the IF relative responses is needed to calculate the sideband 2 offsets. Differences between the documented and official IF offsets, and between the documented and "FWHM Average" IF offsets are shown in tables \ref{tab:atms_folded_offset1} and \ref{tab:atms_folded_offset2} for the folded multiple sideband channels. 

\begin{table}[htp]
  \centering
  \begin{tabular}{|c|c|c|c|}
    \hline
    \textbf{Channel} & $\textbf{Documented d}$\bfrequency{}$\textbf{1}$ & $\textbf{FWHM Average } ${\textbfm{\Delta}}$\textbf{d}$\bfrequency{1} & $\textbf{Official } ${\textbfm{\Delta}}$\textbf{d}$\bfrequency{1} \\   
    & (GHz)  & (GHz)   & (GHz) \\               
    \hline\hline
    11  &  0.217000   &   0.001326  & -0.000468 \\
    12  &  0.322200   &  -0.000069  & -0.001645 \\
    13  &  0.322200   &   0.000377  &  0.001994 \\
    14  &  0.322200   &  -0.000049  &  0.000422 \\
    15  &  0.322200   &   0.000020  &  0.000023 \\
    18  &  7.000000   &   0.004888  &  0.001126 \\
    19  &  4.500000   &   0.595120  &  0.586855 \\
    20  &  3.000000   &  -0.000374  &  0.000633 \\
    21  &  1.800000   &  -0.002576  &  0.000029 \\
    22  &  1.000000   &  -0.003879  & -0.000987 \\
    \hline
  \end{tabular}
  \caption{Differences between documented intermediate sideband 1 offsets and calculated intermediate sideband 1 offsets.}
  \label{tab:atms_folded_offset1}
\end{table}

%    6   &  0.115000   &   0.000134  &  0.000345 \\
\begin{table}[htp]
  \centering
  \begin{tabular}{|c|c|c|c|}
    \hline
     \textbf{Channel} & $\textbf{Documented d}$\bfrequency{2} & $\textbf{FWHM Average } ${\textbfm{\Delta}}$\textbf{d}$\bfrequency{2} &  $\textbf{Official } ${\textbfm{\Delta}}$\textbf{d}$\bfrequency{2}\\   
    & (GHz)  & (GHz)   & (GHz) \\                   
    \hline\hline
    12 & 0.048000 & -0.000026 &  0.001731  \\
    13 & 0.022000 &  0.000122 &  -0.003429 \\
    14 & 0.010000 &  0.000003 &  -0.000973 \\
    15 & 0.004500 &  0.000026 &  0.000005  \\
    \hline
  \end{tabular}
  \caption{Differences between documented intermediate sideband 2 offsets and calculated intermediate sideband 2 offsets.}
  \label{tab:atms_folded_offset2}
\end{table}  

In the case of channel 6 two sideband 1 offsets are calculated, because the channel is an unfolded double passband. Differences between the left and right sideband 1 offsets 
for both the official and FWHM calculations are shown in table \ref{tab:channel_6_offsets}. 

\begin{table}[htp]
  \centering
  \begin{tabular}{|c|c|c|}
    \hline
    \textbf{Channel} & \textbf{FWHM Average} $\textbf{d}$\bfrequency{1}$\textbf{Left - Right}$ & \textbf{Official} $\textbf{d}$\bfrequency{1}$\textbf{Left - Right}$ \\
    & (GHz)   & (GHz) \\               
    \hline\hline
    6   &  -0.000980  &  -0.002101 \\
    \hline
  \end{tabular}
  \caption{Differences between the left and right sideband 1 offsets for channel 6. The differences are shown for 
    both "FWHM Average" and "Official" calculations.}
  \label{tab:channel_6_offsets}
\end{table}



For single passband channels 1 and 2 the response data is provided at measurement frequencies. The differences between documented RF channel central frequencies and calculated RF channel central frequencies for these channels are shown in table \ref{tab:atms_MF_diff}.  

\begin{table}[htp]
  \centering
  \begin{tabular}{|c|c|c|c|}
    \hline
    \textbf{Channel} & $\textbf{Documented }$\bfrequency{o} & $\textbf{FWHM Average } ${\textbfm{\Delta}}\bfrequency{o} & $\textbf{Official } ${\textbfm{\Delta}}\bfrequency{o}  \\   
    & (GHz)  & (GHz)   & (GHz) \\               
    \hline\hline
    1   &    23.799999  &  -0.005694  &  -0.004027 \\  
    2   &    31.400000  &  -0.001022  &  -0.001864 \\  
    \hline
  \end{tabular}
  \caption{Differences between documented RF channel central frequencies and calculated RF channel central frequencies.}
  \label{tab:atms_MF_diff}
\end{table}  


%are shown in table \ref{tab:atms_measured_diff} for both sets of calculations. 

%Differences between the documented and official IF offsets, and between the documented and "FWHM Average" offsets are shown in Table 2.5 for channel 6 which is an unfolded double passband channel. 
  
%We calculate two sets of channel central frequencies for the unfolded IF channels. One set of IF band central frequencies are 
%calculated using the full width half maximum points of the channels. For this set the IF central frequency of a channel is calculated by taking an average of the full width half maximum points associated with the channel. A second set of IF channel central frequencies are calculated by taking the first moment of the relative response data. The differences between the calculated IF channel central frequencies and the documented IF channel central frequencies are shown in table 2 for both sets of calculations.   
%
%We calculate two sets of band central frequencies for the folded IF channels and channel 6. One set of IF band central frequencies are calculated using the full width half maximum points of the bands. For this set the IF central frequency of a band is calculated by taking an average of the full width half maximum points associated with the band.  A second set of IF band central frequencies are calculated by taking the first moment of the relative response data. The IF band central frequencies determine the offsets for the folded channels. These offsets are used to construct reflected bands with respect to the RF channel central frequency.  The differences between the calculated IF band central frequencies and the documented IF band central frequencies are shown in table 2 for both
%sets of calculations. 
%
%With the exception of channel 6 the unfolded channels are single passband channels. For this reason it is necessary to calculate IF band and IF channel central frequencies to construct the channel 6 measured frequencies. Furthermore, the sideband 1 offsets to the left and right of the central frequency as depicted in figure 2.3 may differ, because the band central frequencies for both bands are calculated for channel 6. 


  
%  Tables \ref{tab:atms_single_df0} - \ref{tab:atms_quadruple_df10} show differences between documented intermediate offsets and central frequencies \cite{Muth_etal_2004}, and the intermediate offsets and central frequencies determined using
%  method1 and method2.  %For channel 19 the documented offsets are significantly shifted from the offsets calculated for method 1 and method2. 
  
   
    
  %The central frequencies, band offset information (df1 and df2) and badwidth values for the channels using both methods are shown in tables 2.2, 2.3 and 2.4.
    
%  \begin{table}[htp]
%  \centering
%  \begin{tabular}{|c|c|c|c|}
%    \hline
%    \textbf{Channel} & $\textbf{Documented }$\bfrequency{o} & $\textbf{Method1 } ${\textbfm{\Delta}}\bfrequency{o} & $\textbf{Method2 } ${\textbfm{\Delta}}\bfrequency{o}  \\   
%    & (GHz)  & (GHz)   & (GHz) \\               
%    \hline\hline
%    1   &    23.799999  &  -0.005694  &  -0.004027 \\  
%    2   &    31.400000  &  -0.001022  &  -0.001864 \\  
%    3   &    6.990344   &   0.000392  &   0.000363 \\  
%    4   &    5.530344   &   0.001315  &   0.003213 \\  
%    5   &    4.490344   &  -0.001025  &  -0.001701 \\  
%    7   &    2.890344   &   0.001093  &   0.002429 \\ 
%    8   &    2.350344   &   0.000927  &   0.002215 \\ 
%    9   &    1.790344   &   0.001194  &   0.002537 \\ 
%    10  &    0.087500   &   0.000059  &   0.003473 \\ 
%    16  &    5.450000   &  -0.001355  &  -0.000477 \\ 
%    17  &    0.925000   &   0.000477  &   0.000858 \\
%    \hline
%  \end{tabular}
%  \caption{Differences between documented intermediate central frequencies and intermediate central frequencies for the two methods.}
%  \label{tab:atms_single_df0}
%\end{table}



  
%\begin{table}[htp]
%  \centering
%  \begin{tabular}{|c|c|c|c|}
%    \hline
%    \textbf{Channel} & $\textbf{Documented d}$\bfrequency{1} & $\textbf{Method1 } ${\textbfm{\Delta}}$\textbf{d}$\bfrequency{1} &  $\textbf{Method2 } ${\textbfm{\Delta}}$\textbf{d}$\bfrequency{1}  \\   
%    & (GHz)  & (GHz)   & (GHz) \\                   
%    \hline\hline
%    12 & 0.322200 & -0.000069 &  -0.001645 \\
%    13 & 0.322200 &  0.000377 &   0.001994 \\
%    14 & 0.322200 & -0.000049 &   0.000422 \\
%    15 & 0.322200 &  0.000020 &   0.000023 \\
%    \hline
%  \end{tabular}
%  \caption{Differences between documented intermediate sideband 1 offsets and the intermediate sideband 1 offsets for the two methods.}
%  \label{tab:atms_quadruple_df10}
%\end{table}     
 % Using the calculated relative responses we characterize ATMS response functions by applying two methods. The construction of the measurement   frequencies from the 
%  intermediate frequencies for the two methods is the same as described in section 2.1. 
%  In the first method we determine the band edges by finding the half width half maximum (HWHM) points. The central frequencies using this method are specified to
%  be points which are equidistant from the band edges. (Equation 1)
%  
%  For the second method we consider all significant relative responses provided. The central frequencies in this characterization are 
%  the first moment of the relative response data. (Equation 2)
%  The band edges are determined by the last data point we interpret to be significant. Data points beyond
%  relative minima at the band edges are considered insignificant.
%  
%  Tables 2.2, 2.3 and 2.4 show comparisons between documented offsets and central frequencies, and the offsets and central frequencies determined using
%  the two methods.  
%    
%  The central frequencies, band offset information (df1 and df2) and badwidth values for the channels using both methods are shown in tables 2.2, 2.3 and 2.4.
%  
%  The plots of the channel and band responses with the central frequencies indicated are shown in appendix A 
%  
  
%\end{table}

