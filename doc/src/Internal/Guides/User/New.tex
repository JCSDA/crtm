\chapter*{What's New in v2.0}
%===========================
\addcontentsline{toc}{chapter}{What's New in v2.0}

\section*{New Science}
%---------------------
\addcontentsline{toc}{section}{New Science}

\begin{description}
\item[Multiple transmittance algorithms] There are now two transmittance models available for use in the CRTM: ODAS (Optical Depth in Absorber Space), which is equivalent to the previous CompactOPTRAN algorithm; and ODPS (Optical Depth in Pressure Space) which is similar to the RTTOV-type of transmittance algorithm, except here OPTRAN is used for water vapor line absorption.

The algorithm is selectable by the user via the transmittance coefficient (\f{TauCoeff}) data file used to initialise the CRTM. This method, rather than a switch argument in the \hyperref[sec:CRTM_Init_interface]{\f{CRTM\_Init()}} function, was chosen to allow users to ``mix-and-match'' transmittance algorithms for different sensors in the same initialisation call.

\item[SSU-specific transmittance model] Similar to the multiple transmittance algorithm approach, a separate algorithm \emph{just} for the SSU instrument has been constructed. The algorithm is based on the ODAS approach, but with elements to account for the time-dependence of the SSU CO\subscript{2} cell pressures.

\item[Zeeman-splitting transmittance model for SSMIS upper-level channels] A separate algorithm is available to account for the change in absorption at very low pressures due to the Zeeman-splitting of absorption lines. Currently this algorithm has only been applied to the affected channels in the SSMIS instrument, 19-22.

\item[Visible sensor capability] The CRTM now supports radiative transfer for visible instruments/channels. The treatment of visible channels was handled in the CRTM framework by considering them separate instruments. The sensor identifier for these instruments/channels are differentiated from their infrared counterparts by a ``\f{v.}'' prefix. For example, while \f{modis\_aqua} is the sensor identifier for the infrared channels, \f{v.modis\_aqua} identifies the visible channels.

\item[Inclusion of Matrix Operator Method (MOM) in radiaitve transfer] To handle visible wavelength radiative transfer in the prescence of aerosols, the Advanced Doubling-Adding (ADA) algorithm was adapted to use the MOM technique \citep{Liu_1996}.

\item[Inclusion of additional infrared sea surface emissivity model] Files containing the emissivity data (\f{EmisCoeff}) for the \citet{Nalli_2008a} model are provided. Previously, only the \f{EmisCoeff} files for the \citet{WuSmith_1997} model were provided. Users can now select between the \citet{Nalli_2008a} or \citet{WuSmith_1997} models by specifying the requisite filename in the call to \hyperref[sec:CRTM_Init_interface]{\f{CRTM\_Init()}}.

\item[Surface BRDF for solar-affected shortwave IR channels] A bi-directional reflectance distribution function (BRDF) has been added to account for reflected solar in affected shortware infrared channels \citep{Breon_1993}.

\item[Reflectivity for downwelling infrared over water] The reflectivity for downwelling infrared  radiation over water surface has been changed from Lambertian to specular.

\item[Aerosol type changes] To account for changes in the handling of GOCART \citep{Chin_2002} aerosol model output, additional sea salt coarse modes were added to the list of allowed aerosol types. Also, the separate dry and wet types for organic and black carbon aerosols were combined, with a relatvie humidity of 0\% used to indicate the previous ``dry'' aerosol type. See table \ref{tab:aerosol_type} for the new list of accepted aerosol types.

\end{description}


\section*{Interface Changes}
%---------------------------
\addcontentsline{toc}{section}{Interface Changes}

\begin{description}
\item[CRTM Initialisation function] The changes to the \hyperref[sec:CRTM_Init_interface]{\f{CRTM\_Init()}} interface were relatively minor but do require calling codes to be modified:
  \begin{itemize}
  \item The \f{Sensor\_Id} argument is now mandatory. This argument is used to construct the sensor-specific \f{SpcCoeff} and \f{TauCoeff} filename and in the past was optional to allow for ``generic'' filenames. This is no longer allowed and generic \f{SpcCoeff} and \f{TauCoeff} files are no longer used.
  \item The loading of the \f{CloudCoeff} and \f{AerosolCoeff} datafiles containing the optical properties of cloud and aerosol particulates is no longer mandatory. For cloud-free CRTM runs, the load of the \f{CloudCoeff} and \f{AerosolCoeff} datafiles can be disabled via the optional \f{Load\_CloudCoeff} and \f{Load\_AerosolCoeff} arguments which are logical switches (true or false).
  \end{itemize}

\item[User accessible structures] The structures are defined as those that are used in the argument lists of the main CRTM functions (e.g. initialisation; the forward, tangent-linear, adjoint, and K-matrix models; and destruction). Changes were made to both the structure definitions and their procedures. To mitigate the possibility of memory leaks, the definitions of array members of structures have had their \f{POINTER} attribute replaced with \f{ALLOCATABLE}. This was a first step in preparation for use of Fortran2003 Object Oriented features in the CRTM (once Fortran2003 compiler become widely available), where the derived type structure definitions will be reclassified as objects and their procedures will be type-bound. To delineate this change from previous versions of CRTM the interfaces of the derived type procedures have been altered by:
  \begin{itemize}
  \item changing the procedure names to use the convention \f{CRTM\_}\textit{object}\f{\_}\textit{action} where an \textit{object} can be any of the user accessible CRTM derived types (e.g. \f{CRTM\_Atmosphere\_type}, \f{CRTM\_RTSolution\_type} etc), and the \textit{action} can be those defined operations for the structure (e.g. \f{Create}, \f{Destroy}, \f{Inspect}, etc).
  \item making the first dummy argument of the definition module procedures the derived type itself. This will eventually allow the procedures to be called via an instance of the derived type\footnote{Interested readers can investigate the \f{PASS} attribute that can be used in the \f{PROCEDURE} statement within derived type definitions in Fortran2003.}\footnote{The I/O functions do not yet follow this convention, since they are considered secondary to the definition module procedures used to manipulate the derived types.}
  \end{itemize}
  All of the current derived type definitions and their associated procedures and interfaces are shown in appendix \ref{sec:structure_and_interface_definition}.

\item[\Options{} structure specific changes] The additional changes made to the \hyperref[sec:options_structure]{\f{CRTM\_Options\_type}} definition:
  \begin{itemize}
  \item all usage on/off switches have been changed from integers (0/1) to logicals (true/false),
  \item a logical switch to control input checking, \f{Check\_Input}, has been added.
  \item structure components for \hyperref[sec:ssu_input_structure]{SSU-specific} and \hyperref[sec:zeeman_input_structure]{Zeeman model} input have been added.
  \end{itemize}
\end{description}
