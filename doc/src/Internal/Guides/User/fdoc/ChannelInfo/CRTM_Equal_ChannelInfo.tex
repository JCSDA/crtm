\subsection{\texttt{CRTM\_Equal\_ChannelInfo} interface}
  \label{sec:CRTM_Equal_ChannelInfo_interface}
  \begin{alltt}
 
  NAME:
        CRTM_Equal_ChannelInfo
 
  PURPOSE:
        Function to test if two ChannelInfo structures are equal.
 
  CALLING SEQUENCE:
        Error_Status = CRTM_Equal_ChannelInfo( ChannelInfo_LHS        , &
                                               ChannelInfo_RHS        , &
                                               ULP_Scale  =ULP_Scale  , &
                                               Check_All  =Check_All  , &
                                               Message_Log=Message_Log  )
 
 
  INPUT ARGUMENTS:
        ChannelInfo_LHS: ChannelInfo structure to be compared; equivalent
                         to the left-hand side of a lexical comparison, e.g.
                           IF ( ChannelInfo_LHS == ChannelInfo_RHS ).
                         UNITS:      N/A
                         TYPE:       CRTM_ChannelInfo_type
                         DIMENSION:  Scalar OR Rank-1 array
                         ATTRIBUTES: INTENT(IN)
 
        ChannelInfo_RHS: ChannelInfo structure to be compared to; equivalent
                         to the right-hand side of a lexical comparison, e.g.
                           IF ( ChannelInfo_LHS == ChannelInfo_RHS ).
                         UNITS:      N/A
                         TYPE:       CRTM_ChannelInfo_type
                         DIMENSION:  Same as ChannelInfo_LHS argument
                         ATTRIBUTES: INTENT(IN)
 
  OPTIONAL INPUT ARGUMENTS:
        ULP_Scale:      Unit of data precision used to scale the floating
                        point comparison. ULP stands for "Unit in the Last Place,"
                        the smallest possible increment or decrement that can be
                        made using a machine's floating point arithmetic.
                        Value must be positive - if a negative value is supplied,
                        the absolute value is used. If not specified, the default
                        value is 1.
                        ** NOTE: This is a hook for future changes and is not used.
                        UNITS:      N/A
                        TYPE:       INTEGER
                        DIMENSION:  Scalar
                        ATTRIBUTES: INTENT(IN), OPTIONAL
 
        Check_All:      Set this argument to check ALL the floating point
                        channel data of the ChannelInfo structures. The default
                        action is return with a FAILURE status as soon as
                        any difference is found. This optional argument can
                        be used to get a listing of ALL the differences
                        between data in ChannelInfo structures.
                        If == 0, Return with FAILURE status as soon as
                                 ANY difference is found  *DEFAULT*
                           == 1, Set FAILURE status if ANY difference is
                                 found, but continue to check ALL data.
                        UNITS:      N/A
                        TYPE:       INTEGER
                        DIMENSION:  Scalar
                        ATTRIBUTES: INTENT(IN), OPTIONAL
 
        Message_Log:    Character string specifying a filename in which any
                        messages will be logged. If not specified, or if an
                        error occurs opening the log file, the default action
                        is to output messages to standard output.
                        UNITS:      None
                        TYPE:       CHARACTER(*)
                        DIMENSION:  Scalar
                        ATTRIBUTES: INTENT(IN), OPTIONAL
 
  FUNCTION RESULT:
        Error_Status:   The return value is an integer defining the error status.
                        The error codes are defined in the Message_Handler module.
                        If == SUCCESS the structures were equal
                           == FAILURE - an error occurred, or
                                      - the structures were different.
                        UNITS:      N/A
                        TYPE:       INTEGER
                        DIMENSION:  Scalar
 
  \end{alltt}
