\subsection{\texttt{CRTM\_Assign\_Aerosol} interface}
  \label{sec:CRTM_Assign_Aerosol_interface}
  \begin{alltt}
 
  NAME:
        CRTM_Assign_Aerosol
 
  PURPOSE:
        Function to copy valid CRTM_Aerosol structures.
 
  CALLING SEQUENCE:
        Error_Status = CRTM_Assign_Aerosol( Aerosol_in             , &
                                            Aerosol_out            , &
                                            Message_Log=Message_Log  )
 
  INPUT ARGUMENTS:
        Aerosol_in:      CRTM_Aerosol structure which is to be copied.
                         UNITS:      N/A
                         TYPE:       CRTM_Aerosol_type
                         DIMENSION:  Scalar OR Rank-1 array
                         ATTRIBUTES: INTENT(IN)
 
  OUTPUT ARGUMENTS:
        Aerosol_out:     Copy of the input structure, CRTM_Aerosol_in.
                         UNITS:      N/A
                         TYPE:       CRTM_Aerosol_type
                         DIMENSION:  Same as Aerosol_in argument
                         ATTRIBUTES: INTENT(IN OUT)
 
 
  OPTIONAL INPUT ARGUMENTS:
        Message_Log:     Character string specifying a filename in which any
                         messages will be logged. If not specified, or if an
                         error occurs opening the log file, the default action
                         is to output messages to standard output.
                         UNITS:      N/A
                         TYPE:       CHARACTER(*)
                         DIMENSION:  Scalar
                         ATTRIBUTES: INTENT(IN), OPTIONAL
 
  FUNCTION RESULT:
        Error_Status:    The return value is an integer defining the error status.
                         The error codes are defined in the Message_Handler module.
                         If == SUCCESS the structure assignment was successful
                            == FAILURE an error occurred
                         UNITS:      N/A
                         TYPE:       INTEGER
                         DIMENSION:  Scalar
 
  COMMENTS:
        Note the INTENT on the output Aerosol argument is IN OUT rather than
        just OUT. This is necessary because the argument may be defined upon
        input. To prevent memory leaks, the IN OUT INTENT is a must.
 
  \end{alltt}
