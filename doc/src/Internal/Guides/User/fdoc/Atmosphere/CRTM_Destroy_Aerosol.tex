\subsection{\texttt{CRTM\_Destroy\_Aerosol} interface}
  \label{sec:CRTM_Destroy_Aerosol_interface}
  \begin{alltt}
 
  NAME:
        CRTM_Destroy_Aerosol
  
  PURPOSE:
        Function to re-initialize the scalar and pointer members of
        a CRTM_Aerosol data structure.
 
  CALLING SEQUENCE:
        Error_Status = CRTM_Destroy_Aerosol( Aerosol                , &
                                             Message_Log=Message_Log  )
 
  OPTIONAL INPUT ARGUMENTS:
        Message_Log:  Character string specifying a filename in which any
                      messages will be logged. If not specified, or if an
                      error occurs opening the log file, the default action
                      is to output messages to standard output.
                      UNITS:      N/A
                      TYPE:       CHARACTER(*)
                      DIMENSION:  Scalar
                      ATTRIBUTES: INTENT(IN), OPTIONAL
 
  OUTPUT ARGUMENTS:
        Aerosol:      Re-initialized CRTM_Aerosol structure.
                      UNITS:      N/A
                      TYPE:       CRTM_Aerosol_type
                      DIMENSION:  Scalar OR Rank-1 array
                      ATTRIBUTES: INTENT(IN OUT)
 
  FUNCTION RESULT:
        Error_Status: The return value is an integer defining the error status.
                      The error codes are defined in the Message_Handler module.
                      If == SUCCESS the structure re-initialisation was successful
                         == FAILURE - an error occurred, or
                                    - the structure internal allocation counter
                                      is not equal to zero (0) upon exiting this
                                      function. This value is incremented and
                                      decremented for every structure allocation
                                      and deallocation respectively.
                      UNITS:      N/A
                      TYPE:       INTEGER
                      DIMENSION:  Scalar
 
  COMMENTS:
        Note the INTENT on the output Aerosol argument is IN OUT rather than
        just OUT. This is necessary because the argument may be defined upon
        input. To prevent memory leaks, the IN OUT INTENT is a must.
 
  \end{alltt}
